\documentclass{article}

\usepackage{notes}

\title{GRT Seminar Fall 2024 -- Equivariant Mathematics}
\author{Notes by John S.\ Nolan, speakers listed below}

\begin{document}

\maketitle

\begin{abstract}
	This semester, the GRT Seminar will focus on equivariant mathematics (especially homotopy theory).
  The goal is not necessarily to become homotopy theorists but rather to gain an appreciation for one of the most developed approaches to equivariance.
\end{abstract}

\tableofcontents

\section{(8/28) David Nadler -- $S^1$-Actions and Koszul Duality}

A good reference for this material is GKM (Goresky--Kottwitz--Macpherson, ``Equivairant cohomology, Koszul duality, and the localization theorem'').
We'll focus on the case of $G = S^1$ acting on a (reasonable) topological space $X$, e.g.\ a CW complex.

\subsection{Linearization}

We can ``linearize'' the space $X$ by taking cochains $C^*(X)$, say with coefficients in $k = \QQ$.
Morally, this is like passing to ``functions'' on a space, except that in topology our notion of function is ``locally constant.''
This is not as uninteresting as it sounds, since we also include ``derived'' information.

Following this analogy, ``distributions'' on $S^1$ should act on ``functions'' on $X$ by convolution.
The right notion of ``distributions'' on $S^1$ is captured by the chain complex $C_{-*}(S^1)$.
We may write
\[
  C_{-*}(S^1) = \begin{cases}
    k \cdot 1 & * = 0 \\
    k \cdot \lambda & * = 1,
  \end{cases}
\]
i.e.\ $C_{-*}(S^1) = k[\lambda] / \lambda^2$ with $|\lambda| = -1$.
The multiplication $C_{-*}(S^1) \otimes_k C_{-*}(S^1) \to C_{-*}(S^1)$ is induced by the multiplication map $S^1 \times S^1 \to S^1$.
We may view $C^*(X)$ as a $C_{-*}(S^1)$-module.
In the following examples we'll often use Poincar\'e duality to identify $C^*(X) = C_{-*}(X)$.

\begin{ex}
  Let $S^1$ act on itself via translation.
  This linearizes to $C_{-*}(S^1) \curvearrowright C^{*}(S^1)$ by ``sweeping out cochains along a chain.''
  Algebraically, we may write $C^*(S^1) = k[\nu] / \nu^2$ where $|\nu| = 1$, and the action satisfies $\lambda \cdot \nu = 1$.
  In topology, this is the ``slant product.''
\end{ex}

\begin{ex}
  Let $S^1$ act on $X = \pt$.
  This linearizes to $C_{-*}(S^1) \curvearrowright k$ via the augmentation action.
\end{ex}

\begin{ex}
  Let $S^1$ act on $S^2$ by rotation.
  The $S^1$ action here is trivial at the level of cohomology (any operation of degree $-1$ on $H_{-*}(S^2)$ is trivial).
  However, there is some interesting and subtle behavior: if we sweep out a (the cocycle corresponding to a) longitudinal line, we get all of $S^1$.
  The next example exhibits a similar phenomenon which is more cohomologically meaningful.
\end{ex}

\begin{ex}
  Let $S^1$ act on $S^3$ via the Hopf action.
  View $S^3 = \RR^3 \cup \{ \infty \}$.
  Let $\nu$ be the cocycle corresponding to a point.
  Sweeping $\nu$ out gives a circle (which we can take to be the compactified $z$-axis).
  This is still trivial -- there is a disk $D$ with $\partial D = \lambda \cdot \nu$.
  However, we have $\lambda \cdot D = S^3$.
  
  We end up with a ``secondary sweep'' operation $\lambda_{(2)}$, defined somewhat like the snake lemma: if $\lambda \cdot \nu$ is trivial, we write $\lambda \cdot \nu = \partial D$, and set $\lambda_{(2)} \nu = \lambda D$.
  Note that $\lambda_{(2)}$ is defined in terms of the dg-algebras and dg-modules but cannot be constructed from the cohomology!
  We say that the action $C_{-*}(S^1) \curvearrowright C^*(S^3)$ is not \emph{formal}.
\end{ex}

\subsection{Quotients and equivariant cohomology}

From an action $S^1 \curvearrowright X$, we'd like to construct a quotient $[X / S^1]$.
One way to do this that avoids issues with non-free actions is to choose a contractible space $ES^1$ on which $S^1$ acts freely and define
\[
  [X / S^1] = (X \times ES^1) / S^1.
\]
This space comes with a natural map to $BS^1 = [\pt / S^1] = (ES^1) / S^1$.
In fact, we can take $ES^1 = S^\infty = \cup_n S^{2n+1}$.
Then $BS^1 = \cup_n S^{2n} = \CC\PP^\infty$.

By taking quotients, we've turned $S^1 \curvearrowright X$ into a space $[X / S^1]$ over $BS^1$.
Koszul duality is about going back and forth between these perspectives.

\begin{dfn}
  We define the \emph{equivariant cochains} on $X$ to be $C^*_{S^1}(X) = C^*([X / S^1])$.
  This is naturally a $C^*([\pt / S^1])$-module.
\end{dfn}

\section{(9/4) David Nadler -- Continued}

Next week Peter Rowley will tell us about the $G$-Whitehead theorem and Elmendorf's theorem.

\subsection{Review}

Recall that we are focusing on actions $S^1 \curvearrowright X$ and linearizing by taking (co)chains valued in $k = \QQ$.
This induces an action $C_{-*}(S^1) \curvearrowright C^*(X)$.
Write $C_{-*}(S^1) = k[\lambda]$, where $\lambda$ is in degree $-1$ and acts by ``sweeps.''
On cohomology we may have ``higher sweeps'' $\lambda_{(n)}$ related to differentials in a spectral sequence.

Let
\[
  S^\infty = \cup_{n \geq 0} S^{2n+1} = ES^1 \subset \CC^\infty.
\]
We define $C^*_{S^1}(X) = C^*(X \times^{S^1} S^\infty)$.
We have a natural map $X \times^{S^1} S^\infty \to \pt \times^{S^1} S^\infty$, where
\[
  \pt \times^{S^1} S^\infty = S^\infty / S^1 = \CC\PP^\infty = BS^1.
\]
This induces an action $C^*_{S^1}(\pt) \curvearrowright C^*_{S^1}(X)$.
We have $C^*_{S^1}(\pt) = k[u]$ where $u$ is in degree $2$.

\subsection{Koszul duality}

Koszul duality lets us pass between the action $S^1 \curvearrowright X$ and the fibration $X \times^{S^1} ES^1 \to BS^1$.
At the linear level, we pass between $k[\lambda] \curvearrowright C^*(X)$ and $k[u] \curvearrowright C^*_{S^1}(X)$.

\begin{thm}[Algebraic Koszul duality]
  \[
    \Coh(k[\lambda]) = \Perf(k[u]).
  \]
\end{thm}

\begin{rmk}
  Note that $\Coh(k[\lambda]) \neq \Perf(k[\lambda])$.
  For example, the augmentation module $k = k[\lambda] / (\lambda)$ is coherent but not perfect (any resolution must be infinite).
\end{rmk}

To understand Koszul duality, consiter $C^*(S^\infty)$.
We have an action $S^1 \curvearrowright S^\infty$ and a fibration $S^\infty \to BS^1$.
Because the $S^1$-action is fiberwise, the induced $k[\lambda]$ and $k[u]$-actions on $C^*(S^\infty)$ commute.
That is, $C^*(S^\infty)$ is a $(k[\lambda], k[u])$-bimodule.
We may write $C^*(S^\infty)$ as a free $(k[\lambda], k[u])$-bimodule on one generator (with nontrivial differential so that cohomology is concentrated in degree $0$).
The Koszul duality equivalence is given by
\[
  M \mapsto M \otimes_{C_{-*}(S^1)} C^*(S^\infty) = \Hom_{C_{-*}(S^1)}(C^*(S\infty), M).
\]
To check that this is an equivalence, it suffices to show that it sends the generator $k \in \Coh(k[\lambda])$ to $k[u] \in \Perf(k[u])$ (up to a shift) and that this is an equivalence on endomorphisms.

A related theorem gives an equivalence
\[
  \Perf(k[\lambda]) = \Tors(k[u])
\]
where $\Tors(k[u])$ consists of modules set-theoretically supported at zero.

Suppose $X$ is a finite CW complex with $S^1$-action.
(We need some sort of hypothesis to ensure everything is finite.)
Then Koszul duality exchanges $C^*(X)$ and $C^*(X \times^{S^1} S^\infty)$.
The GKM paper explains how one may extract non-equivariant information from the equivariant cohomology.

\subsection{Cohomology and spectral sequences}

The Koszul duality theorem we mentioned is a statement about cochains.
What happens if we take cohomology?

The fibration $X \hookrightarrow X \times^{S^1} S^\infty \twoheadrightarrow BS^1$ gives a Serre spectral sequence abutting to $H^*_{S^1}(X)$.
The $E_2$ page of this spectral sequence has even columns given by $H^*(X)$ and odd columns all $0$.
Because this is an $E_2$ page, the differentials increase horizontal degree by $2$ and decrease vertical degree by $1$.
These differentials are the sweeps $d_2 = \lambda_{(1)}$.
The differentials on the $E_3$ page are trivial by degree reasons (they go from even columns to odd columns).
On the $E_4$ page, the differentials are higher sweeps $d_4 = \lambda_{(2)}$.
In general, differentials on odd pages are trivial, while differentials on even pages are higher sweeps (sweep, fill, then sweep again).
This is secretly what's powering the computation in the above proof of Koszul duality.

\begin{ex}
  Applying this method to the Hopf action $S^1 \curvearrowright S^3$, we get $H^*_{S^1}(S^3) = H^*(S^2)$ as expected.
\end{ex}

We say that $X$ is \emph{equivariantly formal} if the spectral sequence degenerates at $E_2$, i.e.\ if all sweeps are $0$.

\begin{ex}
  The action $S^1 \curvearrowright S^2$ is equivariantly formal.
\end{ex}

\subsection{Equivariant localization}

We want to think of ``$\Spec C^*_{S^1}(\pt)$'' as something like an affine line.
There is a torsion part (at ``$0$'') Koszul dual to $\Perf k[\lambda]$.
The non-torsion part is governed by the augmentation module $k = C^*(\pt)$ for $k[\lambda]$.
Note that this corresponds to $S^1$-fixed points.
The \emph{Tate construction} gives $\Coh(k[\lambda]) / \Perf(k[\lambda]) = \Perf k[u, u\inv]$.
These are two descriptions of ``$\Spec C^*_{S^1}(\pt)$.''

\begin{thm}[Equivariant localization]
  Let $X$ be a finite CW complex and let $S^1 \curvearrowright X$.
  Then $C^*_{S^1}(X)[u\inv] = C^*_{S^1}(X^{S^1})[u\inv]$.
\end{thm}

The modern perspective is that equivariant localization allows us to compute $C^*_{S^1}(X)$ away from the origin.
To compute $C^*_{S^1}(X)$ completely, we need to understand its behavior at the origin (i.e.\ the torsion part) and the gluing data.

\section{(9/11) Peter Rowley -- $G$-Spaces and the $G$-Whitehead Theorem}

I wasn't here for this day.
I obtained notes from Swapnil Garg and will try to type them up soon.

\section{(9/18) Peter Rowley -- Elmendorf's Theorem}

\subsection{Model categories}

We will take a model-categorical perspective on Elmendorf's theorem.
Recall that a \emph{model category} $\Cc$ is a category with distinguished (non-full) subcategories of fibrations, cofibrations, and weak equivalences (all satisfying conditions we won't get into).
A trivial (co)fibration is a (co)fibration which is also a weak equivalence.

\begin{dfn}
  An object $A \in \Cc$ is \emph{cofibrant} if $\varnothing \to A$ is a cofibration.
  Dually, an object $B \in \Cc$ is \emph{fibrant} if $A \to \pt$ is a cofibration.
\end{dfn}

These can be used to capture the homotopy theory of topological spaces:

\begin{ex}
  The category $\Top$ of (compactly generated weak Hausdorff) topological spaces has a model structure where:
  \begin{itemize}
    \item Fibrations are Serre fibrations.
    \item Weak equivalences are weak homotopy equivalences.
    \item Cofibrations are retracts of relative CW complexes.
      In particular, the cofibrant objects are precisely the CW complexes.
  \end{itemize}
\end{ex}

We may upgrade this to a $G$-equivariant version as follows.

\begin{ex}
  Let $G\Top$ be the category of (cgwh) $G$-spaces, where we say:
\begin{itemize}
  \item $f: X \to Y$ is a fibration if $f^H: X^H \to Y^H$ is a fibration for all $H$.
  \item $f: X \to Y$ is a weak equivalence if $f^H: X^H \to Y^H$ is a weak equivalence for all $H$ (equivalently, $\pi_n^H(f)$ is an isomorphism for all $n$ and $H$).
  \item Cofibrations are retracts of relative $G$-CW complexes.
    In particular, cofibrant objects are $G$-CW complexes.
\end{itemize}
\end{ex}

The right notion of adjunction between model categories is the following:

\begin{dfn}
  A \emph{Quillen adjunction} between model categories is an adjunction
  \[
    F: \Cc \rightleftarrows \Dc : G
  \]
  such that $F$ preserves cofibrations and trivial cofibrations, or equivalently $G$ preserves fibrations and trivial fibrations.
\end{dfn}

\begin{dfn}
  The \emph{homotopy category} $\Ho \Cc$ is $\Cc[W\inv]$ where $W$ is the class of weak equivalences.
\end{dfn}

Under suitable hypotheses, we may define left and right derived functors between homotopy categories.
These are essentially constructed by performing (co)fibrant replacement and then applying the functors as usual.
We may use these to obtain a notion of \emph{equivalence} of model categories.

\begin{dfn}
  A \emph{Quillen equivalence} is a Quillen adjunction $F: \Cc \rightleftarrows \Dc :G$ such that $\Lbf F$ and $\Rbf G$ are mutually inverse equivalences $\Ho \Cc \simeq \Ho \Dc$.
\end{dfn}

\subsection{Orbit categories}

To state Elmendorf's theorem, we need the following definition.

\begin{dfn}
  The \emph{orbit category} $\Oc_G$ is the full subcategory of $G\Top$ on objects of the form $G / H$.
\end{dfn}

Note that $\Map_{\Oc_G}(G/H, G/K)$ corresponds to subconjugacy relations $g\inv H g \subset K$ (since these correspond to maps $eH \mapsto gK$).
In particular, we have
\[
  \Map_{\Oc_G}(G/H, G/H) = WH := N_G(H) / H.
\]

\begin{ex}
  Let $G = \ZZ_2$.
  Then $G$ has exactly two subgroups: $e$ and $\ZZ_2$.
  The category $\Oc_G$ has:
  \begin{itemize}
    \item $\End(\ZZ_2 / e) = \ZZ_2$
    \item $\End(\ZZ_2 / \ZZ_2) = \pt$.
    \item $\Map(\ZZ_2 / e, \ZZ_2 / \ZZ_2) = \pt$.
    \item $\Map(\ZZ_2 / \ZZ_2, \ZZ_2 / e) = \varnothing$.
  \end{itemize}
\end{ex}

More generally, $G/e$ has $G$ as its group of automorphisms.

\subsection{Elmendorf's theorem}

Define a functor $\psi: G\Top \to \Fun(\Oc_G\op, \Top)$ by 
\[
  \psi(X)(G/H) = \Map(G/H, X) = X^H.
\]
We'd like to say that $\psi$ induces a Quillen equivalence.
To accomplish this, we need to define a model structure on $\Fun(\Oc_G\op, \Top)$.
We may use the \emph{projective model structure}, where fibrations and weak equivalences are defined pointwise.

\begin{thm}[Elmendorf]
  The functor $\psi$ is the right adjoint in a Quillen equivalence.
  The left adjoint is given by $\theta: \Fun(\Oc_G\op, \Top) \to G\Top$ defined by $\theta(Y) = Y(G / e)$.
\end{thm}

At the level of $\infty$-categories, we may say that $\psi$ is an equivalence of $(\infty, 1)$-categories.
In practice, Elmendorf's theorem lets us exchange the ``algebra of $G$'' for the ``geometry of $\Oc_G$.''

Let's first understand why $\theta: \Fun(\Oc_G\op, \Top) \rightleftarrows G\Top :\psi$ is a Quillen adjunction.
It follows from the definitions that $\psi$ preserves fibrations and trivial fibrations, so it suffices to show that we actually have an adjunction.
That is, we need to show
\[
  \Hom_{G\Top}(Y(G/e), X) \cong \Hom(Y, X^{(-)}).
\]
The backwards map is just evaluation at $H$.
The forwards map is given by assembling the maps
\[
  Y(G/H) \to Y(G/e)^H \to X^H.
\]

At the level of homotopy categories, $\Lbf \theta$ is equivalent to the bar construction.
More precisely: let $M: \Oc_G \to \Top$ be the forgetful map, and let $\Phi: \Fun(\Oc_G\op, \Top) \to \Top$ be the geometric realization of the bar complex
\[
  \Phi(X) = |B_\bullet(X, \Oc_G, M)|.
\]
Here $B_\bullet(X, \Oc_G, M)$ is the simplicial space
\[
  [n] \mapsto \sqcup_{G / H_{n-1} \to \dots \to G / H_0} X(G / H_0) \times M(G / H_{n-1}).
\]
Then $\Phi$ gives an alternative construction of $\Lbf \theta$ on homotopy categories.

One may understand this explicitly for $G = \ZZ_2$ but I couldn't make the pictures for that, sorry!

\subsection{(9/25) Gabriel Beiner -- Bredon Cohomology}

\subsection{Review}

Recall that, given a group $G$, we can construct an \emph{orbit category} $\Oc_G$.
The objects of $\Oc_G$ are quotients $G / H$, and maps are given by
\[
  \Mor_{\Oc_G}(G/H, G/K) = \Map^G(G/H, G/K).
\]
Elmendorf's theorem gives a (Quillen) equivalence
\begin{align*}
  G\Top &\simeq \Fun(\Oc_G\op, \Top) \\
  X &\mapsto (G/H \mapsto X^H).
\end{align*}

Today we want to discuss a robust notion of cohomology for $G$-spaces which vastly generalizes the classical Borel cohomology.

\subsection{Bredon cohomology}

Suppose $G$ is a finite group.

\begin{dfn}
  A \emph{coefficient system} for $G$ is $M \in \Fun(\Oc_G\op, \Ab)$.
\end{dfn}

\begin{rmk}
  If $G$ is a topological group, we'd replace $\Oc_G\op$ by the homotopy category $h\Oc_G\op$.
\end{rmk}

\begin{ex}
  For any $G$, there is a constant coefficient system $\ul{Z}$ given by $G/H \mapsto \ZZ$.
\end{ex}

\begin{ex}
  Given a $G$-space $X$, we may define coefficient systems:
  \begin{itemize}
    \item $\ul{\pi}_n(X)$ (for $n > 1$) by $G/H \mapsto \pi_n(X^H)$
    \item $\ul{H}_n(X)$ by $G/H \mapsto H_n(X^H)$.
  \end{itemize}
\end{ex}

For a $G$-CW complex $X$, let $\sk_n X$ denote the $n$-skeleton (built inductively out of disks $G / H \times D^k$ for $k < n$).
Let
\[
  \ul{\Cc}_n(X) = \ul{H}_n(\sk_n X, \sk_{n-1} X; \ZZ)
\]
This sends $G/H$ to $H_n(\sk_n X^H, \sk_{n-1} X^H) = \Cc_n^{\textrm{CW}}(X^H)$.
Allowing $n$ to vary, we get cellular chain complexes $\Cc_\bullet(X)(G/H)$ for all $H$.

\begin{dfn}
  Let $X$ be a $G$-CW complex and $M$ a coefficient system for $G$.
  The \emph{Bredon cohomology} of $X$ with coefficients in $M$ is
  \[
    H^n_G(X; M) = H^n(\Hom_{\Fun(\Oc_G\op, \Ab)}(\Cc_\bullet(X), M))
  \]
  The \emph{Bredon homology} of $X$ with coefficients in $M$ is
  \[
    H_n^G(X; M) = \ul{\Cc}_\bullet(X) \otimes_{\Oc_G} M)
  \]
  where $\otimes_{\Oc_G}$ is a coend (sort of like the usual tensor product).
\end{dfn}

\begin{rmk}
  David suggested that we can instead take an $\infty$-categorical perspective.
  Let's think of the orbit category as (the exit-path category of) a stratified space with strata $B(WH) = B(N_G(H)/H)$ for $H \subset G$.
  The coefficient systems are then constructible (co?)sheaves on this stratified space.
  The $G$-CW complexes $X$ above give spaces $\Xc$ over our stratified space (with fiber over $B(WH)$ given by $X^H$)
  Bredon cohomology computes the cohomology of the pullback of a constructible sheaf to $\Xc$.
\end{rmk}

\subsection{Examples}

\begin{ex}
  Let's consider $\ZZ_2$ acting on $S^2$ via rotation by $\pi$.
  We'll compute $H^*_{\ZZ_2}(S^2; \ZZ)$.

  To perform this computation, let's give $S^2$ the $\ZZ_2$-CW structure with:
  \begin{itemize}
    \item Two $0$-cells $\ZZ_2 / \ZZ_2 \times D^0$ (the north and south poles)
    \item One $1$-cell $\ZZ_2 / e \times D^1$ (corresponding to the prime meridian and its opposite, which $\ZZ_2$ permutes)
    \item One $2$-cell $\ZZ_2 / e \times D^2$ (corresponding to the eastern and western hemispheres, which $\ZZ_2$ permutes).
  \end{itemize}
  The category $\Oc_{\ZZ_2}$ was described in the last talk.
  We may describe $\Cc_\bullet(X)$ as follows:
  \begin{itemize}
    \item For $e$-fixed points, the cellular chain complex $\Cc_\bullet(X^e)$ is
      \[
        \begin{tikzcd}
          0 \rar & \ZZ^2 \rar["d_2"] & \ZZ^2 \rar["d_1"] & \ZZ^2 \rar & 0
        \end{tikzcd}
      \]
      The differentials are $d_2(x, y) = (x-y, y-x)$ and $d_1(u, v) = (u+v, -u-v)$.
      Here $\ZZ_2$ (acting on $\ZZ_2 / e$ by outer automorphisms) acts trivially on $\Cc_0$ and swaps the factors of $\Cc_1$ and $\Cc_2$.
    \item For $\ZZ_2$-fixed points, the cellular chain complex $\Cc_\bullet(X^{\ZZ_2})$ is
      \[
        \begin{tikzcd}
          0 \rar & 0 \rar & 0 \rar & \ZZ^2 \rar & 0.
        \end{tikzcd}
      \]
      The automorphism group of $\ZZ_2 / \ZZ_2$ is trivial, so we don't need to care about the group actions.
      However, we do want to remember the homomorphism from this complex to the above complex (given by the identity on $\ZZ^2 = \Cc_0$).
  \end{itemize}
  To compute Bredon cohomology, we take $\Hom(\Cc_\bullet(X), \ul{\ZZ})$, where
  \[
    \ul{\ZZ}(\ZZ_2 / H) = \begin{cases}
      \ZZ \textrm{ with trivial action} & H = e \\
      \ZZ & H = \ZZ_2.
    \end{cases}
  \]
  Some computation (which I didn't entirely catch) shows that $\Hom(\Cc_\bullet(X), \ul{\ZZ})$ is
  \[
    \begin{tikzcd}
      0 \rar & \ZZ^2 \rar["\delta_1"] & \ZZ \rar["\delta_2"] \rar & \ZZ \rar & 0.
    \end{tikzcd}
  \]
  Here the middle $\ZZ$ is generated by $\phi$ satisfying $\phi(u, v) = u + v$.
  The differential $\delta_1$ satisfies $\delta_1(a, b) = a - b$, and $\delta_2 = 0$.
  Thus
  \[
    H^\bullet_{\ZZ_2}(S^2; \ul{\ZZ}) = \begin{cases}
      \ZZ & \bullet = 0, 2 \\
      0 & \textrm{else.}
    \end{cases}
  \]
\end{ex}

\end{document}
