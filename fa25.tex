\documentclass{article}

\usepackage{notes}

\title{GRT Seminar Fall 2024 -- Equivariant Mathematics}
\author{Notes by John S.\ Nolan, speakers listed below}

\begin{document}

\maketitle

\begin{abstract}
	This semester, the GRT Seminar will focus on equivariant mathematics (especially homotopy theory).
  The goal is not necessarily to become homotopy theorists but rather to gain an appreciation for one of the most developed approaches to equivariance.
\end{abstract}

\tableofcontents

\section{8/28 (David Nadler) -- $S^1$-Actions and Koszul Duality}

A good reference for this material is GKM (Goresky--Kottwitz--Macpherson, ``Equivairant cohomology, Koszul duality, and the localization theorem'').
We'll focus on the case of $G = S^1$ acting on a (reasonable) topological space $X$, e.g.\ a CW complex.

\subsection{Linearization}

We can ``linearize'' the space $X$ by taking cochains $C^*(X)$, say with coefficients in $k = \QQ$.
Morally, this is like passing to ``functions'' on a space, except that in topology our notion of function is ``locally constant.''
This is not as uninteresting as it sounds, since we also include ``derived'' information.

Following this analogy, ``distributions'' on $S^1$ should act on ``functions'' on $X$ by convolution.
The right notion of ``distributions'' on $S^1$ is captured by the chain complex $C_{-*}(S^1)$.
We may write
\[
  C_{-*}(S^1) = \begin{cases}
    k \cdot 1 & * = 0 \\
    k \cdot \lambda & * = 1,
  \end{cases}
\]
i.e.\ $C_{-*}(S^1) = k[\lambda] / \lambda^2$ with $|\lambda| = -1$.
The multiplication $C_{-*}(S^1) \otimes_k C_{-*}(S^1) \to C_{-*}(S^1)$ is induced by the multiplication map $S^1 \times S^1 \to S^1$.
We may view $C^*(X)$ as a $C_{-*}(S^1)$-module.
In the following examples we'll often use Poincar\'e duality to identify $C^*(X) = C_{-*}(X)$.

\begin{ex}
  Let $S^1$ act on itself via translation.
  This linearizes to $C_{-*}(S^1) \curvearrowright C^{*}(S^1)$ by ``sweeping out cochains along a chain.''
  Algebraically, we may write $C^*(S^1) = k[\nu] / \nu^2$ where $|\nu| = 1$, and the action satisfies $\lambda \cdot \nu = 1$.
  In topology, this is the ``slant product.''
\end{ex}

\begin{ex}
  Let $S^1$ act on $X = \pt$.
  This linearizes to $C_{-*}(S^1) \curvearrowright k$ via the augmentation action.
\end{ex}

\begin{ex}
  Let $S^1$ act on $S^2$ by rotation.
  The $S^1$ action here is trivial at the level of cohomology (any operation of degree $-1$ on $H_{-*}(S^2)$ is trivial).
  However, there is some interesting and subtle behavior: if we sweep out a (the cocycle corresponding to a) longitudinal line, we get all of $S^1$.
  The next example exhibits a similar phenomenon which is more cohomologically meaningful.
\end{ex}

\begin{ex}
  Let $S^1$ act on $S^3$ via the Hopf action.
  View $S^3 = \RR^3 \cup \{ \infty \}$.
  Let $\nu$ be the cocycle corresponding to a point.
  Sweeping $\nu$ out gives a circle (which we can take to be the compactified $z$-axis).
  This is still trivial -- there is a disk $D$ with $\partial D = \lambda \cdot \nu$.
  However, we have $\lambda \cdot D = S^3$.
  
  We end up with a ``secondary sweep'' operation $\lambda_{(2)}$, defined somewhat like the snake lemma: if $\lambda \cdot \nu$ is trivial, we write $\lambda \cdot \nu = \partial D$, and set $\lambda_{(2)} \nu = \lambda D$.
  Note that $\lambda_{(2)}$ is defined in terms of the dg-algebras and dg-modules but cannot be constructed from the cohomology!
  We say that the action $C_{-*}(S^1) \curvearrowright C^*(S^3)$ is not \emph{formal}.
\end{ex}

\subsection{Quotients and equivariant cohomology}

From an action $S^1 \curvearrowright X$, we'd like to construct a quotient $[X / S^1]$.
One way to do this that avoids issues with non-free actions is to choose a contractible space $ES^1$ on which $S^1$ acts freely and define
\[
  [X / S^1] = (X \times ES^1) / S^1.
\]
This space comes with a natural map to $BS^1 = [\pt / S^1] = (ES^1) / S^1$.
In fact, we can take $ES^1 = S^\infty = \cup_n S^{2n+1}$.
Then $BS^1 = \cup_n S^{2n} = \CC\PP^\infty$.

By taking quotients, we've turned $S^1 \curvearrowright X$ into a space $[X / S^1]$ over $BS^1$.
Koszul duality is about going back and forth between these perspectives.

\begin{dfn}
  We define the \emph{equivariant cochains} on $X$ to be $C^*_{S^1}(X) = C^*([X / S^1])$.
  This is naturally a $C^*([\pt / S^1])$-module.
\end{dfn}

\end{document}
