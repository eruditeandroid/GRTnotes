\documentclass{article}

\usepackage{notes}

\title{GRT Seminar Fall 2024 -- Equivariant Mathematics}
\author{Notes by John S.\ Nolan, speakers listed below}

\begin{document}

\maketitle

\begin{abstract}
	This semester, the GRT Seminar will focus on equivariant mathematics (especially homotopy theory).
  The goal is not necessarily to become homotopy theorists but rather to gain an appreciation for one of the most developed approaches to equivariance.
\end{abstract}

\tableofcontents

\section{(8/28) David Nadler -- $S^1$-Actions and Koszul Duality}

A good reference for this material is GKM (Goresky--Kottwitz--Macpherson, ``Equivairant cohomology, Koszul duality, and the localization theorem'').
We'll focus on the case of $G = S^1$ acting on a (reasonable) topological space $X$, e.g.\ a CW complex.

\subsection{Linearization}

We can ``linearize'' the space $X$ by taking cochains $C^*(X)$, say with coefficients in $k = \QQ$.
Morally, this is like passing to ``functions'' on a space, except that in topology our notion of function is ``locally constant.''
This is not as uninteresting as it sounds, since we also include ``derived'' information.

Following this analogy, ``distributions'' on $S^1$ should act on ``functions'' on $X$ by convolution.
The right notion of ``distributions'' on $S^1$ is captured by the chain complex $C_{-*}(S^1)$.
We may write
\[
  C_{-*}(S^1) = \begin{cases}
    k \cdot 1 & * = 0 \\
    k \cdot \lambda & * = 1,
  \end{cases}
\]
i.e.\ $C_{-*}(S^1) = k[\lambda] / \lambda^2$ with $|\lambda| = -1$.
The multiplication $C_{-*}(S^1) \otimes_k C_{-*}(S^1) \to C_{-*}(S^1)$ is induced by the multiplication map $S^1 \times S^1 \to S^1$.
We may view $C^*(X)$ as a $C_{-*}(S^1)$-module.
In the following examples we'll often use Poincar\'e duality to identify $C^*(X) = C_{-*}(X)$.

\begin{ex}
  Let $S^1$ act on itself via translation.
  This linearizes to $C_{-*}(S^1) \curvearrowright C^{*}(S^1)$ by ``sweeping out cochains along a chain.''
  Algebraically, we may write $C^*(S^1) = k[\nu] / \nu^2$ where $|\nu| = 1$, and the action satisfies $\lambda \cdot \nu = 1$.
  In topology, this is the ``slant product.''
\end{ex}

\begin{ex}
  Let $S^1$ act on $X = \pt$.
  This linearizes to $C_{-*}(S^1) \curvearrowright k$ via the augmentation action.
\end{ex}

\begin{ex}
  Let $S^1$ act on $S^2$ by rotation.
  The $S^1$ action here is trivial at the level of cohomology (any operation of degree $-1$ on $H_{-*}(S^2)$ is trivial).
  However, there is some interesting and subtle behavior: if we sweep out a (the cocycle corresponding to a) longitudinal line, we get all of $S^1$.
  The next example exhibits a similar phenomenon which is more cohomologically meaningful.
\end{ex}

\begin{ex}
  Let $S^1$ act on $S^3$ via the Hopf action.
  View $S^3 = \RR^3 \cup \{ \infty \}$.
  Let $\nu$ be the cocycle corresponding to a point.
  Sweeping $\nu$ out gives a circle (which we can take to be the compactified $z$-axis).
  This is still trivial -- there is a disk $D$ with $\partial D = \lambda \cdot \nu$.
  However, we have $\lambda \cdot D = S^3$.
  
  We end up with a ``secondary sweep'' operation $\lambda_{(2)}$, defined somewhat like the snake lemma: if $\lambda \cdot \nu$ is trivial, we write $\lambda \cdot \nu = \partial D$, and set $\lambda_{(2)} \nu = \lambda D$.
  Note that $\lambda_{(2)}$ is defined in terms of the dg-algebras and dg-modules but cannot be constructed from the cohomology!
  We say that the action $C_{-*}(S^1) \curvearrowright C^*(S^3)$ is not \emph{formal}.
\end{ex}

\subsection{Quotients and equivariant cohomology}

From an action $S^1 \curvearrowright X$, we'd like to construct a quotient $[X / S^1]$.
One way to do this that avoids issues with non-free actions is to choose a contractible space $ES^1$ on which $S^1$ acts freely and define
\[
  [X / S^1] = (X \times ES^1) / S^1.
\]
This space comes with a natural map to $BS^1 = [\pt / S^1] = (ES^1) / S^1$.
In fact, we can take $ES^1 = S^\infty = \cup_n S^{2n+1}$.
Then $BS^1 = \cup_n S^{2n} = \CC\PP^\infty$.

By taking quotients, we've turned $S^1 \curvearrowright X$ into a space $[X / S^1]$ over $BS^1$.
Koszul duality is about going back and forth between these perspectives.

\begin{dfn}
  We define the \emph{equivariant cochains} on $X$ to be $C^*_{S^1}(X) = C^*([X / S^1])$.
  This is naturally a $C^*([\pt / S^1])$-module.
\end{dfn}

\section{(9/4) David Nadler -- Continued}

Next week Peter Rowley will tell us about the $G$-Whitehead theorem and Elmendorf's theorem.

\subsection{Review}

Recall that we are focusing on actions $S^1 \curvearrowright X$ and linearizing by taking (co)chains valued in $k = \QQ$.
This induces an action $C_{-*}(S^1) \curvearrowright C^*(X)$.
Write $C_{-*}(S^1) = k[\lambda]$, where $\lambda$ is in degree $-1$ and acts by ``sweeps.''
On cohomology we may have ``higher sweeps'' $\lambda_{(n)}$ related to differentials in a spectral sequence.

Let
\[
  S^\infty = \cup_{n \geq 0} S^{2n+1} = ES^1 \subset \CC^\infty.
\]
We define $C^*_{S^1}(X) = C^*(X \times^{S^1} S^\infty)$.
We have a natural map $X \times^{S^1} S^\infty \to \pt \times^{S^1} S^\infty$, where
\[
  \pt \times^{S^1} S^\infty = S^\infty / S^1 = \CC\PP^\infty = BS^1.
\]
This induces an action $C^*_{S^1}(\pt) \curvearrowright C^*_{S^1}(X)$.
We have $C^*_{S^1}(\pt) = k[u]$ where $u$ is in degree $2$.

\subsection{Koszul duality}

Koszul duality lets us pass between the action $S^1 \curvearrowright X$ and the fibration $X \times^{S^1} ES^1 \to BS^1$.
At the linear level, we pass between $k[\lambda] \curvearrowright C^*(X)$ and $k[u] \curvearrowright C^*_{S^1}(X)$.

\begin{thm}[Algebraic Koszul duality]
  \[
    \Coh(k[\lambda]) = \Perf(k[u]).
  \]
\end{thm}

\begin{rmk}
  Note that $\Coh(k[\lambda]) \neq \Perf(k[\lambda])$.
  For example, the augmentation module $k = k[\lambda] / (\lambda)$ is coherent but not perfect (any resolution must be infinite).
\end{rmk}

To understand Koszul duality, consiter $C^*(S^\infty)$.
We have an action $S^1 \curvearrowright S^\infty$ and a fibration $S^\infty \to BS^1$.
Because the $S^1$-action is fiberwise, the induced $k[\lambda]$ and $k[u]$-actions on $C^*(S^\infty)$ commute.
That is, $C^*(S^\infty)$ is a $(k[\lambda], k[u])$-bimodule.
We may write $C^*(S^\infty)$ as a free $(k[\lambda], k[u])$-bimodule on one generator (with nontrivial differential so that cohomology is concentrated in degree $0$).
The Koszul duality equivalence is given by
\[
  M \mapsto M \otimes_{C_{-*}(S^1)} C^*(S^\infty) = \Hom_{C_{-*}(S^1)}(C^*(S\infty), M).
\]
To check that this is an equivalence, it suffices to show that it sends the generator $k \in \Coh(k[\lambda])$ to $k[u] \in \Perf(k[u])$ (up to a shift) and that this is an equivalence on endomorphisms.

A related theorem gives an equivalence
\[
  \Perf(k[\lambda]) = \Tors(k[u])
\]
where $\Tors(k[u])$ consists of modules set-theoretically supported at zero.

Suppose $X$ is a finite CW complex with $S^1$-action.
(We need some sort of hypothesis to ensure everything is finite.)
Then Koszul duality exchanges $C^*(X)$ and $C^*(X \times^{S^1} S^\infty)$.
The GKM paper explains how one may extract non-equivariant information from the equivariant cohomology.

\subsection{Cohomology and spectral sequences}

The Koszul duality theorem we mentioned is a statement about cochains.
What happens if we take cohomology?

The fibration $X \hookrightarrow X \times^{S^1} S^\infty \twoheadrightarrow BS^1$ gives a Serre spectral sequence abutting to $H^*_{S^1}(X)$.
The $E_2$ page of this spectral sequence has even columns given by $H^*(X)$ and odd columns all $0$.
Because this is an $E_2$ page, the differentials increase horizontal degree by $2$ and decrease vertical degree by $1$.
These differentials are the sweeps $d_2 = \lambda_{(1)}$.
The differentials on the $E_3$ page are trivial by degree reasons (they go from even columns to odd columns).
On the $E_4$ page, the differentials are higher sweeps $d_4 = \lambda_{(2)}$.
In general, differentials on odd pages are trivial, while differentials on even pages are higher sweeps (sweep, fill, then sweep again).
This is secretly what's powering the computation in the above proof of Koszul duality.

\begin{ex}
  Applying this method to the Hopf action $S^1 \curvearrowright S^3$, we get $H^*_{S^1}(S^3) = H^*(S^2)$ as expected.
\end{ex}

We say that $X$ is \emph{equivariantly formal} if the spectral sequence degenerates at $E_2$, i.e.\ if all sweeps are $0$.

\begin{ex}
  The action $S^1 \curvearrowright S^2$ is equivariantly formal.
\end{ex}

\subsection{Equivariant localization}

We want to think of ``$\Spec C^*_{S^1}(\pt)$'' as something like an affine line.
There is a torsion part (at ``$0$'') Koszul dual to $\Perf k[\lambda]$.
The non-torsion part is governed by the augmentation module $k = C^*(\pt)$ for $k[\lambda]$.
Note that this corresponds to $S^1$-fixed points.
The \emph{Tate construction} gives $\Coh(k[\lambda]) / \Perf(k[\lambda]) = \Perf k[u, u\inv]$.
These are two descriptions of ``$\Spec C^*_{S^1}(\pt)$.''

\begin{thm}[Equivariant localization]
  Let $X$ be a finite CW complex and let $S^1 \curvearrowright X$.
  Then $C^*_{S^1}(X)[u\inv] = C^*_{S^1}(X^{S^1})[u\inv]$.
\end{thm}

The modern perspective is that equivariant localization allows us to compute $C^*_{S^1}(X)$ away from the origin.
To compute $C^*_{S^1}(X)$ completely, we need to understand its behavior at the origin (i.e.\ the torsion part) and the gluing data.

\section{(9/11) Peter Rowley -- $G$-Spaces and the $G$-Whitehead Theorem}

I wasn't here for this day.
I obtained notes from Swapnil Garg and will try to type them up soon.

\section{(9/18) Peter Rowley -- Elmendorf's Theorem}

\subsection{Model categories}

We will take a model-categorical perspective on Elmendorf's theorem.
Recall that a \emph{model category} $\Cc$ is a category with distinguished (non-full) subcategories of fibrations, cofibrations, and weak equivalences (all satisfying conditions we won't get into).
A trivial (co)fibration is a (co)fibration which is also a weak equivalence.

\begin{dfn}
  An object $A \in \Cc$ is \emph{cofibrant} if $\varnothing \to A$ is a cofibration.
  Dually, an object $B \in \Cc$ is \emph{fibrant} if $A \to \pt$ is a cofibration.
\end{dfn}

These can be used to capture the homotopy theory of topological spaces:

\begin{ex}
  The category $\Top$ of (compactly generated weak Hausdorff) topological spaces has a model structure where:
  \begin{itemize}
    \item Fibrations are Serre fibrations.
    \item Weak equivalences are weak homotopy equivalences.
    \item Cofibrations are retracts of relative CW complexes.
      In particular, the cofibrant objects are precisely the CW complexes.
  \end{itemize}
\end{ex}

We may upgrade this to a $G$-equivariant version as follows.

\begin{ex}
  Let $G\Top$ be the category of (cgwh) $G$-spaces, where we say:
\begin{itemize}
  \item $f: X \to Y$ is a fibration if $f^H: X^H \to Y^H$ is a fibration for all $H$.
  \item $f: X \to Y$ is a weak equivalence if $f^H: X^H \to Y^H$ is a weak equivalence for all $H$ (equivalently, $\pi_n^H(f)$ is an isomorphism for all $n$ and $H$).
  \item Cofibrations are retracts of relative $G$-CW complexes.
    In particular, cofibrant objects are $G$-CW complexes.
\end{itemize}
\end{ex}

The right notion of adjunction between model categories is the following:

\begin{dfn}
  A \emph{Quillen adjunction} between model categories is an adjunction
  \[
    F: \Cc \rightleftarrows \Dc : G
  \]
  such that $F$ preserves cofibrations and trivial cofibrations, or equivalently $G$ preserves fibrations and trivial fibrations.
\end{dfn}

\begin{dfn}
  The \emph{homotopy category} $\Ho \Cc$ is $\Cc[W\inv]$ where $W$ is the class of weak equivalences.
\end{dfn}

Under suitable hypotheses, we may define left and right derived functors between homotopy categories.
These are essentially constructed by performing (co)fibrant replacement and then applying the functors as usual.
We may use these to obtain a notion of \emph{equivalence} of model categories.

\begin{dfn}
  A \emph{Quillen equivalence} is a Quillen adjunction $F: \Cc \rightleftarrows \Dc :G$ such that $\Lbf F$ and $\Rbf G$ are mutually inverse equivalences $\Ho \Cc \simeq \Ho \Dc$.
\end{dfn}

\subsection{Orbit categories}

To state Elmendorf's theorem, we need the following definition.

\begin{dfn}
  The \emph{orbit category} $\Oc_G$ is the full subcategory of $G\Top$ on objects of the form $G / H$.
\end{dfn}

Note that $\Map_{\Oc_G}(G/H, G/K)$ corresponds to subconjugacy relations $g\inv H g \subset K$ (since these correspond to maps $eH \mapsto gK$).
In particular, we have
\[
  \Map_{\Oc_G}(G/H, G/H) = WH := N_G(H) / H.
\]

\begin{ex}
  Let $G = \ZZ_2$.
  Then $G$ has exactly two subgroups: $e$ and $\ZZ_2$.
  The category $\Oc_G$ has:
  \begin{itemize}
    \item $\End(\ZZ_2 / e) = \ZZ_2$
    \item $\End(\ZZ_2 / \ZZ_2) = \pt$.
    \item $\Map(\ZZ_2 / e, \ZZ_2 / \ZZ_2) = \pt$.
    \item $\Map(\ZZ_2 / \ZZ_2, \ZZ_2 / e) = \varnothing$.
  \end{itemize}
\end{ex}

More generally, $G/e$ has $G$ as its group of automorphisms.

\subsection{Elmendorf's theorem}

Define a functor $\psi: G\Top \to \Fun(\Oc_G\op, \Top)$ by 
\[
  \psi(X)(G/H) = \Map(G/H, X) = X^H.
\]
We'd like to say that $\psi$ induces a Quillen equivalence.
To accomplish this, we need to define a model structure on $\Fun(\Oc_G\op, \Top)$.
We may use the \emph{projective model structure}, where fibrations and weak equivalences are defined pointwise.

\begin{thm}[Elmendorf]
  The functor $\psi$ is the right adjoint in a Quillen equivalence.
  The left adjoint is given by $\theta: \Fun(\Oc_G\op, \Top) \to G\Top$ defined by $\theta(Y) = Y(G / e)$.
\end{thm}

At the level of $\infty$-categories, we may say that $\psi$ is an equivalence of $(\infty, 1)$-categories.
In practice, Elmendorf's theorem lets us exchange the ``algebra of $G$'' for the ``geometry of $\Oc_G$.''

Let's first understand why $\theta: \Fun(\Oc_G\op, \Top) \rightleftarrows G\Top :\psi$ is a Quillen adjunction.
It follows from the definitions that $\psi$ preserves fibrations and trivial fibrations, so it suffices to show that we actually have an adjunction.
That is, we need to show
\[
  \Hom_{G\Top}(Y(G/e), X) \cong \Hom(Y, X^{(-)}).
\]
The backwards map is just evaluation at $H$.
The forwards map is given by assembling the maps
\[
  Y(G/H) \to Y(G/e)^H \to X^H.
\]

At the level of homotopy categories, $\Lbf \theta$ is equivalent to the bar construction.
More precisely: let $M: \Oc_G \to \Top$ be the forgetful map, and let $\Phi: \Fun(\Oc_G\op, \Top) \to \Top$ be the geometric realization of the bar complex
\[
  \Phi(X) = |B_\bullet(X, \Oc_G, M)|.
\]
Here $B_\bullet(X, \Oc_G, M)$ is the simplicial space
\[
  [n] \mapsto \sqcup_{G / H_{n-1} \to \dots \to G / H_0} X(G / H_0) \times M(G / H_{n-1}).
\]
Then $\Phi$ gives an alternative construction of $\Lbf \theta$ on homotopy categories.

One may understand this explicitly for $G = \ZZ_2$ but I couldn't make the pictures for that, sorry!

\section{(9/25) Gabriel Beiner -- Bredon Cohomology}

\subsection{Review}

Recall that, given a group $G$, we can construct an \emph{orbit category} $\Oc_G$.
The objects of $\Oc_G$ are quotients $G / H$, and maps are given by
\[
  \Mor_{\Oc_G}(G/H, G/K) = \Map^G(G/H, G/K).
\]
Elmendorf's theorem gives a (Quillen) equivalence
\begin{align*}
  G\Top &\simeq \Fun(\Oc_G\op, \Top) \\
  X &\mapsto (G/H \mapsto X^H).
\end{align*}

Today we want to discuss a robust notion of cohomology for $G$-spaces which vastly generalizes the classical Borel cohomology.

\subsection{Bredon cohomology}

Suppose $G$ is a finite group.

\begin{dfn}
  A \emph{coefficient system} for $G$ is $M \in \Fun(\Oc_G\op, \Ab)$.
\end{dfn}

\begin{rmk}
  If $G$ is a topological group, we'd replace $\Oc_G\op$ by the homotopy category $h\Oc_G\op$.
\end{rmk}

\begin{ex}
  For any $G$, there is a constant coefficient system $\ul{Z}$ given by $G/H \mapsto \ZZ$.
\end{ex}

\begin{ex}
  Given a $G$-space $X$, we may define coefficient systems:
  \begin{itemize}
    \item $\ul{\pi}_n(X)$ (for $n > 1$) by $G/H \mapsto \pi_n(X^H)$
    \item $\ul{H}_n(X)$ by $G/H \mapsto H_n(X^H)$.
  \end{itemize}
\end{ex}

For a $G$-CW complex $X$, let $\sk_n X$ denote the $n$-skeleton (built inductively out of disks $G / H \times D^k$ for $k < n$).
Let
\[
  \ul{\Cc}_n(X) = \ul{H}_n(\sk_n X, \sk_{n-1} X; \ZZ)
\]
This sends $G/H$ to $H_n(\sk_n X^H, \sk_{n-1} X^H) = \Cc_n^{\mathrm{CW}}(X^H)$.
Allowing $n$ to vary, we get cellular chain complexes $\Cc_\bullet(X)(G/H)$ for all $H$.

\begin{dfn}
  Let $X$ be a $G$-CW complex and $M$ a coefficient system for $G$.
  The \emph{Bredon cohomology} of $X$ with coefficients in $M$ is
  \[
    H^n_G(X; M) = H^n(\Hom_{\Fun(\Oc_G\op, \Ab)}(\Cc_\bullet(X), M))
  \]
  The \emph{Bredon homology} of $X$ with coefficients in $M$ is
  \[
    H_n^G(X; M) = \ul{\Cc}_\bullet(X) \otimes_{\Oc_G} M)
  \]
  where $\otimes_{\Oc_G}$ is a coend (sort of like the usual tensor product).
\end{dfn}

\begin{rmk}
  David suggested that we can instead take an $\infty$-categorical perspective.
  Let's think of the orbit category as (the exit-path category of) a stratified space with strata $B(WH) = B(N_G(H)/H)$ for $H \subset G$.
  The coefficient systems are then constructible (co?)sheaves on this stratified space.
  The $G$-CW complexes $X$ above give spaces $\Xc$ over our stratified space (with fiber over $B(WH)$ given by $X^H$)
  Bredon cohomology computes the cohomology of the pullback of a constructible sheaf to $\Xc$.
\end{rmk}

\subsection{Examples}

\begin{ex}
  Let's consider $\ZZ_2$ acting on $S^2$ via rotation by $\pi$.
  We'll compute $H^*_{\ZZ_2}(S^2; \ZZ)$.

  To perform this computation, let's give $S^2$ the $\ZZ_2$-CW structure with:
  \begin{itemize}
    \item Two $0$-cells $\ZZ_2 / \ZZ_2 \times D^0$ (the north and south poles)
    \item One $1$-cell $\ZZ_2 / e \times D^1$ (corresponding to the prime meridian and its opposite, which $\ZZ_2$ permutes)
    \item One $2$-cell $\ZZ_2 / e \times D^2$ (corresponding to the eastern and western hemispheres, which $\ZZ_2$ permutes).
  \end{itemize}
  The category $\Oc_{\ZZ_2}$ was described in the last talk.
  We may describe $\Cc_\bullet(X)$ as follows:
  \begin{itemize}
    \item For $e$-fixed points, the cellular chain complex $\Cc_\bullet(X^e)$ is
      \[
        \begin{tikzcd}
          0 \rar & \ZZ^2 \rar["d_2"] & \ZZ^2 \rar["d_1"] & \ZZ^2 \rar & 0
        \end{tikzcd}
      \]
      The differentials are $d_2(x, y) = (x-y, y-x)$ and $d_1(u, v) = (u+v, -u-v)$.
      Here $\ZZ_2$ (acting on $\ZZ_2 / e$ by outer automorphisms) acts trivially on $\Cc_0$ and swaps the factors of $\Cc_1$ and $\Cc_2$.
    \item For $\ZZ_2$-fixed points, the cellular chain complex $\Cc_\bullet(X^{\ZZ_2})$ is
      \[
        \begin{tikzcd}
          0 \rar & 0 \rar & 0 \rar & \ZZ^2 \rar & 0.
        \end{tikzcd}
      \]
      The automorphism group of $\ZZ_2 / \ZZ_2$ is trivial, so we don't need to care about the group actions.
      However, we do want to remember the homomorphism from this complex to the above complex (given by the identity on $\ZZ^2 = \Cc_0$).
  \end{itemize}
  To compute Bredon cohomology, we take $\Hom(\Cc_\bullet(X), \ul{\ZZ})$, where
  \[
    \ul{\ZZ}(\ZZ_2 / H) = \begin{cases}
      \ZZ \textrm{ with trivial action} & H = e \\
      \ZZ & H = \ZZ_2.
    \end{cases}
  \]
  Some computation (which I didn't entirely catch) shows that $\Hom(\Cc_\bullet(X), \ul{\ZZ})$ is
  \[
    \begin{tikzcd}
      0 \rar & \ZZ^2 \rar["\delta_1"] & \ZZ \rar["\delta_2"] \rar & \ZZ \rar & 0.
    \end{tikzcd}
  \]
  Here the middle $\ZZ$ is generated by $\phi$ satisfying $\phi(u, v) = u + v$.
  The differential $\delta_1$ satisfies $\delta_1(a, b) = a - b$, and $\delta_2 = 0$.
  Thus
  \[
    H^\bullet_{\ZZ_2}(S^2; \ul{\ZZ}) = \begin{cases}
      \ZZ & \bullet = 0, 2 \\
      0 & \textrm{else.}
    \end{cases}
  \]
\end{ex}

\section{(10/2) Gabriel Beiner -- Continued}

Recall that, to define Bredon cohomology, we choose a coefficient system $M \in \Fun(\Oc_G\op, \Ab)$.
For a $G$-CW complex $X$, we define
\[
  H^\bullet_G(X, M) = H^n(\Hom_{\Fun(\Oc_G\op, \Ab)}(\ul{\Cc}_\bullet^{\mathrm{CW}}(X), M)).
\]
Our goal today is to compute some more examples.

Recall that we can think of $\Fun(\Oc_G\op, \Ab)$ as an exit path category.
For $G = \ZZ_2$, the large ``open'' stratum corresponds to $\ZZ_2 / e$, while the small ``closed'' stratum corresponds to $\ZZ_2$.

One advantage of this approach is that we may choose $M$ to recover different features of $X$: we can compute cohomology of $X$, $X / G$, or $X^G$ through suitable choices of coefficient.

\subsection{An example}

\begin{ex}
  Let $G = \ZZ_4$ (with generator $\sigma$), and let $X$ be the star-shaped graph with six edges (it looks like $*$).
  We may view $X$ as (a union of segments on) the axes in $xyz$-space, and we let $G$ rotate the $xy$-plane and swap the positive and negative $z$-axes.
  Then $X$ has a natural $G$-CW complex structure, with:
  \begin{itemize}
    \item $0$-cells:
      \begin{itemize}
        \item A basepoint $\gamma = \ZZ_4 / \ZZ_4$.
        \item An orbit $\{ \beta_1, \beta_2 \} = \ZZ_4 / \ZZ_2 \times D^0$ corresponding to endpoints on the $z$-axis.
        \item An orbit $\{ \alpha_i \} = \ZZ_4 / e \times D^0$ corresponding to endpoints on the $x$- and $y$-axes.
      \end{itemize}
    \item $1$-cells:
      \begin{itemize}
        \item An orbit $\{ b_1, b_2 \} = \ZZ_4 / \ZZ_2 \times D^1$ corresponding to segments on the $z$-axis.
        \item An orbit $\{ a_i \} = \ZZ_4 / \ZZ_4 \times D^1$ corresponding to segments on the $x$- and $y$-axes.
      \end{itemize}
  \end{itemize}
  The cellular chain complex here is given by:
  \begin{itemize}
    \item $\ZZ_4 / e$ goes to $\ZZ^2 \oplus \ZZ^4 \to \ZZ \oplus \ZZ^2 \oplus \ZZ^4$.
      The differential is
      \[
        \begin{bmatrix}
          -1 & -1 & -1 & -1 & -1 & -1 \\
          1 & 0 & 0 & 0 & 0 & 0 \\
          0 & 1 & 0 & 0 & 0 & 0 \\
          0 & 0 & 1 & 0 & 0 & 0 \\
          0 & 0 & 0 & 1 & 0 & 0 \\
          0 & 0 & 0 & 0 & 1 & 0 \\
          0 & 0 & 0 & 0 & 0 & 1 
        \end{bmatrix}.
      \]
    \item $\ZZ_4 / \ZZ_2$ goes to $\ZZ^2 \to \ZZ \oplus \ZZ^2$.
      The differential is
      \[
        \begin{bmatrix}
          -1 & -1 \\
          1 & 0 \\
          0 & 1
        \end{bmatrix}
      \]
    \item $\ZZ_4 / \ZZ_4$ goes to $0 \to \ZZ$.
  \end{itemize}
  All of the transition maps (going from chain complexes below to chain complexes above) are given by the obvious inclusions.
  Here the stabilizers act by rotation on each copy of $\ZZ^4$, swaps on each copy of $\ZZ^2$, and the identity on each copy of $\ZZ$.

  If we use the constant coefficient system $\ZZ$, we get the total cochain complex
  \[
    \Hom(\ZZ, \ZZ) \oplus \Hom^{\ZZ_2}(\ZZ^2, \ZZ) \oplus \Hom^{\ZZ_4}(\ZZ^4, \ZZ) \to \Hom^{\ZZ_2}(\ZZ^2, \ZZ) \oplus \Hom^{\ZZ_4}(\ZZ^4, \ZZ).
  \]
  This may be identified with
  \[
    \ZZ^3 \xrightarrow{\delta} \ZZ^2
  \]
  where
  \[
    \delta = \begin{bmatrix}
      -1 & 1 & 0 \\
      -1 & 0 & 1
    \end{bmatrix}
  \]
  Since $\delta$ has full rank, we get 
  \[
    H^\bullet_{\ZZ_4}(X) = H^\bullet(X/\ZZ_4) = \begin{cases}
      \ZZ & \bullet = 0 \\
      0 & \textrm{otherwise.}
    \end{cases}
  \]
\end{ex}

\subsection{Another example}

We saw above that taking cohomology with respect to the constant coefficient system gives the cohomology of the usual quotient (not the Borel quotient).
Let's see what happens for more interesting coefficient systems.

\begin{ex}
  Let $\ZZ_2$ act on $S^2$ by reflection across the equator.
  Here we have a $G$-CW structure with:
  \begin{itemize}
    \item $0$-cell $\ZZ_2 / \ZZ_2 \times D^0$ sitting on the equator
    \item $1$-cell $\ZZ_2 / \ZZ_2 \times D^0$ sitting on the equator (connecting the $0$-cell to itself)
    \item $2$-cell $\ZZ_2 / e \times D^2$ giving the hemispheres.
  \end{itemize}
  The cellular chain complex sends:
  \begin{itemize}
    \item $\ZZ_2 / e$ to $\ZZ^2 \to \ZZ \to \ZZ$.
    \item $\ZZ_2 / \ZZ_2$ to $0 \to \ZZ \to \ZZ$.
  \end{itemize}
  The transition maps between orbits are inclusions.
  Here $\ZZ_2$ acts by swapping on the copy of $\ZZ^2$ and acts trivially on the copies of $\ZZ$.

  Consider the coefficient system $\ZZ \to \ZZ[i]$, where $\ZZ_2$ acts on $\ZZ[i]$ by conjugation.
  This is the sum of the constant coefficient system and a copy of $i\ZZ$ supported at $\ZZ_2 / e \in \Oc_G$.
  We can compute that the corresponding cochain complex is
  \[
    \begin{tikzcd}
      \ZZ \rar["0"] & \ZZ \rar & \ZZ^2
    \end{tikzcd}
  \]
  and by understanding the maps we see
  \[
    H^\bullet_{\ZZ_2}(S^2; M) = \begin{cases}
      \ZZ & \bullet = 0, 2 \\
      0 & \textrm{otherwise.}
    \end{cases}
  \]
\end{ex}

\subsection{Some last remarks}

Bredon cohomology may be characterized by Eilenberg--Steenrod axioms similarly to ordinary cohomology.
However, the dimension axiom is replaced by the axiom that 
\[
  H^\bullet_G(G/H, M) = \begin{cases}
    M(G/H) & \textrm{$\bullet = 0$} \\
    0 & \textrm{otherwise}
  \end{cases}
\]
We can define more general cohomology theories by relaxing the dimension axiom.

However, in general we want to do better: cohomology should be graded by the representation ring $\RO(G)$ rather than just $\ZZ$.
This can be done using \emph{Mackey functors}.

The analogue of the universal coefficient theorem for Bredon cohomology gives a spectral sequence
\[
  \Ext^{p,q}(\ul{\Cc}_\bullet^{\textrm{CW}}(X), M) \Rightarrow H^{p+q}(X; M).
\]

\section{(10/9) Gabriel Beiner -- Smith Theory}

Our goal today is to discuss the following result in the context of equivariant homotopy theory.

\begin{thm}
  Let $G$ be a finite $p$-group.
  Suppose $X$ is a finite $G$-CW complex which is an $\FF_p$-cohomology $n$-sphere (i.e.\ $H^\bullet(X; \FF_p) = H^\bullet(S^n; \FF_p)$).
  Then $X^G$ is either empty or a $\FF_p$-homology for $m \leq n$.

\end{thm}

One can also show the following claims:
\begin{itemize}
  \item If $p$ is odd, then $n - m$ is even.
  \item If $p$ is odd and $n$ is even, then $X^G$ is nonempty.
\end{itemize}

We'll follow an approach similar to Smith's original proof but with some simplifications using Bredon cohomology.

\subsection{Proof of the Smith inequality}

If $H \subset G$ is a normal subgroup, then we have $X^G = (X^H)^{G/H}$.
Arguing by induction, we reduce to proving the result in the case $G = \ZZ / p$.
We don't actually need to assume that $X$ is an $\FF_p$-homology sphere until the end of the proof, so let's drop this assumption for the moment.

The idea of the proof is to find coefficient systems $L$, $M$, and $N$ such that:
\begin{itemize}
  \item $H^\bullet(X; L) = \tilde{H}^\bullet((X / X^G) /  G; \FF_p)$,
  \item $H^\bullet(X; M) = H^\bullet(X; \FF_p)$, and
  \item $H^\bullet(X; N) = H^\bullet(X^G; \FF_p)$.
\end{itemize}
By the Eilenberg--Steenrod axioms, it will suffice to check that these results hold when $X$ is a $0$-cell.

The orbit category for $\ZZ/p$ is $\ZZ/p \to e$, where $\ZZ/p$ has automorphism group $\ZZ/p$.
We'll take:
\begin{itemize}
  \item $L$ is $\FF_p \leftarrow 0$, where $\ZZ/p$ acts trivially on the first factor.
  \item $M$ is $\FF_p[G] \leftarrow \FF_p$.
  \item $N$ is $0 \leftarrow \FF_p$.
\end{itemize}
To check the above homology equivalences, we plug in $X = G \times D^0$ and $X = G / G \times D^0$.

Consider the augmentation $\FF_p[G] \to \FF_p$ (given by $g \mapsto 1$), and let $I = \ker(\FF_p[G] \to \FF_p)$ be the augmentation ideal.
This gives rise to coefficient systems $I^n = (I^n \leftarrow 0)$.
We have two natural short exact sequences of coefficient systems:
\[
  \begin{tikzcd}
    0 \rar & I \rar & M \rar & L \oplus N \rar & 0
  \end{tikzcd}
\]
and
\[
  \begin{tikzcd}
    0 \rar & L \rar & M \rar & I \oplus N \rar & 0.
  \end{tikzcd}
\]
To understand the latter short exact sequence, note that:
\begin{itemize}
  \item At $e$, the SES becomes $0 \to \FF_p \to \FF_p$.
  \item At $\ZZ/p$, the SES becomes $((t-1)^{p-1}) \to \FF_p[t] / (t^p - 1) \xrightarrow{\cdot (t-1)} (t - 1)$.
\end{itemize}
Both of these are exact, so the sequence is an SES.

\begin{rmk}
  David pointed out that these relate to (and partially generalize) the short exact sequence building $\FF_2[\Sigma_2]$ as a self-extension of the trivial representation.
  Note that this short exact sequence does not split (because we are working with $\FF_2$ coefficients).
\end{rmk}

The SESes of coefficient systems give long exact sequences (where coefficients not mentioned are implicitly in $\FF_p$)
\[
  \begin{tikzcd}
    \dots \rar & H^q_G(X; I) \rar & H^q(X) \rar & \tilde{H}^q((X / X^G) / G) \oplus H^q(X^G) \rar & \dots
  \end{tikzcd}
\]
and
\[
  \begin{tikzcd}
    \dots \rar & \tilde{H}^q((X / X^G) / G) \rar & H^q(X) \rar & H^q_G(X; I) \oplus H^q(X^G) \rar & \dots.
  \end{tikzcd}
\]
Let's focus on Betti numbers for simplicity.
Write:
\begin{itemize}
  \item $a_q = \dim \tilde{H}^q((X / X^G) / G)$,
  \item $b_q = \dim H^q(X)$,
  \item $\ol{a}_q = H^q_G(X; I)$, and
  \item $c_q = \dim H^q(X^G)$.
\end{itemize}
Our long exact sequences give inequalities $a_q + c_q \leq b_q + \ol{a}_{q+1}$ and $\ol{a}_q + c_q \leq b_q + a_{q+1}$.
Combining these gives the key inequality (for $q, r \geq 0$ and $r$ odd):
\[
  a_q + c_q + c_{q+1} + \dots + c_{q+r} \leq b_q + b_{q+1} + \dots + b_{q+r} + a_{q+r+1}.
\]
Starting at $q = 0$ and continuing to $r \gg 0$, this gives $\sum_i c_i \leq \sum_i b_i$.
In other words,
\[
  \sum_i \rk H^i(X^G; \FF_p) \leq \sum_i \rk H^i(X; \FF_p).
\]
This is the ``Smith inequality'' and does not use anything other than the fact that $X$ is a finite $G$-CW complex.

\subsection{Finishing the proof of Smith's theorem}

Now taking $X$ to be an $\FF_p$-homology sphere, we see that
\[
  \sum_i \rk H^i(X^G; \FF_p) \leq 2.
\]
To see that $X^G$ is an $\FF_p$-homology sphere, it suffices to show that $\sum_i \rk H^i(X^G; \FF_p) \neq 1$.

To show this, first note that $I^n / I^{n+1} \cong \FF_p$.
From the short exact sequence
\[
  \begin{tikzcd}
    0 \rar & I^{n+1} \rar & I^n \rar & L \rar & 0,
  \end{tikzcd}
\]
we get $\chi(X; I^n) = \chi(X; I^{n+1}) + \chi(X; L)$.
Summing these up gives
\[
  \sum_{i=1}^{p-1} \chi(X; I^i) = \sum_{i=1}^{p-1} \big(\chi(X; I^{i+1}) + \chi(X; L)\big)
\]
and hence (using $I^p = 0$)
\[
  \chi(X; I) = \chi(X; I^p) + (p - 1) \chi(X; L) = (p - 1) \chi(X; L).
\]
Using $\chi(X; M) = \chi(X; I) + \chi(X; L) + \chi(X; N)$, we obtain
\[
  \chi(X; M) = \chi(X; N) + p \chi(X; L).
\]
We may rewrite this as $\chi(X) = \chi(X^G) + p \tilde{\chi}((X / X^G) / G)$.
This forces $\chi(X) \equiv \chi(X^G) \pmod p$.
(Again, this claim doesn't need $X$ to be an $\FF_p$-homology sphere.)
We never have $1 \equiv 2 \pmod p$, so we must have $\sum_i \rk H^i(X^G; \FF_p) \neq 1$.

\subsection{Harnack's curve theorem}

Here's an application of Smith theory to algebraic geometry.

\begin{thm}[Harnack]
  Let $X_\RR$ be the real locus of a smooth projective plane curve of genus $g$.
  Then the number of connected components of $X_\RR$ is bounded above by $g + 1$.
\end{thm}

\begin{proof}
  We have $X_\RR = (X_\CC)^{\ZZ/2}$, so
  \[
    \sum_i \rk H^i(X_\RR; \FF_2) \leq \sum_i \rk H^i(X_\CC; \FF_2).
  \]
  The former is twice the number of connected components of $X_\RR$ (which are copies of $S^1$), and the latter is $2 + 2g$.
\end{proof}

\section{(10/16) Joakim F{\ae}rgeman -- Local Constancy of Automorphic Sheaves over the Moduli of Curves}

Today's talk is a research talk (not about equivariant homotopy theory).
This is based on joint work with Marius Kj{\ae}rsgaard.

\subsection{De Rham Geometric Langlands}

Let $X$ be a smooth projective curve over $\CC$ and $G$ a connected reductive algebraic group (e.g.\ $G = \GL_n$, $\SL_n$, $\PGL_n$, $\SO_{2n+1}$, etc).
The de Rham geometric Langlands correspondence relates the following data.

On the automorphic side, we consider the moduli stack $\Bun_G(X)$ of $G$-bundles on $X$.
We think about the (unbounded derived) category $D(\Bun_G(X))$ of $D$-modules on $\Bun_G(X)$.

\begin{ex}
  If $G = \GL_n$, $\Bun_G(X)(\CC)$ classifies rank $n$ vector bundles on $X$.
\end{ex}

On the spectral side, we have a Langlands dual group $\check{G}$ with root data dual to that of $G$.
Let $\LS_{\check{G}}(X)$ be the stack of $\check{G}$-bundles equipped with a flat connection.
We focus on the category $\QCoh(\LS_{\check{G}}(X))$ of quasicoherent sheaves on $\LS_{\check{G}}(X)$.

\begin{ex}
  Here are the Langlands duals of some common groups:
  \begin{itemize}
    \item The Langlands dual of $\GL_n$ is $\GL_n$.
    \item The Langlands dual of $\SL_n$ is $\PGL_n$.
    \item The Langlands dual of $\SO_{2n+1}$ is $\Sp_{2n}$.
  \end{itemize}
\end{ex}

\begin{ex}
  For $G = \GL_n$, we have $\check{G} = \GL_n$, so $\LS_{\check{G}}(X)(\CC)$ classifies rank $n$ vector bundles on $X$ with a flat connection.
\end{ex}

The de Rham geometric Langlands correspondence is something like an equivalence
\[
  D(\Bun_G(X)) \simeq \QCoh(\LS_{\check{G}}(X)).
\]
For this to be literally true, the right hand side of this should be replaced by ind-coherent sheaves with nilpotent singular support.
Furthermore, we want the equivalence here to satisfy some nice expected properties.

\subsection{Betti Geometric Langlands}

Here's a modified version of the geometric Langlands correspondence, replacing $\LS_{\check{G}}(X)$ with topological local systems on $X$.
This was proposed by David Ben-Zvi and our own David Nadler in 2016.

On the spectral side, we let $\LS_{\check{G}}^\toprm(X) = \Map(\pi_1(X), \check{G}) / \check{G}$.
More concretely, if $X$ has genus $g$, then we may write $\pi_1(X) = \langle \{ a_i \}_{i=1}^g, \{ b_i \}_{i=1}^g | \prod_i [a_i, b_i] = 1 \rangle$.
This lets us write $\LS_{\check{G}}^\toprm(X) = (\check{G}^{2g} \times_{\check{G}} \{ 1 \}) / \check{G}$, where the map $\check{G}^{2g} \to \check{G}$ is given by the product of commutators.

The automorphic side is a bit more complicated.
Given a smooth algebraic stack $\Yfr$ over $\CC$, we may define a category $\Sh(\Yfr)$ of sheaves of $\CC$-vector spaces on the underlying topological space of $\Yfr^\mathrm{an}$.
We don't assume that the fibers are finite dimensional.
If $\Lambda \subset T^* \Yfr$ is a closed conical Lagrangian, we define $\Sh_\Lambda(\Yfr)$ to be the full subcategory of $\Sh(\Yfr)$ consisting of sheaves $\Fsc$ with singular support $\SS(\Fsc) \subset \Lambda$.
Here $\SS(\Fsc)$ is a closed conical subset of $T^* \Yfr$ ``measuring where $\Fsc$ is not locally constant.

\begin{ex}
  If $\Lambda$ is the zero section, then $\Sh_\Lambda(\Yfr)$ is the category of local systems on $\Yfr$.
\end{ex}

\begin{ex}
  Let $\Yfr = \AA^1$, and let $j: \AA^1 \setminus 0 \hookrightarrow \AA^1$ be the inclusion.
  If $\Fsc = j_*(\CC_{\AA^1 \setminus 0})$, then $\SS(\Fsc)$ is the union of the coordinate axes in $T^* \AA^1 = \AA^2$.
\end{ex}

If $\Yfr = \Bun_G(X)$, then $T^*\Bun_G(X)$ is the moduli of pairs $(\Pc_G, \phi)$ where $\Pc_G$ is a $G$-bundle and $\phi \in \Gamma(X, \gfr^*_{\Pc_G} \otimes \Omega^1_X)$.
(Proving this is an exercise in Serre duality.)
We take $\Lambda$ to be the nilpotent cone $\Nilp(X) \subset T^*\Bun_G(X)$, i.e.\ the subset where $\phi$ is nilpotent.

\begin{ex}
  For $G = \GL_n$, we have $T^* \Bun_{\GL_n} = \{ (E, \phi: E \to E \otimes \Omega^1_X) \}$.
\end{ex}

The Betti geometric Langlands correspondence is an equivalence
\[
  \Sh_{\Nilp(X)}(\Bun_G(X)) \simeq \QCoh(\LS_{\check{G}}^\toprm(X)).
\]

\begin{rmk}
  David explained that this came about as an attempt to understand what physicists call ``geometric Langlands.''
  The left hand side is ``topology on top of algebra'' while the right hand side is ``algebra on top of topology.''
\end{rmk}

\begin{rmk}
  The Riemann--Hilbert correspondence gives an analytic equivalence $\LS_{\check{G}}(X) \simeq \LS_{\check{G}}^\toprm(X)$.
  This is not an algebraic equivalence!
  Betti geometric Langlands and de Rham geometric Langlands are compatible with the Riemann--Hilbert equivalence.
\end{rmk}

\subsection{The problem}

The space $\LS_{\check{G}}^\toprm$ depends only on the topological type of $X$, so Betti geometric Langlands implies that the category $\Sh_{\Nilp(X)}(\Bun_G(X))$ is independent of the complex structure of $X$.
This is surprising, as $\Bun_G(X)$ and $\Nilp(X)$ depend on the complex structure of $X$!
We'd like to understand why this claim is true.
If $G$ is abelian, the claim is not too difficult to show directly.

\begin{thm}[F--Kj{\ae}rsgaard]
  The category $\Sh_{\Nilp(X)}(\Bun_G(X))$ is independent of the complex structure of $X$.
  Furthermore, this can be proved without invoking Betti geometric Langlands.
\end{thm}

\begin{rmk}
  David pointed out that what's happening here is based on topological field theory.
  We'd really like to understand what happens for $S^1$ and the point.
  These are hard!
  The problem of extracting 3-manifold invariants is also quite difficult.
\end{rmk}

To understand the problems, let's focus on an example.

\begin{ex}
  Consider $G = \SL_2$.
  Then $\Bun_{\SL_2}(X)$ is the moduli of rank $2$ vector bundles $E$ on $X$ with $\det(E) = \Osc_X$.
  We may write $\Nilp(X) = \cup_{d \geq 1 - g} \ol{\Nilp(X)^d} \cup \Bun_G(X)$, where $\Nilp(X)^d$ consists of extensions
  \[
    \begin{tikzcd}
      0 \rar & \Lsc \rar & E \rar & \Lsc^\vee \rar & 0
    \end{tikzcd}
  \]
  (where $\deg \Lsc = d$) together with $0 \neq \phi: \Lsc^\vee \to \Lsc \otimes \Omega_X^1$.
  The inclusion $\Nilp(X)^d \to \Nilp(X)$ comes from sending the above data to the pair
  \[
    (E, E \to \Lsc^\vee \xrightarrow[\phi] \Lsc \otimes \Omega^1_X \to E \otimes \Omega^1_X).
  \]
  The geometry of $\Nilp(X)^d$ depends on $\Hom(\Lsc^\vee, \Lsc \otimes \Omega^1_X) = \Gamma(X, \Lsc^2 \otimes \Omega^1_X)$.
  We can try to compute this using Riemann--Roch.

  If $\Lsc$ is a line bundle of degree $d$ on $X$, then Riemann--Roch tells us:
  \begin{itemize}
    \item If $\deg \Lsc < 0$, then $h^0(\Lsc) = 0$.
    \item If $\deg \Lsc > 2g-2$, then $h^0(\Lsc) = d + 1 - g$.
  \end{itemize}
  When $0 < d < 2g-2$, we have to understand Brill--Noether theory.
  Let $U^r_d(X)$ be the ``Brill--Noether locus'' of $\Lsc$ of degree $d$ with $h^0(\Lsc) = r + 1$.
  The Brill--Noether theorem tells us that $U^0_d(X)$ is smooth and that $U^r_d(X)$ is smooth for a generic curve $X$.
  Generically $\dim U^r_d(X) = g - (r + 1)(g - d + r)$.

  The upshot is that the geometry of $\Nilp(X)$ is controlled by the geometry of the Brill--Noether loci, which in turn depend on the complex structure of $X$.
\end{ex}

To prove the theorem:
\begin{itemize}
  \item When $G$ has semisimple rank $1$, there is a direct proof based on na\"ive bounds for singular support.
    This proof doesn't adapt well in general.
  \item For general $G$, one shows that there is a specific part of $\Nilp(X)$ ``controlling'' $\Sh_{\Nilp(X)}(\Bun_G(X)$, and this part behaves well when you change $X$.
    This is what you need to do to prove things in general, but it's hard!
\end{itemize}

\section{(10/23) Jakub L\"owit -- Equivariant $K$-Theory and Trace Maps in GRT}

This is another guest talk.
There will be no seminar next week (but there will still be lunch).
In two weeks we will begin to talk about spectra and equivariant stable homotopy theory.

\subsection{Fixed-point schemes}

Let $k$ be a field, $G$ a reductive group over $k$ (usually taken to be $\GL_n$ or a diagonal torus), and $X$ a qcqs scheme over $k$ with $G$-action.

\begin{dfn}
  The \emph{fixed-point scheme} $\Fix_G(X)$ is the fiber product appearing in the Cartesian square
  \[
    \begin{tikzcd}
      \Fix_G(X) \rar & G \times X \dar["\rho \times \pi_2"] \\
      X \rar["\Delta"] & X \times X.
    \end{tikzcd}
  \]
  In terms of $k$-points, we have $\Fix_G(X) = \{ (g, x) | g \cdot x = x \}$.
  There is a natural $G$-action on $\Fix_G(X)$, and we write $\Fix_{G/G}(X)$ for the quotient.
\end{dfn}

\begin{rmk}
  There are some important variants of this construction: we write $\Fix^\red$ for the reduced fixed-point scheme and $\Fix^\Lbf$ for the derived fixed-point scheme.
\end{rmk}

\begin{ex}
  Let $X = \pt$ with the trivial $G$-action.
  Then $\Fix_G(X) = G$, and $\Fix_{G/G}(X) = G/G$ is the adjoint quotient.
\end{ex}

\begin{ex}
  Let $X = G / P$ be a partial flag variety, considered with its $G$-action.
  (As a sub-example, we could consider the action of $\GL_n$ on projective space.)
  Then $\Fix_G(X)$ is a (multiplicative, partial) version of the Grothendieck--Springer alteration.
  The fibers of $\Fix_G(X)$ over points of $G$ may be understood in terms of the Jordan normal form of points of the base.
  Fibers over regular semisimple elements are discrete.
  The fiber over the identity is all of $G / P$.
\end{ex}

\begin{rmk}
  David pointed out that $\Fix^\Lbf_{G/G}(X)$ may also be interpreted as the free loop space of $X / G$.
\end{rmk}

\subsection{Equivariant $K$-theory and trace maps}

Continue with the above setup.
The \emph{equivariant $K$-theory} $K^G(X)$ is $K(\Perf(G \backslash X))$, the non-connective algebraic $K$-theory spectrum.
We get abelian groups $K^G_i(X)$ (indexed by $i \in \ZZ$) forming a $\ZZ$-graded ring.
The group $K^G_0(X)$ is the Grothendieck ring of $G$-equivariant vector bundles on $X$.

Here is the ``kindergarten'' / degree $0$ version of the Dennis trace map.
This is a map $\tr: K_0^G(X) \otimes k \to H_0(\Fix_{G/G}(X), \Osc)$ sending an equivariant vector bundle $\Esc$ to the function
\[
  (g, x) \mapsto \tr_g \Esc_x.
\]
One may check that this gives a well-defined ring homomorphism.

\begin{rmk}
  This is still interesting when $X$ is singular.
  In this case we want to use $\Perf$ instead of $\Coh$.
\end{rmk}

\begin{ex}
  For $X = \pt$ with trivial $G$-action, the Dennis trace map is a map $K^G_0(\pt) \otimes k \to H_0(\Fix_{G/G}(pt), \Osc)$.
  In other words, it is a map from the representation ring $R_G \otimes k$ to $H_0(G / G, \Osc)$, the ring of class functions on $G$.
  In this case, the Dennis trace map is a ring isomorphism.
\end{ex}

\begin{ex}
  Let's suppose for simplicity that $k = \CC$ and $G = \GL_n$.
  For $X = G / P$, the Dennis trace map sends $K_0^G(X) \otimes k$ to $H_0(\Fix_{G/G}(X), \Osc)$.
  This is an isomorphism, and both sides may be identified with $k[T]^{W_P}$.
\end{ex}

Let's get a more conceptual ``grown-up'' understanding of the Dennis trace map.
As mentioned above, we may identify $\Fix^\Lbf_{G/G}(X) \simeq \Lc(G \backslash X) \simeq \Map(S^n, G \backslash X)$.
This allows us to understand the Dennis trace via Hochschild homology $HH^G(X / k)$.
There's a natural HKR map $HH^G(X / k) \to \Rbf \Gamma(\Fix^\Lbf_{G/G}(X), \Osc)$.
In nice cases (e.g.\ in characteristic zero) this map is an equivalence.
For general categorical reasons, we have a ``true'' Dennis trace map $K^G(X, k) \to HH^G(X / k)$.
Composing the Dennis trace map and the HKR map gives a map $K^G(X, k) \to \Rbf \Gamma(\Fix^\Lbf_{G/G}(X), \Osc)$.

\subsection{Perfection in characteristic $p$}

Suppose now that $k = \FF_p$.
Any scheme over $\FF_p$ has a Frobenius endomorphism $\phi$.
We say that $X$ is perfect if $\phi$ is an isomorphism.

\begin{ex}
  The affine line $\AA^1$ is not perfect (we cannot take $p$th roots of arbitrary functions).
\end{ex}

Given a ring $R$, we can consider the \emph{perfection}
\[
  R_\perf = \colim(R \xrightarrow{\phi} R \xrightarrow{\phi} R \xrightarrow{\phi} \dots).
\]
This construction globalizes to give
\[
  X_\perf = \lim(X \xleftarrow{\phi} X \xleftarrow{\phi} \dots).
\]
The result is typically non-noetherian.
However, it does have useful properties, e.g.\ it may be easier to compute resolutions of $X_\perf$.

Let $\Gr$ be the perfection of the affine Grassmannian of $\GL_n$.
This has a stratification by (perfected) affine Schubert varieties $X_{\leq \mu}$.
Let $T \subset \GL_n$ be the diagonal torus.
Then $T$ acts on each $X_{\leq \mu}$.

\begin{thm}[L]
  The trace map $K^T(X_{\leq \mu}, k) \to \Rbf \Gamma(\Fix_T/T(X_{\leq \mu}), \Osc)$ is an equivalence.
  Moreover, the cohomologies on each side are concentrated in degree $0$.
  We also have $K^T(X_{\leq \mu}) \simeq KH^T(X_{\leq \mu})$ where $KH$ is some homotopy-invariant version of $K$-theory.
\end{thm}

The idea of the proof is to show things for points and use projective bundles to extend the results.
This allows us to establish the results for Bott--Samelson resolutions.
Descending the results to affine Schubert varieties is more technical.

\end{document}
