\documentclass{article}

\usepackage{notes}

\title{GRT Seminar Fall 2024 -- Rozansky-Witten Theory}
\author{Notes by John S.\ Nolan, speakers listed below}

\begin{document}

\maketitle

\begin{abstract}
	This semester, the GRT Seminar will focus on Rozansky-Witten theory.
\end{abstract}

\tableofcontents

\section{1/23 (David Nadler) -- Introduction}

Our goal is to discuss Rozansky-Witten theory.
Some related topics include:
\begin{itemize}
	\item Quasicoherent sheaves of categories (as discussed last spring).
	\item Categories of matrix factorizations.\footnote{In more detail: given a smooth variety $X$ and a function $f: X \to \AA^1$, we can construct a category $\MF_f$ which categorifies the vanishing cycles of $f$.}
	\item The cobordism hypothesis.
	\item Local structure theory of holomorphic symplectic varieties.
\end{itemize}

\subsection{What is Rozansky-Witten theory?}

Suppose we have a hyperk\"ahler / holomorphic symplectic manifold $X$.
This means that $X$ has a holomorphic $(2, 0)$-form $\omega$ satisfying the (complex analogues of) the usual symplectic form axioms.
Given such an $X$, there is a conjectural 3-dimensional topological field theory $\Zc_X$, called \emph{Rozansky-Witten theory} with target $X$.

What we mean by 3d TFT is as follows:
\begin{itemize}
	\item Given a closed 3-manifold\footnote{Typically with some extra structure, e.g.\ an orientation} $M^3$, we obtain a number $\Zc_X(M^3)$.
	\item Closed 2-manifolds $M^2$ give vector spaces $\Zc_X(M^2)$.
	\item Closed 1-manifolds $M^1$ give categories\footnote{As is standard for GRT, we use the implicit $\infty$ convention.} $\Zc_X(M^1)$.
	\item Closed 0-manifolds $M^0$ give 2-categories $\Zc_X(M^0)$.
\end{itemize}
In particular, $\Zc_X(\pt)$ is a 2-category.
The \emph{cobordism hypothesis} tells us that we can recover the entire theory $\Zc_X$ from the ``3-dualizable'' 2-category $\Zc_X(\pt)$.
For purposes of geometric representation theory, we are most interested in the low-dimensional behavior, which captures more data about the theory.

Rozansky-Witten theory should satisfy something like:
\begin{itemize}
	\item $Z_X(S^2) = \Oc(X)$.\footnote{By our conventions, this is what is classically called $\Rbf \Gamma(X, \Oc)$, so there is interesting derived information.}
	\item $Z_X(S^1) = \Coh(X)$.
\end{itemize}
These end up inheriting interesting structure from the TFT.

\subsection{Why do we care?}

Recall that 2-dimensional mirror symmetry can be schematically understood as an equivalence between the following 2d TFTs:
\begin{itemize}
	\item An A-model $\Ac$ arising from symplectic geometry
	\item A B-model $\Bc_X$, coming from some K\"ahler manifold $X$, satisfying $\Bc_X(\pt) \simeq \Coh(X)$.
\end{itemize}
In particular, $\Ac(\pt)$ is often some category of geometric interest, and the equivalence $\Ac(\pt) \simeq \Bc_X(\pt)$ lets us resolve questions about $\Ac(\pt)$.

There's an analogue in higher dimensions: we'd like to take a 3d TFT $\Yc$ and give an equivalence $\Yc \simeq \Zc_X$ for some holomorphic symplectic $X$.
This would give an equivalence between some 2-category and $\Zc_X(\pt)$.

\begin{conj}[Teleman]
	Let $G$ be a complex reductive group with maximal compact subgroup $G_c$.
	There is an equivalence between:
	\begin{itemize}
		\item A suitable 2-category of ``categories with $G_c$-action.''
		\item The Rozansky-Witten 2-category of $T^*(G^\vee / G^\vee)$.
	\end{itemize}
\end{conj}

Note that $T^*(G^\vee / G^\vee)$ is stacky and non-proper, which makes it impossible for the corresponding 2-category to be 3-dualizable.
Thus we typically won't obtain 3-manifold invariants from such a theory.
That's terrible for 3-manifold topologists, but this isn't a 3-manifold seminar.

Some other examples of interest for Rozansky-Witten theory include symplectic resolutions and cotangent bundles of smooth algebraic varieties.

\subsection{What is the correct 2-category?}

To rigorously construct Rozansky-Witten theory, we'd need to give a definition of the 2-category $\RW_2 = \Zc_X(\pt)$.
This was studied by Kapustin, Rozansky, and Saulina, but much is still unknown.

Roughly, we expect $\RW_2$ to be a 2-category where:
\begin{itemize}
	\item Objects are smooth Lagrangians $L \subset X$ (or some suitable generalization of these).
	\item 1-morphisms from $L_1$ to $L_2$ are given by some sort of category associated to $L_1 \cap L_2$.
		In the simplest possible case, where $X = T^* W$ is a cotangent bundle, $L_1$ is the zero-section, and $L_2$ is the graph of a differential $df$, then $L_1 \cap L_2$ is the critical locus of $X$ and we assign $\Hom(L_1, L_2) = \MF_f$, the matrix factorization category of $f$.
		Work of Joyce and many others has focused on understanding how much the local setting looks like this.
	\item 2-morphisms and higher are ``natural compatibilities'' between the 1-morphisms.
\end{itemize}

One should think of the matrix factorization category $\MF_f$ as giving a categorical way to measure the critical locus of $f$.
When the critical points of $f$ are Morse, the category $\MF_f$ looks like a direct sum of copies of $\Vect$ (one for each critical point).

There is an important distinction between Rozansky-Witten theory and the 2d A-model.
In the complex setting, there are no ``instantons,'' so the theory is local and we don't run into the full difficulty of Floer theory.
Thus Rozansky-Witten theory is a categorified version of Fukaya theory that avoids the need for instanton corrections.

\subsection{An alternative viewpoint}

If $X = T^* W$ is a cotangent bundle, then $\ShvCat(W)$, the 2-category of (quasicoherent) sheaves of categories on $W$, embeds into $\RW_2$.
The image of this embedding consists of ``conic objects.''
Thus we can understand a key part of Rozansky-Witten theory, at least in this simple case.

The thesis (work in progress) of Enoch Yiu relates $\RW_2$ to $\ShvCat(W \times \AA^1)$.

\section{1/30 (Daigo Ito) -- Theory of Critical Points and Matrix Factorizations}

Recall that we wanted to understand the Rozansky-Witten theory of a holomorphic symplectic variety $M$.
By the cobordism hypothesis, it suffices to understand the 2-category $\RW_2(M)$.
We expect $\RW_2(M)$ to have some vague properties as follows.

The objects of $\RW_2$ should be holomorphic Lagrangians in $M$ (possibly equipped with extra data).
If $M = T^* L_1$, then we should have $\Hom_{\RW_2}(L_1, L_2) = \MF(L_1, f)$, the category of \emph{matrix factorizations} of $f$.
This measures the local geometry of $p \in L_1 \cap L_2 = \Crit(f)$.

Recall the two key differences between this and Lagrangian Floer homology:
\begin{itemize}
	\item There are no instantons, so the full subtleties of Floer theory don't appear.
	\item We are working at a higher category level.
\end{itemize}

Today we will recall the theory of critical points for a function $f: X \to \AA^1$.

\subsection{Milnor fibers}

Let's start by considering a regular map $f: \CC^n \to \CC$.
Assume that $0 \in \CC$ is a critical value.
Call $X_0 = f\inv(0)$ the special fiber -- this is typically singular.
For small $s \in \CC$, let $X_s = f\inv(s)$ be the nearby fiber.

\begin{thm}[Milnor]
	Let $x \in X_0$.
	For $\epsilon > 0$ sufficiently small, let $B(x, \epsilon)$ be the closed ball of radius $\epsilon$ centered at $x$, and let $S(x, \epsilon) = \partial B(x, \epsilon)$.
	Then:
	\begin{enumerate}
		\item $B(x, \epsilon) \cap X_0$ is homeomorphic to the cone over $K_x = S(x, \epsilon) \cap X_0$.
		\item The map $\rho_f = \frac{f}{\abs{f}} : S(x, \epsilon)  \setminus K_x \to S^1$ is a locally trivial fibration.
			We call $\rho_f$ the \emph{Milnor fibration} and the fiber $F_x$ the \emph{Milnor fiber}.
	\end{enumerate}
\end{thm}

The Milnor fibers $F_x$ degenerate to the cone over $K_x$.

\begin{ex}
	If $x$ is nonsingular, then $K_x$ is a sphere, so the cone over $K_x$ is a ball.
	The Milnor fibers $F_x$ are also balls.
\end{ex}

The topology of the Milnor fibers reflects ``how singular the point is'' -- a more singular point leads to a more complicated topology.

\begin{ex}
	Let $(X_0, x) = (z_1^2 - z_2^2 = 0, 0)$.
	Then $F_x$ is homotopy equivalent to $S^1$.
	Looking at real points, the map $f$ describes a family of hyperbolas degenerating to a union of lines.
	Here $\partial B = S^3$ and $K_x = S^1 \sqcup S^1$, so topologically $K_x$ is a double cone.
	The Milnor fibers form a family of cylinders degenerating to this double cone.
\end{ex}

\begin{ex}
	Let $(X_0, x) = (z_1^3 - z_2^2 = 0, 0)$.
	Then $K_x$ is a trefoil knot
	\[
		\bset{(r_1 e^{2\pi i t}, r_2 e^{2 \pi i t})}{t \in \RR} \subset S^1_{r_1} \times S^1_{r_2}.
	\]
	The closures of the Milnor fibers are genus one ``Seifert surfaces'' for $K_x$.
	Thus the Milnor fibers are homotopy equivalent to $S^1 \wedge S^1$.
\end{ex}

More generally, if $(X, x)$ is an isolated hypersurface singularity, then we can write $F_x \simeq (S^n)^{\vee \mu_x}$, where $\mu_x$ is the \emph{Milnor number}.\footnote{There is an explicit formula for the Milnor number, but we won't write it here.}
The $S^n$'s here are the \emph{vanishing cycles} of the singularity.

\subsection{Monodromy}

The singularity carries information beyond the Milnor fibers.
We can capture some of this by looking at the monodromy.

\begin{dfn}
	The \emph{monodromy} of $f$ at $x$ is the map $h_f: F_x \to F_x$ induced by circling around the base.
	This is a homeomorphism of $F_x$ which restricts to the identity on $\partial F_x$.
	Note that $h_f$ is only well-defined up to isotopy (fixing $\partial F_x$).
\end{dfn}

\begin{ex}
	For a Morse function $f = \sum_i x_i^2$, the Milnor fibers are homotopy equivalent to $S^n$.
	We understand the singularity by studying the monodromy of the Milnor fibers as we move around the singular point.
	This monodromy is a Dehn twist, ``corkscrewing'' the cylinder.
\end{ex}

\begin{thm}[Thom-Sebastiani]
	Let $f: (\CC^{n+1}, 0) \to (\CC, 0)$ and $g: (\CC^{m+1}, 0) \to (\CC, 0)$ be germs of hypersurface singularities.
	Define $f \boxplus g: (\CC^{n+1} \times \CC^{m+1}, 0) \to (\CC, 0)$ by $(f \boxplus g)(x, y) = f(x) + g(y)$.
	Then there is a homotopy-commutative diagram
	\[
		\begin{tikzcd}
			F_f * F_g \rar["\sim"] \dar["h_f * h_g"] & F_{f \boxplus g} \dar["h_{f \boxplus g}"] \\
			F_f * F_g \rar["\sim"] & F_{f \boxplus g},
		\end{tikzcd}
	\]
	where $*$ is the join of spaces.
\end{thm}

\subsection{Preview}

Next time we will introduce sheaves that describe the homology of these spaces.
We get a fiber sequence
\[
	\begin{tikzcd}
		i^*\Fc \rar & \psi_f \Fc \rar & \phi_f \Fc \rar & {}
	\end{tikzcd}
\]
of sheaves on $X_0$, where:
\begin{itemize}
	\item $i: X_0 \to X$ is the inclusion,
	\item $\psi_f$ is nearby cycles, and
	\item $\phi_f$ is vanishing cycles.
\end{itemize}

This will categorify to a sequence
\[
	\begin{tikzcd}
		\Perf(X_0) \rar & \Dbcoh(X_0) \rar & \Dsing(X_0).
	\end{tikzcd}
\]
where $\Dsing(X_0)$ agrees with $\MF(X, f)$ in nice cases.

\section{2/6 (Daigo Ito) -- Continued}

Last time we discussed the construction of Milnor fibers, vanishing cycles, monodromy, and Thom-Sebastiani isomorphisms for $f: \CC^{n+1} \to \CC$.
Today we would like to discuss and categorify these stories in a sheaf-theoretic framework.

\subsection{Vanishing and nearby cycles}

Let $\DD \subset \CC$ be a small disk around $0$.
Let $X$ be (an open subset of) a smooth algebraic variety over $\CC$, and let $f: X \to \DD$ be a map.
Consider the diagram 
\[
	\begin{tikzcd}
		X_0 \rar["i"] \dar & X \dar["f"] & \lar["j", swap] X^* \dar["f^*"] & \lar["\tilde{\pi}", swap] \tilde{X^*} \dar \\
		\{0\} \rar & \DD & \lar \DD^* & \lar["\pi", swap] \tilde{\DD^*}
	\end{tikzcd}
\]
where all squares are pullback squares and $\pi: \tilde{\DD^*} \to \DD^*$ is the universal cover $z \mapsto \exp(2\pi i z)$.
Note that $X^* \simeq X_s$ for $s$ small, $s \neq 0$.

Write $\Dbc(-)$ for the bounded constructible derived category of $A$-modules on a space (where $A$ is some fixed coefficient ring, typically $\ZZ$ or $\CC$).
Recall that ``constructible'' means locally constant on the strata of a nice stratification and with finite-rank stalks.

\begin{dfn}
	The \emph{nearby cycle functor} associated with $f$ is $\psi_f: \Dbc(X) \to \Dbc(X_0)$, defined by\footnote{All functors here are derived.}
	\[
		\Fc \mapsto i^* (j \circ \tilde{\pi})_* (j \circ \tilde{\pi})^* \Fc.
	\]
\end{dfn}

Morally, we have a (very non-analytic) specialization map $\spop: X_s \to X_0$, and $\psi_f = \spop_*(\Fc|_{X_s})$.
The previous definition is used to avoid referencing $\spop$.

\begin{ex}
	Consider $f: \DD \to \DD$ by $f(z) = z^2$.
	For $\Fc = \ul{A}_\DD$, we can compute $\psi_f(\Fc) = A_0 \oplus A_0$, reflecting the fact that the nearby fibers have two points.
	The monodromy map swaps the two factors: this can be seen directly using the specialization definition or by considering deck transformations using the formal definition.
\end{ex}

\begin{rmk}
	David mentioned that one can actually rephrase this story so that the only $f$ which we consider is projection to the first coordinate.
	The cost is that we are forced to work with arbitrarily complicated sheaves.
	The reverse (working with the constant sheaf but allowing arbitrarily complicated $f$) is not possible in general, though there is a related theory of ``sheaves of geometric origin.''
\end{rmk}


From the pushforward-pullback adjunction, there is a natural map $r: i^* \Fc \to i^* (j \circ \tilde{\pi})_* (j \circ \tilde{\pi})^* \Fc = \psi_f \Fc$.

\begin{dfn}
	We define the \emph{vanishing cycle functor} $\phi_f: \Dbc(X) \to \Dbc(X_0)$ by $\phi_f(\Fc) = \cone(r)$, so there is a cofiber sequence
	\[
		\begin{tikzcd}
			i^*\Fc \rar & \psi_f \Fc \rar & \phi_f \Fc \rar & {}.
		\end{tikzcd}
	\]
\end{dfn}

We call $\psi_f \ul{A}_X$ (resp.\ $\phi_f \ul{A}_X$) the \emph{nearby (resp.\ vanishing) cycle complex} associated with $f$.
From the cofiber sequence containing these, we obtain a long exact sequence (using $H^*(X) \cong H^*(X_0)$):
\[
	\begin{tikzcd}
		\dots \rar & H^*(X_0) \rar & H^*(X_s) \rar & H^*(X, X_s) \rar & \dots
	\end{tikzcd}
\]
This encompasses much of our discussion from last time.

\begin{prop}
	For $f: \CC^{n+1} \to \CC$, if $X_0$ has only isolated singularities (so $F_f \simeq \vee S^n$), then
	\[
		H^k(X_0, \phi_f \ul{A}_X) = \begin{cases}
			0 & k \neq n \\
			\oplus_{x \in \Sing(X_0)} H^n(F_{f,x}; A) & k = n.
		\end{cases}
	\]
\end{prop}

\begin{rmk}
	One can obtain the monodromy of nearby / vanishing cycles using the deck transformations of $\tilde{\DD^*}$.
	There's also a Thom-Sebastiani theorem in the sheaf-theoretic setting.
\end{rmk}

\subsection{Singularity categories and matrix factorizations}

We can categorify the preceding story using the exact sequence of categories:
\[
	\begin{tikzcd}
		\Perf(X_0) \rar & \Dbcoh(X_0) \rar & \Dsing(X_0),
	\end{tikzcd}
\]
where $\Dsing(X_0)$ is defined as the quotient $\Dbcoh(X_0) / \Perf(X_0)$.
In nice cases, $\Dsing(X_0)$ agrees with the \emph{matrix factorization category} $\MF(X, f)$.
The reason $\Dsing$ is called the ``category of singularities'' is the following:

\begin{prop}
	$X_0$ is smooth if and only if $\Perf(X_0) \simeq \Dbcoh(X_0)$.
\end{prop}

\begin{ex}
	If $x \in X_0$ is singular, then the skyscraper sheaf $k(x_0)$ is not in $\Perf(X)$.
\end{ex}

To decategorify our exact sequence to the sheaf-theoretic statement above, we take ``periodic cyclic homology.''

\section{2/13 (Will Fisher) -- The Singular Category and Matrix Factorizations}

Today's talk is based on Orlov's paper ``Triangulated categories of singularities \dots''
We work over a field $k$.

Last time, Daigo presented the exact sequence
\[
	\begin{tikzcd}
		\Perf(X_0) \rar & \Dbcoh(X_0) \rar & \Dsing(X_0).
	\end{tikzcd}
\]
One can think of this as a categorified version of the usual nearby / vanishing cycles exact sequence (which we recover by taking periodic cyclic homology): $\Dbcoh(X_0)$ knows something about a ``nearby smoothing'' of $X_0$.

For nice $f: X \to \AA^1$ with $X_0 = f\inv(0)$, we can identify $\Dsing(X_0) \simeq \MF(X, f)$.
Our goal is to discuss this result.

\subsection{Definitions}

First, we will need to define everything involved.

\begin{dfn}
	Let $A$ be a commutative ring.
	A chain complex $M \in \Dsf(\Mod_A)$ is \emph{perfect} if it is quasi-isomorphic to a bounded complex of finite projective modules.
	These form a subcategory $\Perf(A) \subset \Dsf(\Mod_A)$.
	Equivalently, $\Perf(A)$ is the smallest triangulated subcategory containing $A$ and closed under and retracts.
\end{dfn}

\begin{dfn}
	If $X$ is a scheme, then $\Perf(X)$ is the full subcategory of $\Dqc(X)$ consisting of objects which are perfect affine locally, i.e.\ locally they can be written as a quotient of vector bundles.
\end{dfn}

We will assume $X$ is a separated noetherian scheme of finite Krull dimension, and that $\Coh(X)$ ``has enough vector bundles,'' i.e.\ for all $\Fc \in \Coh(X)$, there exists a vector bundle $\Pc$ and a surjection $\Pc \twoheadrightarrow \Fc$.
These conditions will hold $X$ is quasiprojective.

\begin{dfn}
	The category $\Dbcoh(X) \subset \Dqc(X)$ consists of objects with bounded and coherent cohomology.\footnote{In general, it is better to treat this as a \emph{property} than as \emph{structure}.}
\end{dfn}

Under the above hypotheses, we have $\Dbcoh(X) = \Dsfb(\Coh(X))$.

\begin{dfn}
	The category $\Dsing(X)$ is the Verdier quotient $\Dbcoh(X) / \Perf(X)$, constructed by formally inverting morphisms in $\Dbcoh(X)$ with cones in $\Perf(X)$.
\end{dfn}

\begin{ex}
	Let $X = \Spec A$ for $A = k[x] / (x^2)$.
	The $A$-module $k = A / (x)$ is coherent but not perfect: it has the infinite resolution
	\[
		\begin{tikzcd}
			\dots \rar A \rar["\cdot x"] & A \rar["\cdot x"] & 0.
		\end{tikzcd}
	\]
	We can use this to compute $\Ext^i(k, k) \cong k$ for $i \geq 0$, showing that $k$ is not perfect.
	In particular, $\Dsing(X)$ is nontrivial.
	This relates to Serre's criterion for regularity in algebraic geometry.
\end{ex}

\subsection{Basic properties}

We state without proof some relevant (but hard) facts about $\Dsing(X)$:
\begin{enumerate}
	\item (Auslander-Buchsbaum-Serre) If $A$ is noetherian and finite-dimensional, then $\Dsing(\Spec A) = 0$ if and only if $A$ is regular.
	\item (Thomason-Trobaugh) If $U \subset X$ is an open subscheme containing the singular locus of $X$, then $\Dsing(X) \xrightarrow{\sim} \Dsing(U)$.
\end{enumerate}

It turns out that $\Dsing(X)$ is smaller than we might initially expect:

\begin{prop}
	Every object of $\Dsing(X)$ is equivalent to $\Fc[k+1]$ for some coherent sheaf $\Fc$ and some $k \in \ZZ$.
\end{prop}

\begin{proof}
	For $A^\bullet \in \Dbcoh(X)$, choose a quasi-isomorphism $P^\bullet \xrightarrow{\sim} A^\bullet$ with $P^\bullet$ a bounded above complex of vector bundles.
	The stupid truncations $\sigma^{\geq -k} P^\bullet$ (obtained by replacing $P^\bullet$ by $0$ for $\bullet < -k$) are perfect.
	Let $f_{-k}: \sigma^{\geq -k} P^\bullet \to A^\bullet$ be the natural map, and consider $\cone(f_{-k})$.
	The long exact sequence of cohomology sheaves for the cofiber sequence shows that, for $k \gg 0$, we have $\Hc^i(\cone(f_{-k})) = 0$ unless $i = -k-1$.
	Thus, taking $\Fc = \Hc^{-k}(\sigma^{\geq -k} P^\bullet)$, we obtain $\cone(f_{-k}) \simeq \Fc[k+1]$.
	This gives $A^\bullet \simeq \Fc[k+1]$ in $\Dsing(X)$.
\end{proof}

We'd also like to be able to take right resolutions of coherent sheaves.
If we were capable of ``dualizing'' in a way that preserves vector bundles, we could take a left resolution of $\Fc^\vee$ and dualize this to get a resolution of $\Fc^{\vee \vee} \simeq \Fc$.

If $X$ is Gorenstein and satisfies our standing assumptions, then $\Oc_X$ has a finite injective resolution and
\[
	\Fc \xrightarrow{\sim} \Rbf\scHom(\Rbf\scHom(\Fc, \Oc_X), \Oc_X)
\]
for $\Fc \in \Coh(X)$.

\begin{lem}
	Let $X$ be as above, and let $\Fc \in \Coh(X)$.
	TFAE:
	\begin{enumerate}
		\item $\scExt^i(\Fc, \Oc_X) = 0$ for all $i > 0$.
		\item There exists a right resolution of $\Fc$ by vector bundles.
	\end{enumerate}
\end{lem}

\begin{cor}
	Every $A^\bullet \in \Dsing(X)$ is equivalent to $\Fc[0]$ for some $\Fc \in \Coh(X)$ with $\scExt^i(\Fc, \Oc_X) = 0$ for $i > 0$.
\end{cor}

\section{2/20 (Will Fisher) -- Continued}

\subsection{Dualizing complexes}

Let's review / clarify some points on dualizing complexes.
This concerns something a bit different than the usual Serre duality context: we only care about the local setting and giving equivalences $\Dbcoh(X)\op \xrightarrow{\sim} \Dbcoh(X)$.

\begin{dfn}
	If $A$ is a noetherian ring, a \emph{dualizing complex} on $\Spec A$ is $\omega^\bullet \in \Dbcoh(\Spec A)$ such that:
	\begin{enumerate}
		\item $\omega^\bullet$ has finite injective dimension, and
		\item $A \to \Rbf\Hom(\omega^\bullet, \omega^\bullet)$ is a quasi-isomorphism.
	\end{enumerate}
	If $X$ is a locally noetherian scheme, a \emph{dualizing complex} on $X$ is $\omega^\bullet \in \Dbcoh(X)$ such that $\omega^\bullet$ is affine locally a dualizing complex.
\end{dfn}

If $\omega$ is a dualizing complex on $X$, then $\Rbf\scHom(-, \omega^\bullet): \Dbcoh(X)\op \to \Dbcoh(X)$ is an equivalence (and in fact is its own inverse).

\begin{rmk}
	Peter Haine pointed out a few things:
	\begin{itemize}
		\item Dualizing complexes are unique up to tensoring with complete line bundles.
		\item A result of Kawasaki (proving a conjecture of Sharp) shows that the commutative rings which admit dualizing complexes are those of finite Krull dimension which arise as quotients of Gorenstein rings.
		\item Upper shriek functors preserve dualizing complexes in this sense.
	\end{itemize}
\end{rmk}

\subsection{Gorenstein schemes}

In the Gorenstein case, dualizing complexes are easy to understand.

\begin{thm}
	If $X$ is locally noetherian and has a dualizing complex, TFAE:
	\begin{enumerate}
		\item $X$ is Gorenstein.
		\item $X$ has an invertible dualizing complex.
		\item $\Oc_X[0]$ is a dualizing complex.
	\end{enumerate}
\end{thm}

We have a few large classes of Gorenstein schemes:

\begin{thm}
	Smooth schemes are dualizing.
\end{thm}

\begin{thm}
	Local complete intersections in Gorenstein schemes are dualizing.
\end{thm}


\subsection{Representing objects of $\Dsing(X)$}

From now on we will assume $X$ is Gorenstein.
Our goal is to show that objects of $\Dsing(X)$ can be represented by particularly nice sheaves and complexes.

\begin{prop}
	If $\Oc_X$ has a finite injective resolution, then there exists $n_0 > 0$ such that, for all $\Fc \in \QCoh(X)$ and all $i > n_0$, we have $\scExt^i(\Fc, \Oc_X) = 0$.
\end{prop}

\begin{prop}
	For $\Fc \in \Coh(X)$, TFAE:
	\begin{enumerate}
		\item $\scExt^i(\Fc, \Oc_X) = 0$ for all $i > 0$.
		\item $\Fc$ has a right resolution by vector bundles.
	\end{enumerate}
\end{prop}

\begin{proof}
	We only prove $\Rightarrow$.
	To obtain the desired right resolution, take a left resolution $\Pc^\bullet \xrightarrow{\sim} \scHom(\Fc, \Oc_X) = \Rbf \scHom(\Fc, \Oc_X)$.
	Then $\Fc \simeq \scHom(\scHom(\Fc, \Oc_X), \Oc_X) \simeq \scHom(\Pc^\bullet, \Oc_X)$ gives the desired right resolution.
\end{proof}

\begin{thm}
	Every object in $\Dsing(X)$ is equivalent to $\Fc[0]$ for $\Fc$ coherent with $\scExt^i(\Fc, \Oc_X) = 0$ for all $i > 0$.
\end{thm}

\begin{proof}
	Let $\Ac^\bullet \in \Dbcoh(X)$.
	Take a left resolution $\Pc^\bullet \xrightarrow{\sim} \Ac^\bullet$.
	For $k \gg 0$, we saw last time that if $\Gc = \Hc^{-k}(\sigma^{\geq -k} \Pc^\bullet)$, then $\Ac^\bullet \simeq \Gc[k+1]$ in $\Dsing(X)$.
	Since $\Oc_X$ has bounded injective resolution, $\Rbf \scHom(\Ac^\bullet, \Oc_X)$ is cohomologically bounded.
	Thus, for $k \gg 0$, we get
	\[
		\scExt^i(\Gc, \Oc_X) \cong \scExt^{i+k+1}(\Ac^\bullet, \Oc_X) = 0
	\]
	for $i > 0$.
	Now take a right resolution $\Gc \to \Qc^\bullet$,\footnote{This is where the Gorenstein hypothesis appears.} and let $\Fc = \Hc^k(\sigma^{\leq k} \Qc^\bullet)$.
	Then it is clear that $\Fc$ has the desired properties.
\end{proof}

\subsection{Equivalence with matrix factorizations}

Let $X = \Spec A$ be smooth, let $f: X \to \AA^1$, and let $X_0 = f\inv(0) = \Spec A / (f)$.
Write $j: X_0 \to X$ for the inclusion.
We would like to show that $\Dsing(X_0) \simeq \MF(X, f)$, the matrix factorization category of $f$.

\begin{prop}
	Let $\Fc$ be coherent on $X_0$ with $\scExt^i(\Fc, \Oc_{X_0}) = 0$ for all $i > 0$.
	Then there exist vector bundles $\Pc_0, \Pc_1$ on $X$ with maps $p_0: \Pc_0 \to \Pc_1$ and $p_1: \Pc_1 \to \Pc_0$ such that:
	\begin{enumerate}
		\item $p_0 p_1 = f \cdot \id_X$,
		\item $p_1 p_0 = f \cdot \id_X$, and
		\item $\coker p_1 = j_* \Fc$.
	\end{enumerate}
\end{prop}

\begin{proof}[Sketch of proof]
	Choose a surjection $\Pc_0 \to j_* \Fc$ with $\Pc_0$ a vector bundle.
	Let $p_1: \Pc_1 \to \Pc_0$ be the kernel of this surjection.
	Then the hypotheses on $\Fc$ imply that $\Pc_1$ is a vector bundle.\footnote{More details can be found on Will Fisher's website.}
	Because multiplication by $f$ annihilates $j_* \Fc$, the composite
	\[
		\Pc_0 \xrightarrow{\cdot f} \Pc_0 \to j_* \Fc
	\]
	is zero, so it factors through $p_1$.
	That is, there exists $p_0: \Pc_0 \to \Pc_1$ such that $p_1 p_0 = f \cdot \id$.
	The computation $p_1 p_0 p_1 = (- \cdot f) \circ p_1 = p_1 \circ (- \cdot f)$ implies $p_0 p_1 = \id$ as well.
\end{proof}

Let $\Pc_0$ and $\Pc_1$ be as above.
Then we get an exact sequence\footnote{This is left as a non-obvious exercise.}
\[
	\begin{tikzcd}
		0 \rar & \Fc \rar & \Pc_1|_{X_0} \rar & \Pc_0|_{X_0} \rar & \Fc \rar & 0.
	\end{tikzcd}
\]
In particular, we obtain $\Fc \simeq \Fc[2]$ in $\Dsing(X)$.

\begin{dfn}
	The \emph{matrix factorization category} of $(X, f)$ has objects given by pairs of vector bundles $\Pc_0, \Pc_1$ on $X$ with maps $p_0: \Pc_0 \to \Pc_1$ and $p_1: \Pc_1 \to \Pc_0$ such that:
	\begin{enumerate}
		\item $p_0 p_1 = f \cdot \id_X$ and 
		\item $p_1 p_0 = f \cdot \id_X$.
	\end{enumerate}
	Morphisms are defined to be homotopy classes of commutative squares.
\end{dfn}

There is a functor $\coker: \MF(X, f) \to \Dsing(X_0)$ sending $(\Pc_0, \Pc_1, p_0, p_1)$ to $\coker(p_1)$.

\begin{thm}[Orlov]
	The functor $\coker: \MF(X, f) \to \Dsing(X_0)$ is an equivalence.
\end{thm}

\begin{ex}
	Let $A = \CC[x]$, $f = x^n$, and $X_0 = \Spec \CC[x] / (x^n)$.
	Then $\Dsing(X_0)$ has objects given by direct sums of $V_i = \CC[x] / (x^i)$ for $1 \leq i \leq n-1$.
	Under the equivalence $\Dsing(X_0) \simeq \MF(X, f)$, $V_i$ corresponds to $(\CC[x], \CC[x], x^{n-i}, x^i)$.
	One can decategorify this to obtain the usual nearby / vanishing cycle picture for $f$.
\end{ex}

\section{2/27 (Elliot Kienzle) -- Matrix Factorizations}

\subsection{David -- Opening remarks}

Next week we will have a guest speaker.
We will also eventually hear from Enoch Yiu, discussing his thesis work on $\Hom$s in the Rozansky-Witten 2-category involving conormals (in $T^*X$) to subvarieties of $X$.

\subsection{Elliot -- Quick review}

Suppose we are given a function / superpotential $W: X \to \CC$.
We get a singular fiber $X_0$ and nearby fibers $X_s$.
These fit into a long exact sequence
\[
	\begin{tikzcd}
		\dots \rar & H^\bullet(X_0) \rar & H^\bullet(X_s) \rar & H^\bullet(X, X_s) \rar & \dots
	\end{tikzcd}
\]
which categorifies to
\[
	\begin{tikzcd}
		\Perf(X_0) \rar & \Dbcoh(X_0) \rar & \Dsing(X_0) \rar[equal] & \MF(X, W).
	\end{tikzcd}
\]

In the above, $\MF(X, W)$ is a category whose objects are ``matrix factorizations'' $P^\bullet = (P_0, P_1, p_0, p_1)$ where $P_0$ and $P_1$ are vector bundles on $X$, $p_0: P_0 \to P_1$, $p_1: P_1 \to P_0$, and $p_0 p_1 = W \cdot \id$ and $p_1 p_0 = W \cdot \id$.
This allows us to factorize polynomials which we could not factorize in the original polynomial rings.

\begin{ex}
	Working on $\RR^2$, we cannot factor $W = x^2 + y^2$ in $\RR[x, y]$.
	However, we do have
	\[
		\begin{bmatrix}
			x & y \\
			-y & x
		\end{bmatrix}
		\begin{bmatrix}
			x & -y \\
			y & x
		\end{bmatrix}
		=
		\begin{bmatrix}
			x^2 + y^2 & 0 \\
			0 & x^2 + y^2
		\end{bmatrix}.
	\]
\end{ex}

One may view $P^\bullet$ as a $\ZZ/2$-graded vector space with an endomorphism of odd degree given by
\[
	\begin{bmatrix}
		0 & p_0 \\
		p_1 & 0
	\end{bmatrix}.
\]
In particular, we get a $\ZZ/2$-graded morphism space $\Hom^\bullet(P, Q)$ (defined without reference to the endomorphisms).
There is a natural differential $\partial$ on $P^\bullet$ given by $\partial \phi = q \phi - (-1)^{|\phi|} \phi p$.
We define $\Mor^\bullet(P, Q) = H^\bullet(\Hom^\bullet(P, Q), \partial)$.
In particular, $\Mor^0(P, Q)$ consists of ``chain maps modulo chain homotopy.''

\subsection{Boring matrix factorizations}

Say that $\MF(X, W)$ is \emph{boring} if every object is equivalent to $0 = (0 \rightleftarrows 0)$.

\begin{lem}
	The following are equivalent for a matrix factorization $P = (P_0 \rightleftarrows P_1)$:
	\begin{enumerate}
		\item $P$ is equivalent to $0$.
		\item $\id_P \sim 0 \in \Mor(P, P)$.
		\item There exists odd $\phi$ such that $\phi_1 p_0 + p_1 \phi_0 = \id_{P_0}$ and $p_1 \phi_1 + \phi_0 p_1 = \id_{P_1}$.
	\end{enumerate}
\end{lem}

\begin{ex}
	The matrix factorization category $\MF(X, 1)$ is boring: given $P$, we can define a chain homotopy $\phi$ from $\id_P$ to $0$ via $\phi = (0, p_0)$.
	More generally, if $W$ is nonvanishing, then $\MF(X, W)$ is boring.
\end{ex}

\begin{ex}
	The matrix factorization category $\MF(\CC, z)$ is boring.
	For a special case of this claim, consider $P$ with $P_0 = P_1 = \Oc$.
	At most one of $p_0$ and $p_1$ can vanish, so assume $p_1$ is nonvanishing.
	Then we can take $\phi_0 = p_0 / z$ and $\phi_1 = 0$.
	The general case is encompassed by the following.
\end{ex}
		
\begin{ex}
	The matrix factorization category $\MF(\CC^n, z_i)$ is boring.
	To see this, note that $p_0 p_1 = z_i$ implies $(\partial_i p_0) p_1 + p_0 (\partial_i p_1) = \id$.
	It's plausible that this generalizes to show that $\MF(\CC^n, W)$ is boring whenever $dW$ is nonvanishing.
\end{ex}

The upshot is that, if $W$ or $dW$ is nonvanishing, then $\MF(X, W)$ is trivial.

\subsection{The first interesting matrix factorization category}

Let $X = V = \CC^n$ and $W = Q(z, z) = z_1^2 + \dots + z_n^2$.
To do this, let us recall some of the history of quantum field theory.

Dirac wanted to find a square root of the Laplacian $\Delta = \partial_1^2 + \dots + \partial_n^2$.
That is, he wanted $\Dslash$ such that $\Dslash^2 = \Delta$.
This $\Dslash$ does not exist as a scalar function: it only exists as a \emph{matrix} of linear differential operators
If we replace $\partial_i^2$ by $x_i$, we end up with a corresponding matrix factorization of our $W$.

Our goal is to prove the following:

\begin{thm}
	Let $\Cl(V, Q)$ be the Clifford algebra of $(V, Q)$.
	Then $\MF(\CC^n, Q) \simeq \Cl(V, Q)\dMod$.
\end{thm}

David pointed out that \emph{Kn\"orrer periodicity} tells us that
\[
	\Cl(\CC^n, Q)\dMod \simeq \begin{cases}
		\MF(\CC, x^2) & n \textrm{ odd} \\
		\MF(\CC^2, x^2 + y^2) & n \textrm{ even}.
	\end{cases}
\]
The category $\MF(\CC, x^2)$ can be identified with $\ZZ/2$-graded $\CC[\epsilon] / (\epsilon^2 - 1)$-modules, where $|\epsilon| = 1$.
The category $\MF(\CC^2, x^2 + y^2)$ can be identified with $\ZZ/2$-graded vector spaces.
One can interpret this in terms of vanishing cycles: for $(\CC, x^2)$ we have nontrivial monodromy $(-1)$, whereas for $(\CC^2, x^2 + y^2)$ we have trivial monodromy.

\section{3/6 (Lucas Mason-Brown) -- Some Progress on the Problem of the Unitary Dual}

This is based on joint wirk with Losev, Matvieievskyi, and Davis.

\subsection{Review of unitary representation theory}

Let $G_\RR$ be a real reductive group (i.e.\ the real points of a reductive algebraic group over $\RR$, or a Lie group which is isogenous to one).
A \emph{unitary representation} of $G_\RR$ is a complex Hilbert space $V$ (typically infinite-dimensional) with a continuous representation $G_\RR \times V \to V$ by unitary operators.\footnote{In general, asking for $G_\RR \to \U(V)$ to be continuous is not the right continuity condition.}
Such a representation is \emph{irreducible} if it has no nontrivial \emph{closed} invariant subspaces.

\begin{ex}
	The trivial representation of any $G_\RR$ is unitary.
\end{ex}

\begin{ex}
	If $G_\RR$ is compact, then every continuous finite-dimensional representation $V$ of $G_\RR$ is unitary (with respect to an inner product defined via averaging).
\end{ex}

\begin{ex}
	Let $X$ be a smooth manifold with an action of $G_\RR$ by diffeomorphisms.
	On $X$, there is a real line bundle of \emph{half-densities} $\Lc^{1/2} = \Fr(TX) \times^{\GL_n(\RR)} \RR_{\abs{\det}^{1/2}}$.
	Local sections of $\Lc^{1/2}$ look like $f(x_1, \dots, x_n) \abs{dx_1 \wedge \dots \wedge dx_n}^{1/2}$.
	There is a real inner product on $\Gamma_c(X, \Lc^{1/2})$ given by $\ip{s_1}{s_2} = \int_X s_1 \otimes s_2$.
	Taking the completion of $\Gamma_c(X, \Lc^{1/2}) \otimes_\RR \CC$, we obtain a unitary $G_\RR$-representation $L^2(X)$.

	If we let $P_\RR$ be a the real points of a parabolic, and $X = G_\RR / P_\RR$, then $L^2(X)$ is the (unnormalized) ``unitary induction'' of the trivial $P_\RR$-representation.
	We can twist this to get $L^2(X, V) = \Ind_{P_\RR}^{G_\RR} V$.
\end{ex}

\begin{ex}
	Let $G_\RR$ be the metaplectic group $\operatorname{Mp}(2n, \RR)$, a non-linear double cover of the real symplectic group.
	This has an ``oscillator representation'' on $L^2(\RR^n)$ which does not come from an action of $G_\RR$ on $\RR^n$.
	The oscillator representation is constructed using the Stone-von Neumann theorem.
\end{ex}

\subsection{The problem of the unitary dual}

An important open problem in unitary representation theorem is to classify the space $\Pi_u(G_\RR)$ of irreducible unitary $G_\RR$-representations (up to unitary equivalence).
Harish-Chandra proposed the following approach.
Fix a maximal compact subgroup $K_\RR \subset G_\RR$, and let $K$ be the complexification of $K_\RR$.

\begin{ex}
	If $G_\RR = \SL_n(\RR)$, we can take $K_\RR = \SO_n(\RR)$ and $K = \SO_n(\CC)$.
\end{ex}

For a unitary $G_\RR$-representation, we can define the \emph{Harish-Chandra module}
\[
	\operatorname{HC}(V) = \bset{v \in V}{\dim \spanop K_\RR v < \infty}.
\]
Harish-Chandra showed that $\operatorname{HC}(V)$ is a $(\gfr, K)$-module with a $(\gfr_\RR, K_\RR)$-invariant positive definite inner product.

\begin{thm}[Harish-Chandra]
	The assignment $V \mapsto \operatorname{HC}(V)$ gives a bijection between $\Pi_u(G_\RR)$ and $\Pi_u(\gfr, K)$ (the set of irreducible unitarizable $(\gfr, K)$-modules).
\end{thm}

Note that $\Pi_u(\gfr, K)$ is quite algebraic, while $\Pi_u(G_\RR)$ is quite analytic!
We can embed $\Pi_u(\gfr, K)$ into $\Pi(\gfr, K)$, the space of all irreducible $(\gfr, K)$-modules.
This latter space is well-understood via Langlands parameters / Beilinson-Bernstein localization / etc.
Thus the question turns into understanding the embedding $\Pi_u(\gfr, K) \hookrightarrow \Pi(g, K)$.

This splits into two subquestions.
The first subquestion, understanding which irreducible $(\gfr, K)$-modules admit a $(\gfr, K)$-invariant bilinear form, was solved in the 1970s.
The remaining subquestion, understanding when this bilinear form is Hermitian, is open.

\subsection{Conjectural solution}

\begin{conj}
	For every twisted Levi subgroup $L_\RR \subset G_\RR$, there are:
	\begin{enumerate}
		\item Standard inductive procedures $I$ producing finitely many elements of $\Pi_u(\gfr, K)$ from a given element of $\Pi_u(L_\RR)$.
			(These include parabolic induction, cohomological induction, and complementary series.)
		\item A finite set of explicit ``rigid'' cuspidal representations $\Pi_{\mathrm{rig}}(L_\RR)$.
	\end{enumerate}
	We may write $\Pi_u(G_\RR) = \cup_{L_\RR} I(\Pi_u(L_\RR))$.
\end{conj}

To make this conjecture worthwhile, we should explain what we mean by rigid representations.
Let $\Oc \subset \gfr^*$ be a nilpotent coadjoint orbit in the dual of the complexified Lie algebra.
Then $\CC[\Oc]$ is a finitely generated graded Poisson algebra.
A quantization of $\CC[\Oc]$ is a nonnegatively filtered algebra $\Ac = \cup_m \Ac_{\leq m}$ such that $[\Ac_{\leq m}, \Ac_{\leq n}] \subset \Ac_{\leq m + n - 1}$ together with an isomorphism of graded Poisson algebras $\tau: \gr(\Ac) \xrightarrow{\sim} \CC[\Oc]$.
This gives a unique quantum comoment map, i.e.\ a filtered algebra homomorphism $\Uc(\gfr) \to \Ac$ reducing to $\Sym \gfr \to \CC[\Oc]$ after passing to associated graded algebras.

\begin{thm}[Namikawa, Losev]
	The set of quantizations of $\CC[\Oc]$ is in natural bijection with points of the Namikawa space $\hfr_\Oc$, a finite-dimensional complex vector space constructed from the geometry of $\Oc$.\footnote{Specifically, $\hfr_\Oc$ is the second cohomology of a $\QQ$-factorial terminal resolution of $\Oc$ (e.g.\ a symplectic resolution of $\Oc$, if one exists).}
\end{thm}

\begin{dfn}
	The \emph{canonical quantization} $\Ac_0$ of $\CC[\Oc]$ is the quantization of $\CC[\Oc]$ corresponding to $0 \in \hfr_\Oc$.
	The \emph{unipotent ideal} for $\Oc$ is $I_0(\Oc) = \ker(\Uc(\gfr) \to \Ac_0)$.
\end{dfn}

\begin{prop}
	The following are equivalent:
	\begin{enumerate}
		\item $\Spec \CC[\Oc]$ is smooth in codimension 2 and $H^2(\Oc, \CC) = 0$.
		\item $\hfr_\Oc = 0$.
		\item $\Oc$ is not induced from a proper Levi subgroup.
		\item $I_0(\Oc)$ is not induced.
	\end{enumerate}
\end{prop}

In this case, we say $\Oc$ is \emph{rigid}.

\begin{dfn}
	A \emph{rigid $(\gfr, K)$-module} is an irreducible $(\gfr, K)$-module $M$ such that $\Ann_{\Uc(\gfr)}(M) = I(\Oc)$ for $\Oc$ rigid.
\end{dfn}

\begin{ex}
	If $\Oc = 0$, the corresponding rigid representations are exactly the one-dimensional representations of $G_\RR$ that factor through the component group of $G_\RR$.
\end{ex}

\begin{ex}
	The oscillator representation of $\operatorname{Mp}_{2n}(\RR)$ is rigid.
\end{ex}

\begin{ex}
	There is an interesting rigid representation related to rigid orbits of for the real split $E_8$-group.
\end{ex}

The speaker and collaborators showed that rigid representations are unitarizable using Hodge theory.

\section{3/13 (Elliot Kienzle) -- Continued}

Write $\MF(V, Q) = \MF(\CC^n, z_1^2 + \dots + z_n^2)$.
Our goal is to discuss the connection between this category and Clifford algebras.

\subsection{Clifford algebras}

Let $V$ be an $n$-dimensional complex vector space with nondegenerate quadratic form $Q$.
Write $T(V) = k \oplus V \oplus V^{\otimes 2} \oplus \dots$ for the tensor algebra of $V$.

\begin{dfn}
	The $n$th \emph{Clifford algebra} is
	\[
		\Cl_n = \Cl(V, Q) = \frac{T(V)}{v \otimes v - Q(v, v) \otimes 1}
	\]
\end{dfn}

As a vector space, $\Cl_n \simeq \Lambda^\bullet V$, the exterior algebra of $V$.
However, the product is ``deformed'' by
\[
	(v, w) \mapsto v \wedge w - Q(v, w).
\]
We still have a $\ZZ/2$-grading of $\Cl_n$ corresponding to the ``degree of monomials'' (which is well-defined modulo $2$).

Note that $\Cl_0 \cong \CC$ and $\Cl_1 \cong \CC \times \CC$.

\begin{thm}[Clifford algebra periodicity]
	We have $\Cl_{n+2} \cong \Mat_2(\Cl_n)$, the ring of $2 \times 2$ matrices with entries in $\Cl_n$.
\end{thm}

This implies $\Cl_n$ is semisimple, and $\Cl_{n+2}\dMod \simeq \Cl_n\dMod$ (where these refer to \emph{graded} modules).
Taking $M_n$ to be the representation ring of $\Cl_n$, we get
\[
	M_n = \begin{cases}
		\ZZ & n \textrm{ even} \\
		\ZZ^2 & \textrm{ odd}
	\end{cases}
\]
One can view this as an algebraic version of complex Bott periodicity.

\subsection{Connection with matrix factorizations}

Every Clifford module $M$ gives rise to a matrix factorization in $\MF(V, Q)$.
More precisely, write $M \cong M^0 \oplus M^1$, and view each $M^i$ as (the fiber of) a trivial vector bundle on $V$.
We can define maps via these vector bundles via $(v, m) \mapsto v \cdot m$ for $v \in V$, $m \in M_i$.
This gives a matrix factorization $M^0 \rightleftarrows M^1$ of $Q$.

\begin{thm}
	This induces an equivalence $\Cl_n\dMod \simeq \MF(V, Q)$.
\end{thm}

\subsection{Matrix factorizations and topology}

We'd like to extract topological information about vanishing cycles from matrix factorizations.
For a nice topological space $X$, let $K^0(X)$ be the Grothendieck group of vector bundles on $X$, i.e.\ the group generated by vector bundles on $X$ with relations given by $[U] + [V] = [W]$ whenever $W$ is an extension of $U$ by $V$.
For a nice subspace $Y \subset X$, we define \emph{relative $K$-theory} via 
\[
	K^0(X, Y) = \tilde{K}^0(X / Y) = \ker(K^0(X / Y) \to K^0(\pt)).
\]
We may extend $K$-theory to a cohomology theory via $\tilde{K}^{-n}(X) = \tilde{K}^0(\Sigma^n X)$.
The \emph{Chern character} gives homomorphisms $\ch: K^0(X) \to H^0(X)$ and $K^0(X, Y) \to H^0(X, Y)$.

Given $\sigma: V_1 \to V_2$ over $X$ such that $\sigma$ is an isomorphism over $Y \subset X$, then we can construct a class $\chi(\sigma) \in K^0(X, Y)$.
To accomplish this, we construct a vector bundle $V$ on $M = (Y \times [0, 1]) \sqcup_{Y \times \{0, 1\}} (X \times \{0, 1\})$\,\footnote{Elliot drew a picture which I will not attempt to replicate here.} by gluing $V_1$ and $V_2$ together using $\sigma$.
We can also construct a vector bundle $V_2'$ which is given by $V_2$ everywhere.
The difference $[V] - [V_2']$ gives a class in $K^0(M)$, and this class in fact lives in $K^0(M, X) = K^0(X, Y)$ (where we mod out by the factor on which$V$ and $V_2'$ agree).
This is our $\chi(\sigma)$.

Applying this to $\MF(V, Q)$, we see that, for every Clifford module $M$, we obtain an \emph{Atiyah-Bott-Shapiro} class $\operatorname{ABS}(M) \in K^0(B, Q\inv(1) \cap B)$ where $B$ is a ball.
Note that $K^0(B, Q\inv(1) \cap B) \cong \tilde{K}^0(S^n)$.
Thus we end up with a map $\operatorname{ABS}: \Mc_n \to \tilde{K}^0(S^n)$.

\begin{thm}[Atiyah-Bott-Shapiro]
	Let $i: \Cl_n \hookrightarrow \Cl_{n+1}$ be the inclusion.
	Then $\ker(\mathop{ABS}) = i^* \Mc_{n+1}$.
	It follows that $\tilde{K}^0(S^n) \cong \Mc_n / i^* \Mc_{n+1}$.
\end{thm}

Note that the Atiyah-Bott-Shapiro construction, as presented here, secretly uses matrix factorizations!

We can also interpret Kn\"orrer periodicity of matrix factorization categories as a categorification of Bott periodicity:

\begin{thm}[Kn\"orrer periodicity]
	There is an equivalence
	\[
		\MF(X, W) \simeq \MF(X \times \CC^2, W + x^2 + y^2).
	\]
	Taking $K$-theory and using the Thom-Sebastiani isomorphism gives Bott periodicity.
\end{thm}

\subsection{From matrix factorizations to Clifford modules}

Let's see how to obtain a Clifford module from a matrix factorization.
Recall that we have $\MF(V, Q) \simeq \Dsing(Q\inv(0))$.
The skyscraper sheaf at the origin of $\CC$ corresponds to $\CC \rightleftarrows \CC$, where both maps are multiplication by $z$.
Taking the $n$th external tensor power of this gives an object $\delta \in \MF(V^n, Q)$, with $\End^*(\delta) = (P \rightleftarrows Q)$.
Our map $\MF(V, Q) \to \Cl_n\dMod$ is just $\Hom(\delta, -)$.

\section{3/20 (Yuji Okitani) -- The Rozansky-Witten 2-Category}

Following the paper by Kapustin and Rozansky, we will attempt to explain the idea behind the Rozansky-Witten 2-category.

\subsection{The construction}

Consider Rozansky-Witten theory with target $T^* X$ for $X$ a smooth algebraic variety.
We construct a 2-category $\RW_2(T^*X)$\footnote{This is only part of the ``true'' Rozansky-Witten 2-category} with:
\begin{enumerate}
	\item Objects: potentials $W: X \to \AA^1$, or rather their graphs $\Gamma_{dW} \subset T^* X$.
	\item Morphism categories: $[W_1, W_2] = \MF(X, W_2 - W_1)$ (corresponding to Lagrangian intersections).
		Note that:
		\begin{enumerate}
			\item $\MF(X, W)$ localizes to $\Crit(W) = \{ dW = 0 \}$.
			\item $\MF(X, W) = \prod_{s \in \AA^1} \Dbcoh(X_s) / \Perf(X_s)$.
		\end{enumerate}
	\item Composition: $\MF(X, W_2 - W_1) \otimes \MF(X, W_3 - W_2) \to \MF(X, W_3 - W_1)$ by tensor product of periodic complexes:
		\[
			(P^0, P^1, d_P) \otimes (Q^0, Q^1, d_Q) = \big((P^0 \otimes Q^0) \oplus (P^1 \otimes Q^1), (P^1 \otimes Q^0) \oplus (P^0 \otimes Q^1), d_{P \otimes Q}\big)
		\]
		where the differential $d_{P \otimes Q}$ is constructed using the usual sign rule.
		It is a good exercise to work this all out.
\end{enumerate}

\subsection{Some examples}

\begin{ex}
	Let's consider a triangle shape in $T^* \AA^1$ given by graphs $\Gamma_{dW_i}$ where $W_1 = 0$, $W_2 = x^2$, and $W_3 = -(x - 2)^2 + 2$.
	Then:
	\begin{itemize}
		\item $[W_1, W_2] = \MF(\AA^1, x^2)$, generated by $(\CC[x], \CC[x])$ with morphisms given by multiplication by $x$.
		\item $[W_2, W_3] = \MF(\AA^2, -2(x - 1)^2)$, generated by $(\CC[x], \CC[x])$ with morphisms given by multiplication by $x - 2$ and $-2(x - 2)$.
		\item $[W_1, W_3] = 0$.
	\end{itemize}
	We don't end up with a problem here due to ``locality in $X$.''
	Basically: for the $W_i$ to have interesting composition, we need all differences $W_i - W_j$ to share a common critical value.
	Note that this requires choosing consistent representatives $W_i$ of the graphs $\Gamma_{dW_i}$.
\end{ex}

\begin{ex}
	Let $X$ be a vector space $V$ equipped with a quadratic form $Q: V \to \CC$.
	Elliot gave an equivalence $\MF(V, Q) \simeq \Cl_n\dMod$.
	One can view this as a deformed version of Koszul duality (for $Q = 0$, this relates $\Sym V^*$ to $\Lambda V$).
	
	We expect that $\otimes: \MF(V, Q_1) \otimes \MF(V, Q_2) \to \MF(V, Q_1 + Q_2)$ should correspond to a convolution product $\star: \Cl(V, Q_1)\dMod \otimes \Cl(V, Q_2)\dMod \to \Cl(V, Q_1 + Q_2)\dMod$.
	The convolution product here is induced by
	\begin{align*}
		\Cl(V, Q_1 + Q_2) &\to \Cl(V, Q_1) \otimes \Cl(V, Q_2) \\
		v &\mapsto 1 \otimes v + v \otimes 1.
	\end{align*}
	We can work this out explicitly in examples.
\end{ex}

Note that $[W, W] = \MF(X, 0)$ is a 2-periodic version of $\Dbcoh(X)$.
For an arbitrary Lagrangian $L$, we can guess that $[L, L]$ should look like a 2-periodic version of $\Coh(L)$, with composition given by tensor product.
In general, we expect that the category of ``2-quasicoherent sheaves'' on $X$ embeds as the ``conic part'' of $\RW_2(T^* X)$.
Next time, Enoch will discuss how to allow conic and non-conic parts of $\RW_2(T^* X)$ to interact.

\section{4/3 (Enoch Yiu) -- Sheaves of Categories with Non-Conic Singular Support}

Let $X$ be a smooth algebraic variety.
We want to understand Rozansky-Witten theory on $T^* X$ in terms of module categories over $\QCoh(X)$, $\Coh(X)$, or $\Perf(X)$.

\subsection{Singular support of sheaves of categories}

To define a notion of singular support, we should first recall some relevant notions.

\begin{enumerate}
	\item The \emph{loop space} of $X$ is the derived fiber product $\Lc X = \Hom(S^1, X) = X \times_{X \times X} X$.
		If $X$ is a smooth algebraic variety, then the Hochschild-Kostant-Rosenberg isomorphism gives an equivalence
		\[
			\Lc X \simeq TX[-1] = \Spec_{\Oc_X} \Omega_X^{-\bullet}.
		\]
		Note that $\Lc X$ is an algebra over $X$.
	\item Note that
		\[
			\Hom_{\Perf(S)}(\Perf(X), \Coh(Y)) \simeq \Coh(X \times_S Y).
		\]
		Taking $X = Y$ and $S = X \times X$, we see that 
		\[
			\End_{\Perf(X) \otimes \Perf(X)\op}(\Perf(X)) \simeq \Coh(\Lc X).
		\]
	\item We can compute
		\[
			\Ext_{\Sym_{\Oc_X} \Omega_X[1]}^*(\Oc_X, \Oc_X) = \Sym_{\Oc_X} \Omega_X^\vee[-2] = \Oc_{T^* X[2]}.
		\]
		This gives a Koszul duality equivalence between $\Coh(\Lc X)$ and a deformation of $\Perf(T^* X[2])$.
	\item If $R$ is an algebra and $M$ is a module, then we get a homomorphism $Z(R) \to \End(M)$.
		This can be understood in terms of a ``whistle'' cobordism in TQFT.
\end{enumerate}

\begin{dfn}
	Let $M \in \Perf(X)\dMod$.
	Then $\End(M)$ is a $\Coh(\Lc X)$-module.
	This induces a map 
	\[
		\End_{\Coh(\Lc X)}(\Oc_X) \to \End_{\End(M)}(\onebb_{\End(M)}).
	\]
	We may think of $\Ext^*(\onebb_{\End(M)}, \onebb_{\End(M)})$ as a topological (i.e.\ non-quasicoherent) sheaf of categories on $T^* X[2]$.
	The singular support $\SS(M)$ is the support of $\Ext^*(\onebb_{\End(M)}, \onebb_{\End(M)})$ over $T^* X[2]$.
	This is a ``conic Lagrangian'' in $T^* X[2]$.
\end{dfn}

Here $\Oc_{T^* X[2]}$ is really an $\EE_3$-algebra.
We are passing to $\Ext^*$'s so that we can work with commutative algebras rather than $\EE_3$-algebras.

\begin{ex}
	Let $i: Z \hookrightarrow X$ be a closed embedding of smooth varieties.
	Then $\Perf(Z)$ is a $\Perf(X)$-module, and
	\[
		\End_{\Perf(X)}(\Perf(Z)) = \Coh(Z \times_X Z) = \Coh(N_Z X[-1]).
	\]
	We may view the shifted normal bundle $N_Z X[-1]$ as $\Lc_Z X / \Lc Z$, where $\Lc_Z X = Z \times_{X \times X} Z$ (?).
	We can compute $\Ext^*(\onebb_{\End(\Perf(Z)}) = \Oc_{T_Z^* X[2]}$.
\end{ex}

\subsection{The conic part of $\RW_2$}

The proposal is essentially that the conic part of $\RW_2(T^* X)$ is $\Perf(X)\dMod$.

Let $M, N \in \Perf(X)\dMod$.
Then $\Hom(M, N)$ is a $\Coh(\Lc X)$-module, and it defines a sheaf on $T^* X[2]$.
The support (over $T^* X[2]$) of $\Hom(M, N)$ is contained in $\SS(M) \cap \SS(N)$.
This agrees with expectations about $\RW_2$!

\begin{ex}
	Let $M = \Perf(X)$ and $N = \Perf(x)$ for some $x \in X$.
	Then $\Hom_{\Perf(X)}(\Perf(X), \Perf(x)) = \Coh(X) \simeq \Vect$.
	This confirms our intuition: the intersection of the singular supports here is just a point, so the $\Hom$ is just $\Vect$.
\end{ex}

Given $f: X \to Y$, we obtain $f^*: \Perf(Y) \to \Perf(X)$ and hence a restriction map $\Perf(X)\dMod \to \Perf(Y)\dMod$.
There is also an induction map $\Perf(Y)\dMod \to \Perf(X)\dMod$.
We can use these to define an external tensor product
\[
	M \boxtimes N = \Indop_{p_1} M \otimes \Indop_{p_2} N.
\]
\end{document}
