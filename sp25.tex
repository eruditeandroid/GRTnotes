\documentclass{article}

\usepackage{notes}

\title{GRT Seminar Fall 2024 -- Rozansky-Witten Theory}
\author{Notes by John S.\ Nolan, speakers listed below}

\begin{document}

\maketitle

\begin{abstract}
	This semester, the GRT Seminar will focus on Rozansky-Witten theory.
\end{abstract}

\tableofcontents

\section{9/5 (David Nadler) -- Introduction}

Our goal is to discuss Rozansky-Witten theory.
Some related topics include:
\begin{itemize}
	\item Quasicoherent sheaves of categories (as discussed last spring).
	\item Categories of matrix factorizations.\footnote{In more detail: given a smooth variety $X$ and a function $f: X \to \AA^1$, we can construct a category $\MF_f$ which categorifies the vanishing cycles of $f$.}
	\item The cobordism hypothesis.
	\item Local structure theory of holomorphic symplectic varieties.
\end{itemize}

\subsection{What is Rozansky-Witten theory?}

Suppose we have a hyperk\"ahler / holomorphic symplectic manifold $X$.
This means that $X$ has a holomorphic $(2, 0)$-form $\omega$ satisfying the (complex analogues of) the usual symplectic form axioms.
Given such an $X$, there is a conjectural 3-dimensional topological field theory $\Zc_X$, called \emph{Rozansky-Witten theory} with target $X$.

What we mean by 3d TFT is as follows:
\begin{itemize}
	\item Given a closed 3-manifold\footnote{Typically with some extra structure, e.g.\ an orientation} $M^3$, we obtain a number $\Zc_X(M^3)$.
	\item Closed 2-manifolds $M^2$ give vector spaces $\Zc_X(M^2)$.
	\item Closed 1-manifolds $M^1$ give categories\footnote{As is standard for GRT, we use the implicit $\infty$ convention.} $\Zc_X(M^1)$.
	\item Closed 0-manifolds $M^0$ give 2-categories $\Zc_X(M^0)$.
\end{itemize}
In particular, $\Zc_X(\pt)$ is a 2-category.
The \emph{cobordism hypothesis} tells us that we can recover the entire theory $\Zc_X$ from the ``3-dualizable'' 2-category $\Zc_X(\pt)$.
For purposes of geometric representation theory, we are most interested in the low-dimensional behavior, which captures more data about the theory.

Rozansky-Witten theory should satisfy something like:
\begin{itemize}
	\item $Z_X(S^2) = \Oc(X)$.\footnote{By our conventions, this is what is classically called $\Rbf \Gamma(X, \Oc)$, so there is interesting derived information.}
	\item $Z_X(S^1) = \Coh(X)$.
\end{itemize}
These end up inheriting interesting structure from the TFT.

\subsection{Why do we care?}

Recall that 2-dimensional mirror symmetry can be schematically understood as an equivalence between the following 2d TFTs:
\begin{itemize}
	\item An A-model $\Ac$ arising from symplectic geometry
	\item A B-model $\Bc_X$, coming from some K\"ahler manifold $X$, satisfying $\Bc_X(\pt) \simeq \Coh(X)$.
\end{itemize}
In particular, $\Ac(\pt)$ is often some category of geometric interest, and the equivalence $\Ac(\pt) \simeq \Bc_X(\pt)$ lets us resolve questions about $\Ac(\pt)$.

There's an analogue in higher dimensions: we'd like to take a 3d TFT $\Yc$ and give an equivalence $\Yc \simeq \Zc_X$ for some holomorphic symplectic $X$.
This would give an equivalence between some 2-category and $\Zc_X(\pt)$.

\begin{conj}[Teleman]
	Let $G$ be a complex reductive group with maximal compact subgroup $G_c$.
	There is an equivalence between:
	\begin{itemize}
		\item A suitable 2-category of ``categories with $G_c$-action.''
		\item The Rozansky-Witten 2-category of $T^*(G^\vee / G^\vee)$.
	\end{itemize}
\end{conj}

Note that $T^*(G^\vee / G^\vee)$ is stacky and non-proper, which makes it impossible for the corresponding 2-category to be 3-dualizable.
Thus we typically won't obtain 3-manifold invariants from such a theory.
That's terrible for 3-manifold topologists, but this isn't a 3-manifold seminar.

Some other examples of interest for Rozansky-Witten theory include symplectic resolutions and cotangent bundles of smooth algebraic varieties.

\subsection{What is the correct 2-category?}

To rigorously construct Rozansky-Witten theory, we'd need to give a definition of the 2-category $\RW_2 = \Zc_X(\pt)$.
This was studied by Kapustin, Rozansky, and Saulina, but much is still unknown.

Roughly, we expect $\RW_2$ to be a 2-category where:
\begin{itemize}
	\item Objects are smooth Lagrangians $L \subset X$ (or some suitable generalization of these).
	\item 1-morphisms from $L_1$ to $L_2$ are given by some sort of category associated to $L_1 \cap L_2$.
		In the simplest possible case, where $X = T^* W$ is a cotangent bundle, $L_1$ is the zero-section, and $L_2$ is the graph of a differential $df$, then $L_1 \cap L_2$ is the critical locus of $X$ and we assign $\Hom(L_1, L_2) = \MF_f$, the matrix factorization category of $f$.
		Work of Joyce and many others has focused on understanding how much the local setting looks like this.
	\item 2-morphisms and higher are ``natural compatibilities'' between the 1-morphisms.
\end{itemize}

One should think of the matrix factorization category $\MF_f$ as giving a categorical way to measure the critical locus of $f$.
When the critical points of $f$ are Morse, the category $\MF_f$ looks like a direct sum of copies of $\Vect$ (one for each critical point).

There is an important distinction between Rozansky-Witten theory and the 2d A-model.
In the complex setting, there are no ``instantons,'' so the theory is local and we don't run into the full difficulty of Floer theory.
Thus Rozansky-Witten theory is a categorified version of Fukaya theory that avoids the need for instanton corrections.

\subsection{An alternative viewpoint}

If $X = T^* W$ is a cotangent bundle, then $\ShvCat(W)$, the 2-category of (quasicoherent) sheaves of categories on $W$, embeds into $\RW_2$.
The image of this embedding consists of ``conic objects.''
Thus we can understand a key part of Rozansky-Witten theory, at least in this simple case.

The thesis (work in progress) of Enoch Yiu relates $\RW_2$ to $\ShvCat(W \times \AA^1)$.

\end{document}
