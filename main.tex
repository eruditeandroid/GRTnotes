\documentclass{article}

\usepackage{notes}

\title{GRT Seminar Fa23-Sp24 Notes}

\begin{document}

\maketitle

\begin{abstract}
	The seminar covers Ben-Zvi--Sakellaridis--Venkatesh, ``Relative Langlands Duality.''
\end{abstract}

\tableofcontents

\section{8/31 (David Nadler) -- ???}

I missed this day.
If you have good notes from this day, send them to me and I will type them up.

\section{9/7 (Elliot Kienzle) -- Hamiltonian $G$-Spaces and Quantization}

Elliot's notes for his talks are available at \url{https://chessapig.github.io/files/notes/G-spaces.pdf}.

The original Langlands program studies a duality of Lie groups $G \leftrightarrow G^\vee$.
Relative Langlands seeks to upgrade this to a duality of Hamiltonian $G$-actions $(G \curvearrowright M) \leftrightarrow (G^\vee \curvearrowright M^\vee)$.
This is proposed for hyperspherical varieties $M$, of which a typical example is $M = T^* X$ for $X$ a spherical variety.

We can approach and motivate this using quantization.
Start by considering the action of $G$ on $L^2(X)$ for $X$ a spherical variety (discussed in an earlier paper of Sakellaridis-Venkatesh discussing ``harmonic analysis on spherical varieties'').

\subsection{Symplectic geometry and quantization}

The original motivation for symplectic geometry comes from classical mechanics.
Suppose that we have a particle moving in $\RR^n$.
We can capture the data of the position and momentum using the cotangent bundle $T^* \RR^n$.
By Newton's second law, the time evolution of the particle is described by (the flow along) a vector field on $T^* \RR^n$.

We can generalize this to a symplectic manifold $(M, \omega)$, which is a manifold $M$ with a closed, non-degenerate $2$-form $\omega$.
To make this easier to work with, we can fix a metric $\ip{}{}$ on $M$ and write $\omega(x, y) = \ip{x}{Jy}$ where $J^2 = -1$ (i.e.\ $J^2$ is an almost complex structure).
We think of $J^2$ as ``multiplication by $-i$.''
Given a Hamiltonian $H \in \Cinfty(M)$, we obtain a Hamiltonian vector field $X_H = J \nabla H$.
More invariantly, we can define $X_H$ via the formula $\omega(X_H, -) = dH$.

Moving to quantum mechanics, we view a particle in $\RR^n$ as a $\CC$-valued function $\psi$ on $\RR^n$ (not $T^*\RR^n$).
In this case, the Hilbert space is $L^2(\RR^n)$.
A free particle evolves according to Schr\"odinger's equation:
\[
	i \dot{\psi} = \Delta \psi.
\]
We can summarize the classical and quantum pictures in the following table.

\begin{center}
\begin{tabular}{c|c|c}
	& Classical & Quantum \\ \hline
	State Space & Symplectic manifold $(M, \omega)$ & Hilbert space $\Hc$ \\
	Observables & $f \in \Cinfty(M)$ & Bounded operators $A \in \End(\Hc)$ \\
	Evolution & Vector fields $X_H$ for $H \in \Cinfty(M)$ & Unitary operators $U(t) = e^{itA}$ for $A \in \End(\Hc)$ \\
	Lie Algebra of observables & Poisson bracket $\{ f, g \} = X_f(g)$ & Commutator $[A, B]$
\end{tabular}
\end{center}

To obtain a quantum system from a classical system (heuristically), we pass from nonlinear evolution of points in $T^* M$ to linear evolution of functions on $M$.
(This linearity is forced on us by our desire to have superposition of states.)
The dream of quantization is, given a symplectic manifold $(M, \omega)$, to construct a Lie algebra homomorphism $(\Cinfty(M), \{ , \}) \to (\End(\Hc), [ , ])$ for some Hilbert space $\Hc$.
Unfortunately, this is impossible to do consistently / functorially in general.
However, there are some cases in which we can get good answers.

We will focus on geometric quantization, which behaves (loosely) as follows:
\begin{itemize}
	\item For $M = T^*X$, we obtain $\Hc = L^2(X)$.
	\item For $M$ a compact K\"ahler manifold, we obtain $\Hc = H^0(M, \Lc)$ for some line bundle $\Lc$ on $M$.
\end{itemize}

\subsection{$G$-Spaces}

We want to incorporate symmetries into the previous picture.
Suppose $G$ is a compact Lie group / reductive algebraic group (depending on context).
We say a symplectic $G$-space is a symplectic manifold $(M, \omega)$ with $G$-action preserving $\omega$.
We can hope to quantize this to a linear representation $G \curvearrowright \Hc$.
(There are subtleties that arise here -- for geometric quantization, we would like a $G$-equivariant polarization.)

In general, it is better to consider Hamiltonian $G$-actions, where $\gf$ acts by Hamiltonian vector fields.
This allows us to construct a moment map $\mu: M \to \gf^*$ which is equivariant (with respect to the coadjoint action on $\gf^*$).

Let us start by understanding the coadjoint action $G \curvearrowright \gf^*$ using Kirillov's ``orbit method.''
For $\alpha \in \gf^*$, consider the coadjoint orbit $\Oc_\alpha$.
This $\Oc_\alpha$ turns out to be a symplectic manifold (with ``Kirillov-Kostant-Souriau'' / ``KKS'' form) with Hamiltonian $G$-action, and the moment map $\Oc_\alpha \to \gf^*$ is just the inclusion.

\begin{ex}
	Consider $G = \SO(3)$.
	The coadjoint action is just $\SO(3)$ acting on $\RR^3$ by rotations.
	Thus the generic orbits are spheres $S^2$.
\end{ex}

The orbits $\Oc_\alpha$ will look like generalized flag manifolds, and conversely every generalized flag manifold arises in this way.
(This is the first place where our compactness hypothesis comes in).

\begin{prop}
	A coadjoint orbit $\Oc_\alpha$ is quantizable if and only if $\alpha$ is in the orbit of an integer point of the root lattice $\tf_\ZZ^* \subset \tf^*$ (viewed as a subspace of $\gf^*$ via the Killing form).
\end{prop}

\begin{ex}
	Continuing on with our $\SO(3)$ example, we see that a symplectic sphere is quantizable if and only if it has integer area.
\end{ex}

In these cases, the quantization of $\Oc_\alpha$ is $H^0(\Oc_\alpha, \Lc_\alpha)$ where $\Lc_\alpha$ is the line bundle corresponding to the character $\alpha$.
By the Borel-Weil theorem, $H^0(\Oc_\alpha, \Lc_\alpha)$ is the irrep $V_\alpha$ of $G$ with highest weight $\Lc_\alpha$.

We can summarize this in the following table:

\begin{center}
	\begin{tabular}{c|c}
		Classical & Quantum \\ \hline
		Symplectic action $G \curvearrowright M$ & Representation $G \curvearrowright \Hc$ \\
		Coadjoint orbit $\Oc_\alpha$ & Highest weight representation $E_\alpha$
	\end{tabular}
\end{center}

\section{9/14 (Elliot Kienzle) -- Continued}

\subsection{Symplectic reduction}

Suppose we have a Hamiltonian action $G \curvearrowright M$.
This yields a $G$-equivariant moment map $\mu: M \to \gf^*$, and the image of $\mu$ will necessarily be a collection of coadjoint orbits $\Oc_\alpha$.
We can use these orbits to decompose $M$.

First consider the orbit $\Oc_0 = \{ 0 \}$.
We note that $\mu\inv(0)$ is $G$-invariant, so we can consider the quotient $\mu\inv(0) / G$.
We define this to be the \emph{symplectic quotient}: $M // G := \mu\inv(0) / G$.

We will assume that $0$ is a regular value of the moment map and that $G$ acts on $\mu\inv(0)$ freely.
We can drop these assumptions if we consider things in a suitable derived / stacky sense.

\begin{thm}[Marsden-Weinstein]
	The symplectic quotient $M \sslash G$ carries a natural symplectic structure.
\end{thm}

\begin{ex}
	If $X$ is a (not necessarily symplectic) manifold with a $G$-action, then $T^*X \sslash G = T^* (X/G)$.
\end{ex}

\begin{ex}
	Let $M = T^* \RR^2 \cong \CC^2$.
	This has a $\U(1)$-action via 
	\[
		e^{i\theta} (z_1, z_2) = (e^{i \theta} z_1, e^{i \theta} z_2).
	\]
	We can define a (shifted) moment map $\mu: \CC^2 \to \RR$ via 
	\[
		\mu(z_1, z_2) = \abs{z_1}^2 + \abs{z_2}^2 - 1.
	\]
	Then $\CC^2 \sslash \U(1) = S^3 / \U(1) = S^2 = \PP^1$ (consider the Hopf fibration).
\end{ex}

Morally, we should think of every symplectic manifold as a symplectic reduction of a (possibly infinite-dimensional) affine space.

Note that
\[
\dim M \sslash G = \dim M - 2 \dim G.
\]
The slogan is that ``in symplectic geometry, groups act twice.''

\begin{thm}[Guillemin-Sternberg, etc.]
	The geometric quantization of a symplectic quotient satisfies
	\[
		\Hc(M \sslash G) = \Hc(M)^G,
	\]
	where the right hand side is the subspace of $G$-invariant vectors in $G$.
\end{thm}

We can also define the symplectic reduction along any coadjoint orbit $\Oc_\alpha$ as $M \sslash_\alpha G = \mu\inv(\Oc_\alpha) / G$.
This gives a decomposition of $M$ as 
\[
	M = \cup_{\alpha \in \mu(M)} \mu\inv(\Oc_\alpha) = \cup_{\alpha \in \mu(M)} (G\textrm{-bundles over } M \sslash_\alpha G),
\]
at least if we avoid critical points.

Elliot has some fancy art of this decomposition.

Let's focus on the simplest possible case:

\begin{dfn}
	A Hamiltonian $G$-space $M$ is \emph{multiplicity-free} if $\dim M \sslash_\alpha G = 0$ for all $\alpha$.
\end{dfn}

\begin{rmk}
	If $M$ is compact, then a Morse theory argument shows that $M \sslash_\alpha G = \pt$ for all $\alpha$.
\end{rmk}

Here are some relevant examples.

\begin{ex}
	For a coadjoint orbit $\Oc_\alpha$, we have $\Oc_\alpha \sslash_\alpha G = \pt$, so coadjoint orbits are multiplicity-free.
	Here we are ignoring stacky / derived quotients even though the action is typically nonfree.
\end{ex}

\begin{ex}
	Consider $\PP^1$ with $\U(1)$ acting by rotation.
	Then $\mu$ is the height function on $\PP^1 = S^2$.
	If the top height is $1$ and the bottom height is $-1$, then $\mu\inv(1)$ and $\mu\inv(1)$ are both points.
	For any $x \in (-1, 1)$, we have $\mu\inv(x) = S^1$ and therefore $\PP^1 \sslash_x \U(1) = \pt$.
	Thus this action is multiplicity-free.
\end{ex}

\begin{ex}
	Let $\U(1)^2$ acts on $\PP^2$ (extending the standard action on $\AA^2 \subset \PP^2$).
	The fibers of the moment map over points in the interior of $\mu(M)$ are $2$-tori, which degenerate to circles on the boundary lines of $\mu(M)$ and points at the corners of $\mu(M)$.
\end{ex}

A non-example is given by the $\U(1)$ action on $\CC^2$ from earlier in the lecture.
This is an obvious non-example because the dimension of the symplectic quotient is nonzero.
The slogan is that ``multiplicity-free manifolds have maximal symmetry.''

\subsection{(David) -- Interlude}

For a Lie group $G$, we have $T^*G = G \times \gf^*$.
Consider $G \curvearrowright T^*G$ induced by the adjoint action of $G$ on itself.
We obtain a moment map $\mu: G \times \gf^* \to \gf^*$ given by the formula
\[
	\mu(g, x) = \Ad_g(x) - x.
\]
Then $\mu\inv(0) = \{ (g, x) \in G \times \gf^* \, | \, g \in G_x \}$, where $G_x$ is the centralizer of $x \in G$.

The multiplicity-freeness property for a general Hamiltonian $G$-space $M$ can be understood as the requirement that the centralizers $G_x$ act transitively on the preimages $\mu\inv(x)$.

It is a good exercise to classify multiplicity-free Hamiltonian $G$-spaces for $G = \U(1)$ or $G = \SU(2)$.

\subsection{(Elliot) -- A few last words}

Multiplicity-freeness has a useful consequence for quantization: if $M$ is multiplicity-free, then each highest weight representation $E_\alpha$ appears in $\Hc(M)$ at most once.
In fact, $E_\alpha$ will appear if and only if $\Oc_\alpha \in \mu(M)$.

We will be interested in hyperspherical varieties as a large family of multiplicity-free symplectic manifolds.
More on that next time!

\section{9/21 (Mark Macerato) -- Hyperspherical Varieties}

\subsection{(David) -- Multiplicity-freeness}

There may have been minor errors in the discussion last time, but the basic ideas were right.
Suppose for simplicity that $T$ is an \emph{abelian} Lie group, and consider the cotangent bundle $T^*T \cong T \times \tf^*$.
The moment map for the translation action of $T$ on itself is the projection $T \times \tf^* \to \tf^*$.
This gives a (trivial) family of abelian groups over $\tf^*$.

If we have another Hamiltonian $T$-space $X$, we obtain a moment map $\mu_X: X \to \tf^*$.
We can view our family of abelian groups over $\tf^*$ as acting fiberwise on $X$.
The multiplicity-freeness condition is requiring that the orbits of this action are fiberwise discrete.

This story still works for non-abelian $G$ (but you have to be careful about left versus right actions).
In this case, the fiber over $v \in \gf^*$ will be given by the stabilizer $G_v$.

\begin{ex}
	We can describe Hamiltonian $\U(1)$-spaces as lying over $\uf(1) \cong \RR$.
	The multiplicity-freeness condition implies that the fibers are (disjoint unions of) copies of $S^1$ and points.
	For example, we can consider the height function on the sphere, or the projection of a cylinder $S^1 \times \RR$, or many related examples -- these all give multiplicity-free Hamiltonian $\U(1)$-spaces.
\end{ex}

\begin{ex}
	If we take $G = \SU(2)$, we obtain a similar (but distinct) picture because $\suf(2) / \SU(2) \cong [0, \infty)$ (the $\SU(2)$-orbits in $\suf(2)$ are spheres).
	The fibers of $T^*\SU(2) \to \suf(2)$ are $\SU(2)$ (over $0$) and $S^1$ (over points in $(0, \infty)$).
	We can analyze multiplicity-free Hamiltonian $G$-spaces as before.
\end{ex}

In general, the left action $G \curvearrowright T^*G$ (via $g \cdot (h, v) = (gh, \Ad_g v)$) is not multiplicity-free.
Consider the moment map $T^*G \cong G \times \gf^* \to \gf^*$ given by projection (this depends on how we trivialize $T^*G$).
For a coadjoint orbit $\Oc$, the preimage $\mu\inv(\Oc)$ is $G \times \Oc$.
The multiplicity-freeness here reduces to the question of whether the action $G_v \curvearrowright G$ has discrete orbits.
This is not true in general (see e.g.\ the $\SU(2)$ example above), proving the claim.

\emph{A later clarification:} 
Really, we should think of $T^*G \rightrightarrows \gf^*$ as a groupoid, where the ``source'' and ``target'' maps are $\mu_L$ and $\mu_R$ (the moment maps for the left / right actions, respectively).
Given a groupoid, we can obtain a group scheme (encapsulating the ``automorphism groups of points'') as a fiber product, e.g.\
\[
	\begin{tikzcd}
		\{ [X, g] = 0 \} \rar \dar & T^* G \dar \\
		\Delta \rar & \gf^* \times \gf^*.
	\end{tikzcd}
\]
Understanding things from this perspective clears up the difficulties with left / right actions.

Hamiltonian $G$-spaces ($M \to \gf^*$) will be module objects for this groupoid.

\subsection{(Mark) -- Towards hyperspherical varieties}

We will change settings to algebraic geometry (following section 3 of Ben-Zvi--Sakellaridis-Venkatesh).
Fix an algebraically closed field $k$ of characteristic zero (e.g.\ $\CC$ or $\ol{\QQ_\ell}$).
Let $G$ be a connected reductive group over $k$.

Recall that a spherical variety is a normal $G$-variety $X$ such that there exists a Borel subgroup $B \subset G$ with an open orbit in $X$.
We can rephrase the last condition without picking a Borel: we require that $G$ has an open orbit on $X \times \Fl_G$.
If $X$ is affine, this is equivalent to requiring that the coordinate ring $k[X]$ is multiplicity-free as a $G$-module.

\begin{ex}[``Group case'']
	Let $H$ be a connected reductive group and $G = H \times H$.
	For $X = H$ and $G \hookrightarrow X$ via $(h_1, h_2) \cdot h = h_1 h h_2\inv$, $H$ is a spherical variety.

	If we fix a Borel $B \subset H$, we have a unipotent subgroup $U \subset B$ and a surjection $B \twoheadrightarrow T = B/U$.
	By Levi's theorem, this splits, giving $T \hookrightarrow B \subset G$.
	We get a vector space decomposition $\gf = \uf^- \oplus \tf \oplus \uf$.
	Consider the open embedding $U^- \times B \to H$ given by $(u, b) \mapsto ub$.
	The Borel subgroup $B^- \times B \subset G$ has an open orbit in $H$.
	This leads to a Bruhat decomposition $H = \sqcup_{w \in W} B w B$.
\end{ex}

We can obtain Bruhat decompositions for more general spherical varieties.
This is a rich theory that has been worked out by several authors (Knapp, Brion, etc.).
But let's move on to hyperspherical varieties, which give a symplectic point of view.

Instead of a spherical variety $X$, let us consider $M = T^*X$ with the moment map $\mu: T^*X \to M$.
For simplicity, we will assume our base spherical variety $X$ is affine, smooth, and irreducible.
In this case $M$ is \emph{coisotropic}, which means that the $G$-invariant function field $k(M)^G$ is Poisson-commutative.

Another way of saying this is as follows.
Let $\cf = \gf^* \sslash G \cong \gf \sslash G$ be the ``Chevalley space.''
Letting $\eta \in M$ be the generic point, we obtain a Stein factorization $M \to \cf_M \to \cf$.
The map $\tilde{\mu}: M \to \cf_M$ has connected generic fiber, and $\cf_M \to \cf$ is finite.
The second definition of ``coisotropic'' is that the group $G_{K(\cf_M)}$ acts on $M_{K(\cf_M)}$ with an open (hence dense) orbit.

\begin{thm}[Losev]
	If $M$ is a smooth Hamiltonian $G$-variety, then \emph{all} of the fibers of $\tilde{\mu}: M \to \cf_M$ are connected.\footnote{This is the closest analogue in algebraic geometry of the connectedness theorem of Atiyah-Guillemin-Sternberg.}
\end{thm}

A third definition of coisotropic is that the generic $G$-orbit on $M$ is coisotropic in the usual sense.

``Coisotropic'' is the algebraic geometry version of ``multiplicity-free.''
Elliot gave a discussion of why this recovers the earlier condition in symplectic geometry, but it was a bit too fast to type up.

\section{9/28 (Mark Macerato) -- Continued}

\subsection{(David) -- Groupoids and Hamiltonian $G$-spaces}

Recall the homework problem of classifying multiplicity-free $\SU(2)$-spaces.

The corrected general picture is as follows.
Consider the cotangent bundle $T^*G$ with natural Hamiltonian $G$-actions on the left and right.
These yield moment maps $\mu_L, \mu_R: T^*G \to \gf^*$.
If we trivialize $T^*G \cong G \times \gf^*$, these maps are given by $(g, X) \mapsto X$ and $(g, X) \mapsto \Ad_g X$.

We should think of $T^*G \rightrightarrows \gf^*$ as a groupoid.
The ``objects'' are $X \in G$, and the ``morphisms'' are $g: X \to \Ad_g X$.
Composition is given by group multiplication.

We may view any Hamiltonian $G$-space $Y$ (with moment map $\mu: Y \to \gf^*$) as a module over this groupoid.
Specifically, we have a natural map $T^*G \times_{\gf^*} Y \to Y$, the projection of the fiber product onto the second factor.
On elements, this is given by $(g, X, y) \mapsto gy$, which lies in the fiber of $Y$ over $\Ad_g X \in \gf^*$.

Consider the pullback
\[
	\begin{tikzcd}
		\Sc \rar \dar & T^*G \dar \\
		\gf^* \rar["\Delta"] & \gf^* \times \gf^*.
	\end{tikzcd}
\]
In equation, $\Sc = \{ [g, X] = 0 \}$.
From the groupoid perspective, $\Sc \to \gf^*$ is obtained by only considering automorphisms of objects in our original groupoid (i.e.\ forgetting about isomorphisms between different objects).
We can view $\Sc \to \gf^*$ as the relative group over $\gf^*$ with fibers given by stabilizers $\Stab_G(X)$.

The ``multiplicity-free'' condition can now be restated: it means that the $\Sc$-action on $Y$ relative to $\gf$ has only finitely many orbits.

For the exercise about $\SU(2)$, we have $\gf^* = \RR^3$, and $\Sc$ has fiber $\SU(2)$ over the identity and $\U(1)$ over other fibers.
We really only care about $\gf^* / \SU(2)$, which looks like a real ray $[0, \infty)$.
This allows us to produce some examples of multiplicity-free Hamiltonian $\SU(2)$-spaces - these spaces should have maps to $[0, \infty)$ with fibers over $X \in \gf^* / \SU(2) \cong [0, \infty)$ looking like (finite disjoint unions of) orbits of $\Stab_{\SU(2)}(X)$-actions.

\begin{ex}
	The $2$-sphere $S^2$ has multiplicity-free $\SU(2)$-action via the action coming from $\SU(2) \to \SO(3)$.
\end{ex}

\begin{ex}
	The standard representation $\CC^2$ has multiplicity-free $\SU(2)$-action.
\end{ex}

\begin{ex}
	The blowup of $\CC^2$ at the origin (with a corrected symplectic form) has multiplicity-free $\SU(2)$-action.
\end{ex}

Are these all of the possible examples (up to finite covers)?
It would be good to figure this out.

\subsection{(Mark) -- Coisotropic $G$-varieties}

Recall our setup: $G$ is connected and reductive, and $M$ is a smooth affine Hamiltonian $G$-variety.
We have a moment map $\mu: M \to \gf^*$, and we can compose this with a GIT quotient map to get $\ol{\mu}: M \to \cf_G$, where $\cf_G = \gf^* \sslash G$ is called the Chevalley base.
This admits a ``Knop factorization''
\[
\begin{tikzcd}
	M \ar[rr, "\ol{\mu}"] \ar[dr, "\tilde{\mu}_M"] & & \cf_G \\
	& \cf_M \ar[ur, "\pi"] &
\end{tikzcd}
\]
where $\pi$ is finite and $\tilde{\mu}_M$ has generically connected fiber.

\begin{dfn}
	We say that $M$ is \emph{coisotropic} if any of the following equivalent conditions hold.
	\begin{enumerate}
		\item $k(M)^G$ is Poisson-commutative.\footnote{In this setup, we can replace this by the condition that $k[M]^G$ is Poisson commutative, since $\mathrm{Frac} k[M]^G = k(M)^G$.}
		\item The generic orbit of $G$ on $M$ is coisotropic.
		\item The generic fiber of $\tilde{\mu}_M$ has a dense $G$-orbit.
	\end{enumerate}
\end{dfn}

Let's see why $1$ and $2$ are equivalent.
Choose $f_1, \dots, f_n \in K(M)$ which separate generic orbits (this is possible by a theorem of Rosenlicht).
This yields $\ul{f} = (f_1, \dots, f_n): U \to \AA^n$ (for $U \subset M$ open), and we can restrict this to a surjective smooth map $U' \to W$  such that $U'$ is dense in $U$ and $W \subset \AA^n$ is a locally closed subvariety.
Replace $U$ by $U'$.
The fibers of $\ul{f}$ are exactly the $G$-orbits in $U$.
Therefore, for $x \in U$, we see that $df_1(x), \dots, df_n(x)$ span the conormal space $T^*_U(G \cdot x)_x$.
Thus $G \cdot x$ is coisotropic at $x$ if and only if $T_U^*(G \cdot x)_x$ is isotropic, if and only if the $f_1, \dots, f_n$ Poisson-commute at $x$.

\subsection{Approaching hyperspherical varieties}

Suppose that $M$ is a smooth affine Hamiltonian $G$-variety as before.
We will also require that $M$ comes with a $\GG_m$-action (equivalently, a grading on $k[M]$) such that
\begin{enumerate}
	\item The $\GG_m$-action on $M$ commutes with the $G$-action.
	\item The symplectic form $\omega$ on $M$ has weight $2$, i.e.\ $\lambda \cdot \omega = \lambda^2 \omega$.
\end{enumerate}
David noted that this latter condition implies that $\omega$ is exact: if $v$ is the vector field generating the $\GG_m$-action, then Cartan's magic formula (using that $\omega$ is closed) gives
\[
	2 \omega = \Lc_v \omega = d(i_v \omega).
\]
The $2$ here is needed to ensure that we can construct a ``$\GG_m$-equivariant Kostant slice.''

We want to define what it means for $M$ to be hyperspherical.
The first condition will be that $M$ is coisotropic.

The second condition is that $\mu(M) \subset \gf^*$ meets the nilpotent cone $\Nc_G = \chi\inv(0)$ (for $\chi: \gf^* \twoheadrightarrow \gf^* \sslash G$).
Equivalently, $\ol{\mu})(M)$ contains $0 \in \cf_G$.
This implies that $M \sslash G \to \cf_M$ is surjective (it is always an open immersion, so we get $M \sslash G = \cf_M$).
There will be two more conditions (which we will discuss next time).

\section{10/5 (Mark Macerato) -- Continued}

\subsection{Pre-hyperspherical varieties}

Let $G$ be a connected reductive group and $G_{\gr} = G \times \GG_m$.
We consider a smooth affine Hamiltonian $G$-variety with auxiliary $\GG_m$-action governing the grading.
This yields a map $M \to \gf^* \to \cf_G$, which has a Knop factorization $M \to \cf_M \to \cf_G$.
Here $\GG_m \curvearrowright \gf^*$ quadratically, and the map $M \to \gf^*$ is $\GG_m$-equivariant.

\begin{dfn}
	We say that $M$ is \emph{pre-hyperspherical} if
	\begin{enumerate}
		\item $M$ is coisotropic, i.e.\ $k(M)^G$ is Poisson commutative (equivalently, the generic fiber of $M \to \cf_M$ has a dense $G$-orbit),
		\item $\mu(M) \cap \Nc_G \neq \emptyset$ (for $\Nc_G$ the nilpotent cone of $G$), and
		\item The stabilizer of a generic point of $M$ is connected.
	\end{enumerate}
\end{dfn}

\begin{ex}
	Let $G = \Sp_{2n}$ and $M = \CC^{2n} \oplus \CC^{2n}$.
	Here $\mu_M: \CC^{2n} \oplus \CC^{2n} \to (\CC^{2n} \oplus \CC^{2n}) \sslash \Sp_{2n} \cong \AA^1$ via $(v, w) \mapsto \omega(v, w)$.
	Thus
	\[
		\mu_M\inv(1) = \{ (v, w) \in \CC^{2n} \, | \, \omega(v, w) = 1 \},
	\]
	and $\Sp_{2n}$ acts transitively on this fiber.
	Meanwhile, $\mu\inv(0)$ can be decomposed as:
	\[
		\mu\inv(0) = \{ (v, w) \, | \, v, w \textrm{ lin.\ ind.\, } \omega(v, w) = 0 \} \cup \{ (v, w) \, | \, v, w \textrm{lin. dep., not both } 0 \} \cup \{ (0, 0) \}.
	\]
	The first set here is the unique open orbit, and the last set is the unique closed orbit.
	The middle set contains a $\PP^1$ worth of orbits.
	In particular, $\mu(M)$ meets $\Nc_G$.
	The stabilizer of a generic point of $M$ can be identified with $\Sp_{2n-2}$.
\end{ex}

\begin{prop}
	In general, if $M$ is pre-hyperspherical, there exists a unique closed orbit $M_0 \subset M$ for $G_{\gr} = G \times \GG_m$.
\end{prop}

We call $M_0$ the \emph{core} of $M$.

\begin{proof}
	Consider the GIT quotient $M \to M \sslash G_{\gr}$, and recall that closed orbits of $G_{\gr}$ correspond to points of $M \sslash G_{\gr}$.
	Thus it suffices to show that $M \sslash G_{\gr} = \pt$, or equivalently $k[M]^{G \times \GG_m} = k$.
	Note
	\[
		k[M]^{G \times \GG_m} = k[\cf_M]^{\GG_m} = k[\cf_m]_0,
	\]
	the weight $0$ component.
	By construction, $k[\cf_G] \to k[\cf_M]$ is finite (and therefore integral), so $k[\cf_M]_0 \to k[\cf_G]_0$ is integral (an exercise in counting degrees).
	Now we have
	\[
		k[\cf_G]_0 = k[\cf_G]^{\GG_m} = k[\gf^*]^{G \times \GG_m} = k,
	\]
	so $k[\cf_M]_0$ is integral over $k$.
	Since $k$ is algebraically closed, we see $k[\cf_M]_0 = k$.
\end{proof}

\subsection{(David) -- Weinstein manifolds}

Suppose we have an exact symplectic manifold $(M, \omega = d\lambda)$.
Take $Z = \omega\inv(\lambda)$ (this is a vector field on $M$).
Then $Z$ gives a flow on $M$, and the core $M_0$ is the subset of points of $M$ which do not escape to infinity along this flow.
Assuming $M_0$ is isotropic (this is implied by the Weinstein condition), the flow gives an action of $\CC^\times$ on $M_0$.

\begin{ex}
	Consider the surface singularity $x^2 + y^2 + z^{n+1} = 0$.
	Let $M$ be a symplectic resolution of this.
	The core $M_0$ is the chain of $\PP^1$'s appearing as the zero fiber.
	In terms of geometric representation theory, we can call $M_0$ a ``subregular Springer fiber.''
\end{ex}

Suppose $M$ is $G$-Hamiltonian -- then we are in a situation very similar to what Mark is talking about.
That is, \emph{pre-hyperspherical varieties are analogous to $G$-Hamiltonian Weinstein manifolds}.
This precludes examples like the above, since the union of $\PP^1$'s cannot be a single $G$-orbit.
This fact is essentially kin to the last Proposition.

\begin{ex}
	Consider $G \times G$ acting on $M = T^* G$ by left and right translation.
	The core $M_0$ is the zero section.
\end{ex}

\subsection{(Mark) -- Hyperspherical varieties}

Let $\mu_M: M \to M \sslash G$ be the GIT quotient map.

\begin{prop}
	The core $M_0$ is the unique closed $G$-orbit in $\mu_M\inv(0)$.
\end{prop}

\begin{proof}
	Note that $G$ has a unique closed orbit $M_0' \subset \mu_M\inv(0)$ (by standard GIT).
	Since $\GG_m$ commutes with $G$, the $\GG_m$ action takes closed $G$-orbits to closed $G$-orbits.
	Therefore $\GG_m$ preserves $M_0'$, and we get $G \times \GG_m \curvearrowright M_0'$.
	It follows that $M_0'$ contains a closed $G_{\gr}$ orbit, hence contains $M_0$.
	But $M_0'$ is itself a $G$-orbit, so $M_0 = M_0'$.
\end{proof}

Pick $x \in M_0$, and let $H = \Stab_G(x)$, so $M_0 = G / H$.
Since $M_0$ is affine, $H$ must be reductive.
Since $\mu_M\inv(0)$ maps to $\Nc_G \subset \gf^*$, we get an element $f = \mu(x) \in \Nc_G$.

Because $\GG_m \curvearrowright M_0 = G / H$, we get a homomorphism $\GG_M \to \Aut_G(G / H) \cong N_G(H) / H$.
In fact, this factors through $(Z_G(H) / Z(H))^0$ (where ${}^0$ denotes the connected component of the identity element).
Let $\ol{\pi}: \GG_M \to Z_G(H) / Z(H)$ be the induced map.

\begin{dfn}
	A pre-hyperspherical variety $M$ is hyperspherical if
	\begin{enumerate}
		\item $\ol{\pi}$ lifts to a homomorphism $\pi: \GG_m \to Z_G(H) \subset G$, which moreover lifts to a homomorphism $\rho: \SL_2 \to G$ such that
			\[
				d\rho \begin{pmatrix}
					0 & 1 \\
					0 & 0
				\end{pmatrix} = f
			\]
			under the standard identification $\gf \cong \gf^*$.
		\item Consider the \emph{sheared $\GG_m$-action} $(\GG_m)_{sh} \curvearrowright M$ induced by
			\begin{align*}
				(\GG_m)_{sh} &\to G \times \GG_m \\
				g &\mapsto (\pi(g)\inv, g).
			\end{align*}
			By construction, $(\GG_m)_{sh}$ fixes $x$, and thus we get $(\GG_m)_{sh} \curvearrowright (T_x M_0)^\perp / (T_x M_0 \cap T_x M_0^\perp) := N_x M_0$, the \emph{symplectic normal space}.
			The condition is that this $(\GG_m)_{sh}$-action on $N_x M_0$ is given by linear scaling.
	\end{enumerate}
\end{dfn}

\end{document}

