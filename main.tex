\documentclass{article}

\usepackage{notes}

\title{GRT Seminar Fa23-Sp24 Notes}

\begin{document}

\maketitle

\begin{abstract}
	The seminar covers Ben-Zvi--Sakellaridis--Venkatesh, ``Relative Langlands Duality.''
\end{abstract}

\tableofcontents

\section{8/31 (David Nadler) -- ???}

I missed this day.
If you have good notes from this day, send them to me and I will type them up.

\section{9/7 (Elliot Kienzle) -- Hamiltonian $G$-Spaces and Quantization}

Elliot's notes for his talks are available at \url{https://chessapig.github.io/files/notes/G-spaces.pdf}.

The original Langlands program studies a duality of Lie groups $G \leftrightarrow G^\vee$.
Relative Langlands seeks to upgrade this to a duality of Hamiltonian $G$-actions $(G \curvearrowright M) \leftrightarrow (G^\vee \curvearrowright M^\vee)$.
This is proposed for hyperspherical varieties $M$, of which a typical example is $M = T^* X$ for $X$ a spherical variety.

We can approach and motivate this using quantization.
Start by considering the action of $G$ on $L^2(X)$ for $X$ a spherical variety (discussed in an earlier paper of Sakellaridis-Venkatesh discussing ``harmonic analysis on spherical varieties'').

\subsection{Symplectic geometry and quantization}

The original motivation for symplectic geometry comes from classical mechanics.
Suppose that we have a particle moving in $\RR^n$.
We can capture the data of the position and momentum using the cotangent bundle $T^* \RR^n$.
By Newton's second law, the time evolution of the particle is described by (the flow along) a vector field on $T^* \RR^n$.

We can generalize this to a symplectic manifold $(M, \omega)$, which is a manifold $M$ with a closed, non-degenerate $2$-form $\omega$.
To make this easier to work with, we can fix a metric $\ip{}{}$ on $M$ and write $\omega(x, y) = \ip{x}{Jy}$ where $J^2 = -1$ (i.e.\ $J^2$ is an almost complex structure).
We think of $J^2$ as ``multiplication by $-i$.''
Given a Hamiltonian $H \in \Cinfty(M)$, we obtain a Hamiltonian vector field $X_H = J \nabla H$.
More invariantly, we can define $X_H$ via the formula $\omega(X_H, -) = dH$.

Moving to quantum mechanics, we view a particle in $\RR^n$ as a $\CC$-valued function $\psi$ on $\RR^n$ (not $T^*\RR^n$).
In this case, the Hilbert space is $L^2(\RR^n)$.
A free particle evolves according to Schr\"odinger's equation:
\[
	i \dot{\psi} = \Delta \psi.
\]
We can summarize the classical and quantum pictures in the following table.

\begin{center}
\begin{tabular}{c|c|c}
	& Classical & Quantum \\ \hline
	State Space & Symplectic manifold $(M, \omega)$ & Hilbert space $\Hc$ \\
	Observables & $f \in \Cinfty(M)$ & Bounded operators $A \in \End(\Hc)$ \\
	Evolution & Vector fields $X_H$ for $H \in \Cinfty(M)$ & Unitary operators $U(t) = e^{itA}$ for $A \in \End(\Hc)$ \\
	Lie Algebra of observables & Poisson bracket $\{ f, g \} = X_f(g)$ & Commutator $[A, B]$
\end{tabular}
\end{center}

To obtain a quantum system from a classical system (heuristically), we pass from nonlinear evolution of points in $T^* M$ to linear evolution of functions on $M$.
(This linearity is forced on us by our desire to have superposition of states.)
The dream of quantization is, given a symplectic manifold $(M, \omega)$, to construct a Lie algebra homomorphism $(\Cinfty(M), \{ , \}) \to (\End(\Hc), [ , ])$ for some Hilbert space $\Hc$.
Unfortunately, this is impossible to do consistently / functorially in general.
However, there are some cases in which we can get good answers.

We will focus on geometric quantization, which behaves (loosely) as follows:
\begin{itemize}
	\item For $M = T^*X$, we obtain $\Hc = L^2(X)$.
	\item For $M$ a compact K\"ahler manifold, we obtain $\Hc = H^0(M, \Lc)$ for some line bundle $\Lc$ on $M$.
\end{itemize}

\subsection{$G$-Spaces}

We want to incorporate symmetries into the previous picture.
Suppose $G$ is a compact Lie group / reductive algebraic group (depending on context).
We say a symplectic $G$-space is a symplectic manifold $(M, \omega)$ with $G$-action preserving $\omega$.
We can hope to quantize this to a linear representation $G \curvearrowright \Hc$.
(There are subtleties that arise here -- for geometric quantization, we would like a $G$-equivariant polarization.)

In general, it is better to consider Hamiltonian $G$-actions, where $\gf$ acts by Hamiltonian vector fields.
This allows us to construct a moment map $\mu: M \to \gf^*$ which is equivariant (with respect to the coadjoint action on $\gf^*$).

Let us start by understanding the coadjoint action $G \curvearrowright \gf^*$ using Kirillov's ``orbit method.''
For $\alpha \in \gf^*$, consider the coadjoint orbit $\Oc_\alpha$.
This $\Oc_\alpha$ turns out to be a symplectic manifold (with ``Kirillov-Kostant-Souriau'' / ``KKS'' form) with Hamiltonian $G$-action, and the moment map $\Oc_\alpha \to \gf^*$ is just the inclusion.

\begin{ex}
	Consider $G = \SO(3)$.
	The coadjoint action is just $\SO(3)$ acting on $\RR^3$ by rotations.
	Thus the generic orbits are spheres $S^2$.
\end{ex}

The orbits $\Oc_\alpha$ will look like generalized flag manifolds, and conversely every generalized flag manifold arises in this way.
(This is the first place where our compactness hypothesis comes in).

\begin{prop}
	A coadjoint orbit $\Oc_\alpha$ is quantizable if and only if $\alpha$ is in the orbit of an integer point of the root lattice $\tf_\ZZ^* \subset \tf^*$ (viewed as a subspace of $\gf^*$ via the Killing form).
\end{prop}

\begin{ex}
	Continuing on with our $\SO(3)$ example, we see that a symplectic sphere is quantizable if and only if it has integer area.
\end{ex}

In these cases, the quantization of $\Oc_\alpha$ is $H^0(\Oc_\alpha, \Lc_\alpha)$ where $\Lc_\alpha$ is the line bundle corresponding to the character $\alpha$.
By the Borel-Weil theorem, $H^0(\Oc_\alpha, \Lc_\alpha)$ is the irrep $V_\alpha$ of $G$ with highest weight $\Lc_\alpha$.

We can summarize this in the following table:

\begin{center}
	\begin{tabular}{c|c}
		Classical & Quantum \\ \hline
		Symplectic action $G \curvearrowright M$ & Representation $G \curvearrowright \Hc$ \\
		Coadjoint orbit $\Oc_\alpha$ & Highest weight representation $E_\alpha$
	\end{tabular}
\end{center}

\section{9/14 (Elliot Kienzle) -- Continued}

\subsection{Symplectic reduction}

Suppose we have a Hamiltonian action $G \curvearrowright M$.
This yields a $G$-equivariant moment map $\mu: M \to \gf^*$, and the image of $\mu$ will necessarily be a collection of coadjoint orbits $\Oc_\alpha$.
We can use these orbits to decompose $M$.

First consider the orbit $\Oc_0 = \{ 0 \}$.
We note that $\mu\inv(0)$ is $G$-invariant, so we can consider the quotient $\mu\inv(0) / G$.
We define this to be the \emph{symplectic quotient}: $M // G := \mu\inv(0) / G$.

We will assume that $0$ is a regular value of the moment map and that $G$ acts on $\mu\inv(0)$ freely.
We can drop these assumptions if we consider things in a suitable derived / stacky sense.

\begin{thm}[Marsden-Weinstein]
	The symplectic quotient $M \sslash G$ carries a natural symplectic structure.
\end{thm}

\begin{ex}
	If $X$ is a (not necessarily symplectic) manifold with a $G$-action, then $T^*X \sslash G = T^* (X/G)$.
\end{ex}

\begin{ex}
	Let $M = T^* \RR^2 \cong \CC^2$.
	This has a $\U(1)$-action via 
	\[
		e^{i\theta} (z_1, z_2) = (e^{i \theta} z_1, e^{i \theta} z_2).
	\]
	We can define a (shifted) moment map $\mu: \CC^2 \to \RR$ via 
	\[
		\mu(z_1, z_2) = \abs{z_1}^2 + \abs{z_2}^2 - 1.
	\]
	Then $\CC^2 \sslash \U(1) = S^3 / \U(1) = S^2 = \PP^1$ (consider the Hopf fibration).
\end{ex}

Morally, we should think of every symplectic manifold as a symplectic reduction of a (possibly infinite-dimensional) affine space.

Note that
\[
\dim M \sslash G = \dim M - 2 \dim G.
\]
The slogan is that ``in symplectic geometry, groups act twice.''

\begin{thm}[Guillemin-Sternberg, etc.]
	The geometric quantization of a symplectic quotient satisfies
	\[
		\Hc(M \sslash G) = \Hc(M)^G,
	\]
	where the right hand side is the subspace of $G$-invariant vectors in $G$.
\end{thm}

We can also define the symplectic reduction along any coadjoint orbit $\Oc_\alpha$ as $M \sslash_\alpha G = \mu\inv(\Oc_\alpha) / G$.
This gives a decomposition of $M$ as 
\[
	M = \cup_{\alpha \in \mu(M)} \mu\inv(\Oc_\alpha) = \cup_{\alpha \in \mu(M)} (G\textrm{-bundles over } M \sslash_\alpha G),
\]
at least if we avoid critical points.

Elliot has some fancy art of this decomposition.

Let's focus on the simplest possible case:

\begin{dfn}
	A Hamiltonian $G$-space $M$ is \emph{multiplicity-free} if $\dim M \sslash_\alpha G = 0$ for all $\alpha$.
\end{dfn}

\begin{rmk}
	If $M$ is compact, then a Morse theory argument shows that $M \sslash_\alpha G = \pt$ for all $\alpha$.
\end{rmk}

Here are some relevant examples.

\begin{ex}
	For a coadjoint orbit $\Oc_\alpha$, we have $\Oc_\alpha \sslash_\alpha G = \pt$, so coadjoint orbits are multiplicity-free.
	Here we are ignoring stacky / derived quotients even though the action is typically nonfree.
\end{ex}

\begin{ex}
	Consider $\PP^1$ with $\U(1)$ acting by rotation.
	Then $\mu$ is the height function on $\PP^1 = S^2$.
	If the top height is $1$ and the bottom height is $-1$, then $\mu\inv(1)$ and $\mu\inv(1)$ are both points.
	For any $x \in (-1, 1)$, we have $\mu\inv(x) = S^1$ and therefore $\PP^1 \sslash_x \U(1) = \pt$.
	Thus this action is multiplicity-free.
\end{ex}

\begin{ex}
	Let $\U(1)^2$ acts on $\PP^2$ (extending the standard action on $\AA^2 \subset \PP^2$).
	The fibers of the moment map over points in the interior of $\mu(M)$ are $2$-tori, which degenerate to circles on the boundary lines of $\mu(M)$ and points at the corners of $\mu(M)$.
\end{ex}

A non-example is given by the $\U(1)$ action on $\CC^2$ from earlier in the lecture.
This is an obvious non-example because the dimension of the symplectic quotient is nonzero.
The slogan is that ``multiplicity-free manifolds have maximal symmetry.''

\subsection{(David) -- Interlude}

For a Lie group $G$, we have $T^*G = G \times \gf^*$.
Consider $G \curvearrowright T^*G$ induced by the adjoint action of $G$ on itself.
We obtain a moment map $\mu: G \times \gf^* \to \gf^*$ given by the formula
\[
	\mu(g, x) = \Ad_g(x) - x.
\]
Then $\mu\inv(0) = \{ (g, x) \in G \times \gf^* \, | \, g \in G_x \}$, where $G_x$ is the centralizer of $x \in G$.

The multiplicity-freeness property for a general Hamiltonian $G$-space $M$ can be understood as the requirement that the centralizers $G_x$ act transitively on the preimages $\mu\inv(x)$.

It is a good exercise to classify multiplicity-free Hamiltonian $G$-spaces for $G = \U(1)$ or $G = \SU(2)$.

\subsection{(Elliot) -- A few last words}

Multiplicity-freeness has a useful consequence for quantization: if $M$ is multiplicity-free, then each highest weight representation $E_\alpha$ appears in $\Hc(M)$ at most once.
In fact, $E_\alpha$ will appear if and only if $\Oc_\alpha \in \mu(M)$.

We will be interested in hyperspherical varieties as a large family of multiplicity-free symplectic manifolds.
More on that next time!

\section{9/21 (Mark Macerato) -- Hyperspherical Varieties}

\subsection{(David) -- Multiplicity-freeness}

There may have been minor errors in the discussion last time, but the basic ideas were right.
Suppose for simplicity that $T$ is an \emph{abelian} Lie group, and consider the cotangent bundle $T^*T \cong T \times \tf^*$.
The moment map for the translation action of $T$ on itself is the projection $T \times \tf^* \to \tf^*$.
This gives a (trivial) family of abelian groups over $\tf^*$.

If we have another Hamiltonian $T$-space $X$, we obtain a moment map $\mu_X: X \to \tf^*$.
We can view our family of abelian groups over $\tf^*$ as acting fiberwise on $X$.
The multiplicity-freeness condition is requiring that the orbits of this action are fiberwise discrete.

This story still works for non-abelian $G$ (but you have to be careful about left versus right actions).
In this case, the fiber over $v \in \gf^*$ will be given by the stabilizer $G_v$.

\begin{ex}
	We can describe Hamiltonian $\U(1)$-spaces as lying over $\uf(1) \cong \RR$.
	The multiplicity-freeness condition implies that the fibers are (disjoint unions of) copies of $S^1$ and points.
	For example, we can consider the height function on the sphere, or the projection of a cylinder $S^1 \times \RR$, or many related examples -- these all give multiplicity-free Hamiltonian $\U(1)$-spaces.
\end{ex}

\begin{ex}
	If we take $G = \SU(2)$, we obtain a similar (but distinct) picture because $\suf(2) / \SU(2) \cong [0, \infty)$ (the $\SU(2)$-orbits in $\suf(2)$ are spheres).
	The fibers of $T^*\SU(2) \to \suf(2)$ are $\SU(2)$ (over $0$) and $S^1$ (over points in $(0, \infty)$).
	We can analyze multiplicity-free Hamiltonian $G$-spaces as before.
\end{ex}

In general, the left action $G \curvearrowright T^*G$ (via $g \cdot (h, v) = (gh, \Ad_g v)$) is not multiplicity-free.
Consider the moment map $T^*G \cong G \times \gf^* \to \gf^*$ given by projection (this depends on how we trivialize $T^*G$).
For a coadjoint orbit $\Oc$, the preimage $\mu\inv(\Oc)$ is $G \times \Oc$.
The multiplicity-freeness here reduces to the question of whether the action $G_v \curvearrowright G$ has discrete orbits.
This is not true in general (see e.g.\ the $\SU(2)$ example above), proving the claim.

\emph{A later clarification:} 
Really, we should think of $T^*G \rightrightarrows \gf^*$ as a groupoid, where the ``source'' and ``target'' maps are $\mu_L$ and $\mu_R$ (the moment maps for the left / right actions, respectively).
Given a groupoid, we can obtain a group scheme (encapsulating the ``automorphism groups of points'') as a fiber product, e.g.\
\[
	\begin{tikzcd}
		\{ [X, g] = 0 \} \rar \dar & T^* G \dar \\
		\Delta \rar & \gf^* \times \gf^*.
	\end{tikzcd}
\]
Understanding things from this perspective clears up the difficulties with left / right actions.

Hamiltonian $G$-spaces ($M \to \gf^*$) will be module objects for this groupoid.

\subsection{(Mark) -- Towards hyperspherical varieties}

We will change settings to algebraic geometry (following section 3 of Ben-Zvi--Sakellaridis-Venkatesh).
Fix an algebraically closed field $k$ of characteristic zero (e.g.\ $\CC$ or $\ol{\QQ_\ell}$).
Let $G$ be a connected reductive group over $k$.

Recall that a spherical variety is a normal $G$-variety $X$ such that there exists a Borel subgroup $B \subset G$ with an open orbit in $X$.
We can rephrase the last condition without picking a Borel: we require that $G$ has an open orbit on $X \times \Fl_G$.
If $X$ is affine, this is equivalent to requiring that the coordinate ring $k[X]$ is multiplicity-free as a $G$-module.

\begin{ex}[``Group case'']
	Let $H$ be a connected reductive group and $G = H \times H$.
	For $X = H$ and $G \hookrightarrow X$ via $(h_1, h_2) \cdot h = h_1 h h_2\inv$, $H$ is a spherical variety.

	If we fix a Borel $B \subset H$, we have a unipotent subgroup $U \subset B$ and a surjection $B \twoheadrightarrow T = B/U$.
	By Levi's theorem, this splits, giving $T \hookrightarrow B \subset G$.
	We get a vector space decomposition $\gf = \uf^- \oplus \tf \oplus \uf$.
	Consider the open embedding $U^- \times B \to H$ given by $(u, b) \mapsto ub$.
	The Borel subgroup $B^- \times B \subset G$ has an open orbit in $H$.
	This leads to a Bruhat decomposition $H = \sqcup_{w \in W} B w B$.
\end{ex}

We can obtain Bruhat decompositions for more general spherical varieties.
This is a rich theory that has been worked out by several authors (Knapp, Brion, etc.).
But let's move on to hyperspherical varieties, which give a symplectic point of view.

Instead of a spherical variety $X$, let us consider $M = T^*X$ with the moment map $\mu: T^*X \to M$.
For simplicity, we will assume our base spherical variety $X$ is affine, smooth, and irreducible.
In this case $M$ is \emph{coisotropic}, which means that the $G$-invariant function field $k(M)^G$ is Poisson-commutative.

Another way of saying this is as follows.
Let $\cf = \gf^* \sslash G \cong \gf \sslash G$ be the ``Chevalley space.''
Letting $\eta \in M$ be the generic point, we obtain a Stein factorization $M \to \cf_M \to \cf$.
The map $\tilde{\mu}: M \to \cf_M$ has connected generic fiber, and $\cf_M \to \cf$ is finite.
The second definition of ``coisotropic'' is that the group $G_{K(\cf_M)}$ acts on $M_{K(\cf_M)}$ with an open (hence dense) orbit.

\begin{thm}[Losev]
	If $M$ is a smooth Hamiltonian $G$-variety, then \emph{all} of the fibers of $\tilde{\mu}: M \to \cf_M$ are connected.\footnote{This is the closest analogue in algebraic geometry of the connectedness theorem of Atiyah-Guillemin-Sternberg.}
\end{thm}

A third definition of coisotropic is that the generic $G$-orbit on $M$ is coisotropic in the usual sense.

``Coisotropic'' is the algebraic geometry version of ``multiplicity-free.''
Elliot gave a discussion of why this recovers the earlier condition in symplectic geometry, but it was a bit too fast to type up.

\section{9/28 (Mark Macerato) -- Continued}

\subsection{(David) -- Groupoids and Hamiltonian $G$-spaces}

Recall the homework problem of classifying multiplicity-free $\SU(2)$-spaces.

The corrected general picture is as follows.
Consider the cotangent bundle $T^*G$ with natural Hamiltonian $G$-actions on the left and right.
These yield moment maps $\mu_L, \mu_R: T^*G \to \gf^*$.
If we trivialize $T^*G \cong G \times \gf^*$, these maps are given by $(g, X) \mapsto X$ and $(g, X) \mapsto \Ad_g X$.

We should think of $T^*G \rightrightarrows \gf^*$ as a groupoid.
The ``objects'' are $X \in G$, and the ``morphisms'' are $g: X \to \Ad_g X$.
Composition is given by group multiplication.

We may view any Hamiltonian $G$-space $Y$ (with moment map $\mu: Y \to \gf^*$) as a module over this groupoid.
Specifically, we have a natural map $T^*G \times_{\gf^*} Y \to Y$, the projection of the fiber product onto the second factor.
On elements, this is given by $(g, X, y) \mapsto gy$, which lies in the fiber of $Y$ over $\Ad_g X \in \gf^*$.

Consider the pullback
\[
	\begin{tikzcd}
		\Sc \rar \dar & T^*G \dar \\
		\gf^* \rar["\Delta"] & \gf^* \times \gf^*.
	\end{tikzcd}
\]
In equation, $\Sc = \{ [g, X] = 0 \}$.
From the groupoid perspective, $\Sc \to \gf^*$ is obtained by only considering automorphisms of objects in our original groupoid (i.e.\ forgetting about isomorphisms between different objects).
We can view $\Sc \to \gf^*$ as the relative group over $\gf^*$ with fibers given by stabilizers $\Stab_G(X)$.

The ``multiplicity-free'' condition can now be restated: it means that the $\Sc$-action on $Y$ relative to $\gf$ has only finitely many orbits.

For the exercise about $\SU(2)$, we have $\gf^* = \RR^3$, and $\Sc$ has fiber $\SU(2)$ over the identity and $\U(1)$ over other fibers.
We really only care about $\gf^* / \SU(2)$, which looks like a real ray $[0, \infty)$.
This allows us to produce some examples of multiplicity-free Hamiltonian $\SU(2)$-spaces - these spaces should have maps to $[0, \infty)$ with fibers over $X \in \gf^* / \SU(2) \cong [0, \infty)$ looking like (finite disjoint unions of) orbits of $\Stab_{\SU(2)}(X)$-actions.

\begin{ex}
	The $2$-sphere $S^2$ has multiplicity-free $\SU(2)$-action via the action coming from $\SU(2) \to \SO(3)$.
\end{ex}

\begin{ex}
	The standard representation $\CC^2$ has multiplicity-free $\SU(2)$-action.
\end{ex}

\begin{ex}
	The blowup of $\CC^2$ at the origin (with a corrected symplectic form) has multiplicity-free $\SU(2)$-action.
\end{ex}

Are these all of the possible examples (up to finite covers)?
It would be good to figure this out.

\subsection{(Mark) -- Coisotropic $G$-varieties}

Recall our setup: $G$ is connected and reductive, and $M$ is a smooth affine Hamiltonian $G$-variety.
We have a moment map $\mu: M \to \gf^*$, and we can compose this with a GIT quotient map to get $\ol{\mu}: M \to \cf_G$, where $\cf_G = \gf^* \sslash G$ is called the Chevalley base.
This admits a ``Knop factorization''
\[
\begin{tikzcd}
	M \ar[rr, "\ol{\mu}"] \ar[dr, "\tilde{\mu}_M"] & & \cf_G \\
	& \cf_M \ar[ur, "\pi"] &
\end{tikzcd}
\]
where $\pi$ is finite and $\tilde{\mu}_M$ has generically connected fiber.

\begin{dfn}
	We say that $M$ is \emph{coisotropic} if any of the following equivalent conditions hold.
	\begin{enumerate}
		\item $k(M)^G$ is Poisson-commutative.\footnote{In this setup, we can replace this by the condition that $k[M]^G$ is Poisson commutative, since $\mathrm{Frac} k[M]^G = k(M)^G$.}
		\item The generic orbit of $G$ on $M$ is coisotropic.
		\item The generic fiber of $\tilde{\mu}_M$ has a dense $G$-orbit.
	\end{enumerate}
\end{dfn}

Let's see why $1$ and $2$ are equivalent.
Choose $f_1, \dots, f_n \in K(M)$ which separate generic orbits (this is possible by a theorem of Rosenlicht).
This yields $\ul{f} = (f_1, \dots, f_n): U \to \AA^n$ (for $U \subset M$ open), and we can restrict this to a surjective smooth map $U' \to W$  such that $U'$ is dense in $U$ and $W \subset \AA^n$ is a locally closed subvariety.
Replace $U$ by $U'$.
The fibers of $\ul{f}$ are exactly the $G$-orbits in $U$.
Therefore, for $x \in U$, we see that $df_1(x), \dots, df_n(x)$ span the conormal space $T^*_U(G \cdot x)_x$.
Thus $G \cdot x$ is coisotropic at $x$ if and only if $T_U^*(G \cdot x)_x$ is isotropic, if and only if the $f_1, \dots, f_n$ Poisson-commute at $x$.

\subsection{Approaching hyperspherical varieties}

Suppose that $M$ is a smooth affine Hamiltonian $G$-variety as before.
We will also require that $M$ comes with a $\GG_m$-action (equivalently, a grading on $k[M]$) such that
\begin{enumerate}
	\item The $\GG_m$-action on $M$ commutes with the $G$-action.
	\item The symplectic form $\omega$ on $M$ has weight $2$, i.e.\ $\lambda \cdot \omega = \lambda^2 \omega$.
\end{enumerate}
David noted that this latter condition implies that $\omega$ is exact: if $v$ is the vector field generating the $\GG_m$-action, then Cartan's magic formula (using that $\omega$ is closed) gives
\[
	2 \omega = \Lc_v \omega = d(i_v \omega).
\]
The $2$ here is needed to ensure that we can construct a ``$\GG_m$-equivariant Kostant slice.''

We want to define what it means for $M$ to be hyperspherical.
The first condition will be that $M$ is coisotropic.

The second condition is that $\mu(M) \subset \gf^*$ meets the nilpotent cone $\Nc_G = \chi\inv(0)$ (for $\chi: \gf^* \twoheadrightarrow \gf^* \sslash G$).
Equivalently, $\ol{\mu})(M)$ contains $0 \in \cf_G$.
This implies that $M \sslash G \to \cf_M$ is surjective (it is always an open immersion, so we get $M \sslash G = \cf_M$).
There will be two more conditions (which we will discuss next time).

\section{10/5 (Mark Macerato) -- Continued}

\subsection{Pre-hyperspherical varieties}

Let $G$ be a connected reductive group and $G_{\gr} = G \times \GG_m$.
We consider a smooth affine Hamiltonian $G$-variety with auxiliary $\GG_m$-action governing the grading.
This yields a map $M \to \gf^* \to \cf_G$, which has a Knop factorization $M \to \cf_M \to \cf_G$.
Here $\GG_m \curvearrowright \gf^*$ quadratically, and the map $M \to \gf^*$ is $\GG_m$-equivariant.

\begin{dfn}
	We say that $M$ is \emph{pre-hyperspherical} if
	\begin{enumerate}
		\item $M$ is coisotropic, i.e.\ $k(M)^G$ is Poisson commutative (equivalently, the generic fiber of $M \to \cf_M$ has a dense $G$-orbit),
		\item $\mu(M) \cap \Nc_G \neq \emptyset$ (for $\Nc_G$ the nilpotent cone of $G$), and
		\item The stabilizer of a generic point of $M$ is connected.
	\end{enumerate}
\end{dfn}

\begin{ex}
	Let $G = \Sp_{2n}$ and $M = \CC^{2n} \oplus \CC^{2n}$.
	Here $\mu_M: \CC^{2n} \oplus \CC^{2n} \to (\CC^{2n} \oplus \CC^{2n}) \sslash \Sp_{2n} \cong \AA^1$ via $(v, w) \mapsto \omega(v, w)$.
	Thus
	\[
		\mu_M\inv(1) = \{ (v, w) \in \CC^{2n} \, | \, \omega(v, w) = 1 \},
	\]
	and $\Sp_{2n}$ acts transitively on this fiber.
	Meanwhile, $\mu\inv(0)$ can be decomposed as:
	\[
		\mu\inv(0) = \{ (v, w) \, | \, v, w \textrm{ lin.\ ind.\, } \omega(v, w) = 0 \} \cup \{ (v, w) \, | \, v, w \textrm{lin. dep., not both } 0 \} \cup \{ (0, 0) \}.
	\]
	The first set here is the unique open orbit, and the last set is the unique closed orbit.
	The middle set contains a $\PP^1$ worth of orbits.
	In particular, $\mu(M)$ meets $\Nc_G$.
	The stabilizer of a generic point of $M$ can be identified with $\Sp_{2n-2}$.
\end{ex}

\begin{prop}
	In general, if $M$ is pre-hyperspherical, there exists a unique closed orbit $M_0 \subset M$ for $G_{\gr} = G \times \GG_m$.
\end{prop}

We call $M_0$ the \emph{core} of $M$.

\begin{proof}
	Consider the GIT quotient $M \to M \sslash G_{\gr}$, and recall that closed orbits of $G_{\gr}$ correspond to points of $M \sslash G_{\gr}$.
	Thus it suffices to show that $M \sslash G_{\gr} = \pt$, or equivalently $k[M]^{G \times \GG_m} = k$.
	Note
	\[
		k[M]^{G \times \GG_m} = k[\cf_M]^{\GG_m} = k[\cf_m]_0,
	\]
	the weight $0$ component.
	By construction, $k[\cf_G] \to k[\cf_M]$ is finite (and therefore integral), so $k[\cf_M]_0 \to k[\cf_G]_0$ is integral (an exercise in counting degrees).
	Now we have
	\[
		k[\cf_G]_0 = k[\cf_G]^{\GG_m} = k[\gf^*]^{G \times \GG_m} = k,
	\]
	so $k[\cf_M]_0$ is integral over $k$.
	Since $k$ is algebraically closed, we see $k[\cf_M]_0 = k$.
\end{proof}

\subsection{(David) -- Weinstein manifolds}

Suppose we have an exact symplectic manifold $(M, \omega = d\lambda)$.
Take $Z = \omega\inv(\lambda)$ (this is a vector field on $M$).
Then $Z$ gives a flow on $M$, and the core $M_0$ is the subset of points of $M$ which do not escape to infinity along this flow.
Assuming $M_0$ is isotropic (this is implied by the Weinstein condition), the flow gives an action of $\CC^\times$ on $M_0$.

\begin{ex}
	Consider the surface singularity $x^2 + y^2 + z^{n+1} = 0$.
	Let $M$ be a symplectic resolution of this.
	The core $M_0$ is the chain of $\PP^1$'s appearing as the zero fiber.
	In terms of geometric representation theory, we can call $M_0$ a ``subregular Springer fiber.''
\end{ex}

Suppose $M$ is $G$-Hamiltonian -- then we are in a situation very similar to what Mark is talking about.
That is, \emph{pre-hyperspherical varieties are analogous to $G$-Hamiltonian Weinstein manifolds}.
This precludes examples like the above, since the union of $\PP^1$'s cannot be a single $G$-orbit.
This fact is essentially kin to the last Proposition.

\begin{ex}
	Consider $G \times G$ acting on $M = T^* G$ by left and right translation.
	The core $M_0$ is the zero section.
\end{ex}

\subsection{(Mark) -- Hyperspherical varieties}

Let $\mu_M: M \to M \sslash G$ be the GIT quotient map.

\begin{prop}
	The core $M_0$ is the unique closed $G$-orbit in $\mu_M\inv(0)$.
\end{prop}

\begin{proof}
	Note that $G$ has a unique closed orbit $M_0' \subset \mu_M\inv(0)$ (by standard GIT).
	Since $\GG_m$ commutes with $G$, the $\GG_m$ action takes closed $G$-orbits to closed $G$-orbits.
	Therefore $\GG_m$ preserves $M_0'$, and we get $G \times \GG_m \curvearrowright M_0'$.
	It follows that $M_0'$ contains a closed $G_{\gr}$ orbit, hence contains $M_0$.
	But $M_0'$ is itself a $G$-orbit, so $M_0 = M_0'$.
\end{proof}

Pick $x \in M_0$, and let $H = \Stab_G(x)$, so $M_0 = G / H$.
Since $M_0$ is affine, $H$ must be reductive.
Since $\mu_M\inv(0)$ maps to $\Nc_G \subset \gf^*$, we get an element $f = \mu(x) \in \Nc_G$.

Because $\GG_m \curvearrowright M_0 = G / H$, we get a homomorphism $\GG_M \to \Aut_G(G / H) \cong N_G(H) / H$.
In fact, this factors through $(Z_G(H) / Z(H))^0$ (where ${}^0$ denotes the connected component of the identity element).
Let $\ol{\pi}: \GG_M \to Z_G(H) / Z(H)$ be the induced map.

\begin{dfn}
	A pre-hyperspherical variety $M$ is hyperspherical if
	\begin{enumerate}
		\item $\ol{\pi}$ lifts to a homomorphism $\pi: \GG_m \to Z_G(H) \subset G$, which moreover lifts to a homomorphism $\rho: \SL_2 \to G$ such that
			\[
				d\rho \begin{pmatrix}
					0 & 1 \\
					0 & 0
				\end{pmatrix} = f
			\]
			under the standard identification $\gf \cong \gf^*$.
		\item Consider the \emph{sheared $\GG_m$-action} $(\GG_m)_{sh} \curvearrowright M$ induced by
			\begin{align*}
				(\GG_m)_{sh} &\to G \times \GG_m \\
				g &\mapsto (\pi(g)\inv, g).
			\end{align*}
			By construction, $(\GG_m)_{sh}$ fixes $x$, and thus we get $(\GG_m)_{sh} \curvearrowright (T_x M_0)^\perp / (T_x M_0 \cap T_x M_0^\perp) := N_x M_0$, the \emph{symplectic normal space}.
			The condition is that this $(\GG_m)_{sh}$-action on $N_x M_0$ is given by linear scaling.
	\end{enumerate}
\end{dfn}

\section{10/12 (Mark Macerato) -- Continued}

\subsection{Refresher on the definition}

Recall that $M$ is a smooth affine graded Hamiltonian $G$-variety with moment map $\mu: M \to \gf^*$.
The grading means that we have a $\GG_m$ action (acting on $\omega_M$ with weight $2$) such that the $G$ and $\GG_m$ actions commute.
Therefore $\mu$ is $\GG_m$-equivariant, where $\GG_m$ acts on $\gf^*$ with weight $2$.

Recall that we say $M$ is pre-hyperspherical if:
\begin{enumerate}
	\item $M$ is coisotropic (i.e.\ $k(M)^G$ is commutative, equivalently generic $G$-orbits are cut out by Poisson-commuting $G$-invariant functions).
	\item The image of $\mu$ meets the nilpotent cone $\Nc_G^*$.
		Thus there exists a unique closed $G \times \GG_m$-orbit in $M$, the ``core'' $M_0$ of $M$.
		This is the unique closed $G$-orbit in $\tilde{\mu}_M\inv(0)$, where $\tilde{\mu}_M: M \to \cf_M = M \sslash G$ and we write $0$ for an element of $\cf_M$ mapping to $0 \in \cf_G = \gf^* \sslash G$.)
	\item The stabilizer of a generic point $m \in M$ is connected.
\end{enumerate}

We now move on to the full hyperspherical condition.
Fix $x \in M_0$, and let $f = \mu(x) \in \Nc_G^*$.
Then $M_0 \cong G / H$ where $H = \Stab_G(x)$, and $H$ is reductive.
Let $\ol{\pi}: \GG_m \to \Aut_G(G/H) \cong N_G(H) / H$ be the natural map.

Condition (4a) is that $\ol{\pi}$ lifts to a homomorphism $\pi: \GG_m \to N_G(H) \cap [G, G]$.

\begin{prop}
	The lift $\pi$ factors through $Z_G(H)$.
\end{prop}

\begin{proof}
	Let $t = d\pi(1) \in \gf$, and let $\hf = \Lie(H)$.
	We need to show that $t$ centralizes $\hf$.
	Identify $\gf \cong \gf^*$, so we can view $f$ as an element of $\gf$.
	Then $[t, f] = 2f$ (since $\GG_m$ acts on $\gf^*$ with weight $2$, and $\mu: M \to \gf^*$ is $G$-equivariant).
	By the Jacobson-Morozov theorem, there exists $e \in \gf$ such that $[t, e] = -2e$ and $[f, e] = t$, giving an embedding $\slf_2 \hookrightarrow \gf$.
	
	View $\gf$ as an $\slf_2$-module via the above embedding.
	Then $\hf$ is spanned by highest weight vectors in $\gf$, so all of the weights of $t$ on $\hf$ are nonnegative.
	Write $\hf = \hf_+ \oplus \hf_0$, where $\hf_+$ is the sume of the strictly positive $t$-weight spaces.
	We have $\hf_+ \subset \im (\ad(f)) \cap \gf^f$ (since $H$ fixes $x$, we see that $H$ fixes $f = \mu(x)$).

	Note that $\im (\ad(f)) \cap \gf^f$ is an ideal in $\gf^f$ (by a direct Lie algebra computation), so $\im(\ad(f)) \cap \hf$ is an ideal of $\hf$.
	Furthermore, $\im (\ad(f)) \cap \gf^f$ is a nilpotent Lie algebra (since contained in the positive weight part of $\gf$).
	Thus $\im (\ad(f)) \cap \hf$ is a nilpotent ideal in $\hf$, but $\hf$ is reductive, so $\im (\ad(f)) \cap \hf$ must be central.
	Since $\im (\ad(f)) \cap \hf \subset [\hf, \hf]$, it follows that $\im(\ad(f)) \cap \hf = 0$, and thus $\hf_+ = 0$.
	We see that $\hf = \hf_0$, so in particular $x$ centralizes $\hf$.
\end{proof}

\begin{prop}
	The lift $\pi$ is unique if it exists.
\end{prop}

To state condition (4b) for hyperspherical varieties, let $\tilde{\GG}_m$ be $\GG_m$ acting on $M$ via the sheared action $\GG_m \xrightarrow{(\pi(-)\inv, \id)} G \times \GG_m$.
Thus $\tilde{\GG}_m$ acts on the symplectic normal space $T_x M_0^\perp / T_x M_0$, where $T_x M_0^\perp$ is the symplectic orthogonal subspace.
Condition (4b) is that $\tilde{\GG}_m$ acts on the symplectic normal space by $t \cdot v = tv$ (i.e.\ the action has weight $1$).
This is a useful condition, though the geometric interpretation is not clear.

We are interested in considering sheared $\GG_m$ actions more generally -- this turns out to be the natural thing to do from the perspective of geometric Satake.

\subsection{(David) -- More on shearing}

Consider the vector field $2 p \partial_p - q \partial_q$ on $\RR^2 = T^*\RR$.
This gives a hyperbolic $\GG_m$-action and corresponds to the Liouville form $\lambda = 2 p dq + q dp$.
For $H = pq$, we get an action of $G = \GG_m$ via the Hamiltonian vector field $X_H = p \partial_p - q \partial_q$.
Subtracting $X_H$ from our Liouville vector field, we get a $\tilde{\GG}_m$-action via the vector field $p \partial_p$.
This $\tilde{\GG}_m$-action preserves the zero section!
Thus, by considering shearing actions, we can preserve certain desirable isotropic / Lagrangian submanifolds.

\subsection{(Mark) -- A construction}

Let $H \subset G$ be a reductive group and $H \curvearrowright S$ be a symplectic representation of $H$.
Let $\pi: \GG_m \to [G, G] \cap Z_G(H)$.
Choose a nilpotent element $f \in \gf^*$.

We equip $S$ with a commuting $\GG_m$ action via scaling (of weight $1$).
Write $\gf = \jf \oplus \uf^- \oplus \gf_0 \oplus \uf$, where $\jf$ is the centralizer of $\pi$ and $f$, $\uf^-$ is the subspace with negative $\pi$ eigenvalues, $\gf_0$ is the subspace with zero $\pi$ eigenvalues (but nonzero $f$ eigenvalues), and $\uf$ is the subspace with positive $\pi$ eigenvalues.
Integrate $\uf$ to a unipotent subgroup $U \subset G$.

... And we're out of time -- we will finish next week.

\section{10/19 (Mark Macerato) -- Concluded}

The goal for today is to discuss the Ben-Zvi--Sakellaridis--Venkatesh construction of hyperspherical varieties via Whittaker induction.
This will be a complicated story, but it covers several examples of interest.

\subsection{Hamiltonian induction}

Let $G$ be an algebraic group and $H$ a subgroup.
We can view every $G$-variety as an $H$-variety by forgetting some of the action; this gives a restriction functor $\Res_H^G$ from $G$-varieties to $H$-varieties.
More interestingly, this functor has a left adjoint, ``induction'' $\Ind_H^G$, sending an $H$-variety to a $G$-variety.
Specifically, for a $G$-variety $Y$, we have a diagonal $H$-action on $G \times Y$ via $h \cdot (g, y) = (g h\inv, h y)$.
The induction is the balanced product $G \times^H Y := (G \times Y) / H$.

Every $G$-variety $X$ gives a Hamiltonian $G$-variety $T^* X$ (and similarly for $H$-varieties).
The Hamiltonian condition here appears as follows: if $D(X)$ is differential operators on $X$, then we get a natural map $\Uc \gf \to D(X)$.
Taking associated gradeds, we get a map $\Sym \gf \to \Oc(T^* X)$.
This corresponds (under $\Spec$) to the moment map $T^* X \to \gf^*$.

Given a Hamiltonian $G$-variety, we can restrict to the $H$-action and get a Hamiltonian $H$-variety.
This defines a ``functor'' $h\Res_H^G$ from Hamiltonian $G$-varieties to Hamiltonian $H$-varieties.
The term ``functor'' here is in quotes because the correct notion of a morphism between symplectic varieties is not the standard notion of morphism of varieties.
Instead, we need to consider ($G$-stable) Lagrangian correspondences as our morphisms.

To understand why, suppose we have a morphism $f: X \to Y$ of varieties.
We do not obtain a natural morphism of varieties between $T^*X$ and $T^*Y$.
Instead, we get a Lagrangian correspondence
\[
	T^*X \xleftarrow{f^*} T^*Y \times_Y X \xrightarrow{\pi_1} T^*Y.
\]
We can view $T^*Y \times_Y X$ as the conormal variety to $\Gamma_f \subset T^*X \times T^*Y$.
Morally, we should think of Lagrangians as being the symplectic analogue of points (they give the maximum amount of information we can cut out by Poisson-commuting functions).

The Hamiltonian induction functor acts on cotangent bundles by
\[
	h\Ind_H^G T^*Y = T^*(G \times^H Y) = T^*(G \times Y) \sslash H = (T^*G \times T^*Y) \sslash H.
\]
From this, we can understand what to do for a general Hamiltonian $H$-variety $M$.
We define
\[
	h\Ind_H^G(M) = (T^*G \times M) \sslash H,
\]
where we use the diagonal $H$-action (from $h \mapsto (h\inv, h) \in H \times H$).
Then $G$-stable Lagrangian correspondences between $h\Ind_H^G Y$ and $M$ correspond to $H$-stable Lagrangian correspondences between $Y$ and $h\Res_H^G M$ (i.e.\ $h\Ind_H^G$ is left adjoint to $h\Res_H^G$).

\subsection{(David) -- The groupoid picture}

Recall that we should think of Hamiltonian $H$-spaces $M \to \hf^*$ as coming with an action of the groupoid $T^*H \rightrightarrows \hf^*$.
To construct the induced Hamiltonian $G$-space, we consider $T^*G$ with its natural maps to $\hf^* \times \hf^*$, and we combine this with $M$ as Mark was indicating above.

\subsection{(Mark) -- Sheared Hamiltonian $G$-spaces}

Recall that we started by considering \emph{graded} Hamiltonian $G$-variety i.e.\ those equipped with a commuting $\GG_m$-action (of weight $2$ on $\omega$).
To discuss Whittaker induction, it is better to generalize to \emph{sheared} Hamiltonian $G$-variety.

\begin{dfn}
	Let $\pi: \GG_m \to \Aut(G)$.
	A \emph{$\pi$-sheared Hamiltonian $G$-variety} is a Hamiltonian $G$-variety $M$ with a $\GG_m$-action such that, for $\lambda \in \GG_m$, $g \in G$, and $x \in M$, we have
	\[
		\lambda \cdot (g \cdot x) = \pi(\lambda)(g)(\lambda \cdot x)
	\]
	and such that $\mu: M \to \gf^*$ is $\GG_m$-equivariant, where $\GG_m$ acts on $\gf^*$ via 
	\[
		\lambda \cdot \xi = \lambda^2 \pi(\lambda)^t (\xi).
	\]
\end{dfn}

\begin{ex}
	Consider the action of $\GG_a$ on $\pt$, where we assume $\mu: \pt \to \AA^1$ sends $\pt$ to $x \in \AA^1$.
	We can consider the action $\pi: \GG_m \curvearrowright \GG_a$ by $\lambda \cdot x = \lambda^2 x$.
	Then $\pt$ is a $\pi$-sheared Hamiltonian $\GG_a$-variety, but is not a graded Hamiltonian $G$-variety.
\end{ex}

However, suppose that $\pi$ arises from an inner automorphism, i.e.\ we have $\pi: \GG_m \to G$ and the action is $\lambda \cdot g = \pi(\lambda) g \pi(\lambda)\inv$.
Then we have an equivalence of categories between graded Hamiltonian $G$-varieties and $\pi$-sheared Hamiltonian $G$-varieties, given by sending a graded Hamiltonian $G$-variety $\GG_m \curvearrowright M$ to the sheared variety with $\GG_m$-action given by $(\pi(-)\inv, \id): \GG_m \hookrightarrow G \times \GG_m$.

\begin{ex}
	Let $\GG_m \curvearrowright \gf^*$ and consider $2\rho: \GG_m G$.
	This gives the sheared action $\lambda \cdot \xi = 2\rho(-\lambda) \lambda^2 \xi$.
\end{ex}

\subsection{Whittaker induction}

Suppose $G$ is connected and reductive, and say we have $\pi: \GG_m \to [G, G]$.
Let $f \in \gf^*$ be nilpotent and assume $\pi(\lambda) \cdot f = \lambda^{-2} f$.
Consider a reductive subgroup $H \subset G$ such that $H$ commutes with $\pi(\GG_m)$ and $H$ fixes $f$.
We can write
\[
	\gf = \jf \oplus \gf_+ \oplus \gf_0 \oplus \gf_-
\]
where $\jf$ is the centralizer of $\Lie(\pi(\GG_m))$ and $f$, and $\gf_{+/0/-}$ collects the other $\pi$-eigenspaces according to their eigenvalues (whether positive, zero, or negative).
Another way to construct this is using the Jacobson-Morozov theorem as discussed last time.

Let $\uf = \gf_+$ and $\uf_+ = \oplus_{i \geq 1} \gf_i$ (the sum of eigenspaces of $\pi$-weight strictly greater than one).\footnote{For ease of reading, one is recommended to focus on examples where $\uf = \uf_+$, but $\uf / \uf_+$ is a point with a nontrivial moment map.}
Then $\uf$ and $\uf_+$ integrate to subgroups $U_+ \subset U \subset G$.

Whittaker induction sends graded Hamiltonian $H$-varieties to a graded Hamiltonian $G$-varieties, (but uses shearing and unshearing with respect to $\pi$ in the process).
Note that $H \cap U$ is trivial (since our hypotheses imply $\hf \subset \jf$), $H$ normalizes $U$, and $HU \subset G$.\footnote{I had to leave at this point -- many thanks to Yuji Okitani for graciously sharing his notes.}

\begin{ex}
	Let $G = \GL_n$, $\pi: \GG_m \xrightarrow{2\rho} \GL_n$, and let $H$ be trivial.
	Let
	\[
		f = \begin{bmatrix}
			0 & & & \\
			1 & 0 & & \\
			  & \ddots & \ddots & \\
			  & & 1 & 0
		\end{bmatrix}.
	\]
	Then
	\[
		U = \begin{bmatrix}
			1 & * & \dots & * \\
			  & 1 & \dots & * \\
			  & & \ddots & * \\
			  & & & 1
		\end{bmatrix}
	\]
	and $HU = U$.
	In this case,
	\begin{align*}
		\WhitInd_{H,2\rho,f}^{\GL_n} &= h\Ind_{HU}^G(\pt \times (\uf / \uf_+)_f) \\
		&= h\Ind_U^{\GL_n}(\pt_f) \\
		&= T^* \GL_n \sslash_f U.
	\end{align*}
\end{ex}

Suppose $M$ is a hyperspherical variety.
Then $M = \WhitInd_{H,\pi,f}^G S$, where $H$ is the stabilizer of the unique closed $M_0$ and $S$ is the fiber of the symplectic normal bundle over any $x \in M_0$ (and $\pi$ and $f$ are as in the discussion from last time?). 

\begin{ex}
	Let $M = T^*X$ where $X = G/H$.
	Taking $S = \pt$, $\pi$ the trivial homomorphism, and $f = 0$, we obtain $M$ via Whittaker induction.
\end{ex}

\begin{ex}
	Let $G = \GL_{2n}$, $H = \Sp_{2n}$, and $V = \CC^{2n} \oplus (\CC^{2n})^*$.
	Then
	\[
		\WhitInd_{H,1,0}^G(V) \simeq T^*((\GL_{2n} \times \CC^{2n}) / \Sp_{2n}),
	\]
	and this contains as an open dense subset
	\[
		\WhitInd_{1,2\rho,\psi}^G(\pt) \simeq T^*(\GL_{2n} \sslash_\psi U)
	\]
	where $\psi: U \to \GG_a$ takes a $2n \times 2n$ matrix in $U$ and sums the entries directly above the diagonal $1$s (except for the $n$th such entry, which is not contained in a diagonal block of the matrix).
\end{ex}

\section{10/26 (Jeremy Taylor) -- Relative Local Langlands and Examples}

The goal of these lectures is to explain the next section of the paper and discuss the geometric Satake theorem.

\subsection{The relative local Langlands conjecture}

The relative local Langlands conjecture is that, given $G \curvearrowright M = T^*X$ with $M$ hyperspherical (in particular affine), there is a dual action $\check{G} \curvearrowright \check{M}$ (with $\check{G}$ the Langlands dual of $G$ and $\check{M}$ some hyperspherical variety) satisfying an implicitly derived equivalence
\[
	\Sh(X_\Kc / G_\Oc) = \QCoh(\check{M} / \check{G}).
\]
Here $\Kc = k((t))$ (so $X_\Kc$ is the loop space of $M$) and $\Oc = k[[t]]$ (so $G_\Oc$ is the arc space of $G$).

\begin{ex}
	For $M = \pt$ and $G$ arbitrary, we have 
	\[
		\check{M} = T^*(\check{G} / (\check{N}, f)) = \check{G} \times (f + \check{\gf}^e),
	\]
	where $\check{N}$ is strictly upper triangular matrices and $f$ is a regular nilpotent lower triangular matrix (an element of $(\check{\nf})^- = \check{\nf}^*$).
	The relative local Langlands conjecture gives
	\[
		\Sh(\pt / G) = \QCoh(f + \check{\gf}^e).
	\]
	Note that $f + \check{\gf}^e = \check{\gf} \sslash \check{G}$.
\end{ex}

Suppose that $M = T^*X$.
We can construct $\check{M}$ from its Whittaker data as follows.
For $\check{H}_X \subset \check{G}$ a reductive subgroup, $\check{H}_X \curvearrowright S_X$ a symplectic representation, and $f \in \check{\gf}$ nilpotent (all described later), we have
\[
	\check{M} = \WhitInd_{\check{H}_X,f}^{\check{G}}(S_X) = \big((S_X \times (\check{\uf} / \check{\uf}_+)_f) \times_{\check{\hf}_X^* \oplus \check{\uf}_f} T^* G\big) / (\check{H}_X \check{U}).
\]
This is analogous to parabolic induction from the classical theory of linear algebraic groups -- it is a process for constructing hyperspherical varieties for $\check{G}$ from hyperspherical varieties for simpler subgroups.
Note that if the $\slf_2$-triple containing $f$ has even weights for its action on $\check{\gf}$, then $\check{\uf} / \check{\uf}_+ = \pt$.

Coweights of $\check{H}_X$ should be spanned by $T$-weights in $k[X]^N$, where $N \subset G$ is the standard nilpotent subgroup.
The highest weights of the representation $S_x$ can be determined by looking at divisors in $X$ which are stable under $G$.
Let $P_X$ be the maximal subgroup of $G$ stabilizing the open $B$-orbit in $X$, and take $f$ to be a principal nilpotent element of the Lie algebra of the dual Levi $\check{L}_X \subset \check{P}_X$.

\begin{ex}
	If $S_X = 0$ and $f = 0$, then $\check{M} = T^*(\check{G} / \check{H})$.
\end{ex}

\subsection{Example 1: $M = \pt$}

\begin{ex}
	Let $M = T^*X = \pt$.
	In this case, we see that coweights of $\check{H}_X$ are zero, and thus $\check{H}_x$ must be $1$.
	The highest weights of $S_x$ must be empty, so $S_X = 0$.
	Here $P_X = G$, so $\check{L}_X = \check{P}_X = \check{G}$, and $f$ is a principal nilpotent element of $\check{G}$.
	The Whittaker induction formula yields 
	\[
		\check{M} = T^*(\check{G} / (\check{N}, f)) = \big((f + \check{n}^\perp) \times \check{G}\big) / \check{N}.
	\]
	There is an isomorphism $(f + \check{n}^\perp) \Leftrightarrow \check{N} \times (f + \check{g}^e)$ given by $\Ad_n(x) \leftrightarrow (n, x)$.
	Thus we may identify $\check{M} = (f + \check{\gf}^e) \times \check{G}$.

	This has commuting $\GG_m$-action given by
	\[
		z \cdot (x, g) = \big(\Ad_{2\rho(z)}(z^2 x), g \cdot 2\rho(z)\big).
	\]
	The core $\check{M}_0$ (i.e.\ the unique closed $\check{G} \times \GG_m$-orbit) is $f \times \check{G}$ (which is isomorphic to $\check{G} / \check{H_X} = \check{G}$).

	The automorphic side of relative local Langlands is 
	\[
		\Sh(M_\Kc / G_\Oc) = \Sh(\pt / G) = \Mod_{\mathrm{Nilp}}(H^*(\pt / G)) = \Mod_{\mathrm{Nilp}}\big((\Sym \gf[-2])^G\big).
	\]
	The spectral side of relative local Langlands is
	\[
		\QCoh(\check{M} / \check{G}) = \QCoh(f + \check{g}^e) = \QCoh(\check{\gf} \sslash \check{G}).
	\]
	We can identify the two, proving relative local Langlands in this case.
\end{ex}

\section{11/2 (Jeremy Taylor) -- Continued}

Recall that if we have a hyperspherical variety $G \curvearrowright M$, there should exist a dual hyperspherical variety $\check{G} \curvearrowright \check{M}$.
The operation $\check{(-)}$ should be an involution.

\subsection{Whittaker data of the dual}

For simplicity, let's assume $M = T^*X$ for some spherical variety $X$.\footnote{For a simple example, one can consider the case where $G$ is a torus, so $X$ is a toric variety.
This recovers known constructions / data in the theory of toric varieties.}
Then $\check{M}$ can be constructed from its Whittaker data.
This Whittaker data consists of:
\begin{itemize}
	\item A reductive subgroup $\check{H}_X \subset \check{G}$
	\item A symplectic representation $\check{H}_X \curvearrowright S_X$
	\item A nilpotent element $f \in \check{\gf}$
\end{itemize}
all satisfying some compatibility conditions.
We can identify the core $\check{M}_0$ with $\check{G} / \check{H}_X$.
The symplectic normal bundle to $\check{M}_0$ is $S_X$.

This data arises from $X$ in the following way.
The group $\check{H}_X$ has coweights $\Lambda_{\check{H}_X} \subset \Lambda_{\check{G}}$ spanned (over $\ZZ$) by the $T$-weights in $k[X]^N$.
Let the \emph{valuation cone} $V$ be the set of $G$-invariant valuations $k(X) \to \QQ \cup \{ \infty \}$; then $V$ embeds in $\check{\Lambda}_{\check{H}_X} \otimes \QQ$ as a fundamental domain for the action of the Weyl group $W_X$.\footnote{We consider valuations because we like to think about compactifying varieties by adding divisors, and each divisor gives a valuation.
This gives the classical construction of projective algebraic curves from their function fields.}
The simple coroots $\Delta_X$ are generators of the dual cone in $\Lambda_{\check{H}_X} \otimes \QQ$, normalized to be primitive in $\ZZ \cdot R$ (where $R$ is the set of roots of $G$).
We still need to find the simple roots.

\begin{prop}
	If $\alpha \in \Delta_X$, then one (or both) of the following is true:
	\begin{enumerate}
		\item $\alpha$ is a root of $G$
		\item $\alpha$ is a sum of two roots of $G$.
	\end{enumerate}
\end{prop}

If $\alpha = \gamma \in R$, then the corresponding simple root $\check{\alpha}$ is $\check{\gamma}$ restricted to the torus of $\check{H}_X$.
If $\alpha = \gamma_1 + \gamma_2$ for $\gamma_i \in R$, then we can choose the $\gamma_i$ to be ``strongly orthogonal,'' and then $\check{\alpha}$ is $\check{\gamma}_1$ restricted to the torus of $\check{H}_X$ (and this is equal to the corresponding restriction of $\check{\gamma}_2$).

The symplectic representation $\check{H}_X \curvearrowright S_X$ has highest weights given by $B$-invariant but not $G$-invariant divisors in $X$.
To construct the nilpotent $f \in \check{\gf}$, let $P_X \subset G$ be the maximal subgroup stabilizing an open $B$-orbit and $L_X$ the corresponding Levi.
Let $f$ be a principal element of $\Lie(\check{L}_X)$, where $\check{L}_X$ is the dual Levi to $L_X$.

\subsection{Example 2: $G = \PGL_2$, $X = \PGL_2 / T$}

We may view $X = \PGL_2 / T = (\PP^1 \times \PP^1) \setminus \Delta$, where the correspondence is given by
\[
	\begin{bmatrix}
		a & b \\
		c & d
	\end{bmatrix} \Leftrightarrow \left(\frac{c}{a}, \frac{d}{b}\right).
\]
One can draw this as a (ruled) quadric in $3$-dimensional space.
The diagonal $\Delta$ corresponds to the divisor of points at infinity.

The coweights are given by
\[
	k[X]^N = \oplus_n L(2n)^N \otimes L(2n)^T = \oplus_n L(2n)_{2n}
\]
(where $L(\lambda)$ is the representation with highest weight $\lambda$), so we have $\Lambda_{\check{H}_X} = \Lambda_{\check{G}}$.
We can describe a generator of this (labeled by a root $\alpha of \check{G}$) as
\[
	f_\alpha = \frac{cd}{ad - bc} \in k[X].
\]
One can check that
\[
	\begin{bmatrix}
		x & z \\
		0 & y
	\end{bmatrix} f_\alpha = \frac{x}{y} f_\alpha.
\]
The valuation cone is spanned by the function giving the order of vanishing at $\Delta$.
Thus $\check{H}_X = \SL_2$.

The highest weights of $S_X$ are given by $B$-invariant but not $G$-invariant divisors.
These are $\{0\} \times (\PP^1 \setminus \{0\})$ and $(\PP^1 \setminus \{0\}) \times \{0\}$.
We can compute $S_X = L(\epsilon_1) \oplus L(-\epsilon_2)$.
Furthermore, $f = 0$.
Thus $\check{M} = (S_X \times T^* \check{G}) \sslash \check{H}_X = S_X$.

David talked a bit about how this contrasts with the picture of $T$ acting on $T^*T$.
In the latter picture, we have two missing divisors ($\{0\}$ and $\{\infty\}$).
The valuation cone is the whole $\check{\Lambda}_{\check{H}_X} \otimes \QQ$, and there is no $\alpha$.

\section{11/9 (Jeremy Taylor) -- Continued}

\subsection{Review of examples}

We have constructed two examples of dual hyperspherical varieties:
\begin{itemize}
	\item $G \curvearrowright \pt$ is dual to $\check{G} \curvearrowright T^*(\check{G} / (\check{N},f)) = \check{G} \times (f + \check{g}^e)$.
	\item $\PGL_2 \curvearrowright T^*(\PGL_2 / T)$ is dual to $\SL_2 \curvearrowright T^* L(1)$.
\end{itemize}
In both examples, we should be able to go backwards.

For the second example, let $X = \PGL_2 / T$, which we can view as a quadric surface.
The space $L(1)$ is not a homogeneous space for $\SL_2$, but we can view it as a closure of the open $\SL_2$-orbit $\SL_2 / N = \AA^2 \setminus \{0\}$.
Note that there is one minimal $\PGL_2$-invariant compactification of $X$, namely adding in a $\PP^1$ at infinity.
By contrast, $L(1)$ has two natural $\SL_2$-invariant divisors in the compactification of $\AA^2 \setminus \{0\}$, namely divisors at zero and infinity.
This is reflected in the corresponding valuation cones: $T^*(\PGL_2 / T)$ has valuation cone pointing off in one direction of $\QQ$, while $T^* L(1)$ has valuation cone pointing off in both directions of $\QQ$.

All of the preceding data can be understood in terms of toric varieties / toric compactifications (where the relevant torus is a maximal torus in the acting reductive group).

Our construction of the dual hyperspherical variety $\check{G} \curvearrowright \check{M}$ to $G \curvearrowright M = T^*X$ relied on Whittaker induction.
This requires the following data:
\begin{enumerate}
	\item A reductive subgroup $\check{H}_X \subset \check{G}$
	\item A symplectic representation $\check{H}_X \curvearrowright S_X$
	\item A nilpotent element $f \in \check{\gf}$
\end{enumerate}
satisfying some compatibility conditions.
In our first example, the first and second pieces of data were trivial.
In our second example, the third piece of data was trivial.
Let's consider an example where all three are nontrivial.

\subsection{Example 3: $G = \SO_5$, $X = \SO_5 / \SO_4$}

Let's consider $G = \SO_5$ acting on $T^*X$ where 
\[
	X = G / \SO_4 = \{ x_1 x_5 + x_2 x_4 + x_3^2 = 1 \}.
\]
If we worked over $\RR$, $X$ would be the $4$-sphere.
Note that this fits into the same pattern of $\SO_{2n+1} \curvearrowright T^*(\SO_{2n+1}/\SO_{2n})$ as our second example (since $\PGL_2 = \SO(3)$).

We can write the Lie algebra $\sof_5 = \{ X^T J + JX = 0 \}$ where $J$ is the $5 \times 5$ matrix with ones on the skew diagonal and zeroes elsewhere.
The maximal torus $T$ has Lie algebra $\tf$ consisting of matrices of the form
\[
	\begin{bmatrix}
		y_1 & & & & \\
		& y_2 & & & \\
		& & 0 & & \\
		& & & -y_2 & \\
		& & & & -y_1
	\end{bmatrix}.
\]
Let $\epsilon_i \in \tf^*$ send such a matrix to $y_i$.
The simple roots are $\epsilon_1 - \epsilon_2$ and $\epsilon_2$.
The other positive roots are $\epsilon_1$ and $\epsilon_1 + \epsilon_2$.
Here $L(\epsilon_1)$ is the standard $5$-dimensional representation.

To find $\check{H}_X$, we start by finding the coweight lattice $\Lambda_{\check{H}_X}$, which in this case is $\ZZ \epsilon_1$.
To see this, note that $k[X]^U = \oplus_\lambda L(\lambda)^U \otimes L(\lambda^*)^{\SO_4}$.
Then we can see that all nonzero summands come from $L(n \epsilon_1)$ for some $n \in \ZZ$.
Another proof involves noting that the elements $b$ of the standard Borel subalgebra act on the function $x_5 \in k[X]$ via multiplication by $\epsilon_1(b)$.

The valuation cone consists of the half of $\Lambda_{\check{H}_X} \otimes \QQ$ containing the valuation $v: k(X) \to \ZZ$ measuring the order of vanishing at infinity.
Thus $\check{H}_X$ has a root, so it must be a copy of $\SL_2$ or $\PGL_2$ corresponding to the coroot $\epsilon_1$ of $\Sp_4$.
Looking at $\Sp_4$ shows that $\check{H}_X$ must be a copy of $\SL_2$.

We're pressed for time, so we'll just list the remaining data.
The symplectic representation $S_X$ is $L(1)$.
The parabolic $P_X$ is the parabolic corresponding to $\epsilon_2$.
Since $\epsilon_1$ and $\epsilon_2$ are orthogonal, we have $(\SL_2)_{\epsilon_1} \times (\SL_2)_{\epsilon_2} \curvearrowright \Sp_4$.
Similar phenomena hold as we continue the pattern.

\section{11/16 (Jeremy Taylor) -- Concluded}

This is the last talk for the next semester.
We will meet again on Thursday of RRR week to go for a hike.

\subsection{Review}

For a hyperspherical variety $G \curvearrowright M = T^*X$, we can construct a dual hyperspherical variety $\check{G} \curvearrowright \check{M}$ via Whittaker induction.
These have many interesting conjectural relations, e.g.\ the relative local Langlands conjecture:
\[
	\Sh(X_\Kc / G_\Oc) \simeq \QCoh(\check{M} / \check{G})
\]

We have discussed three examples so far:
\begin{itemize}
	\item $G \curvearrowright \pt$ is dual to $\check{G} \curvearrowright T^*(\check{G} / (\check{N},f)) = \check{G} \times (f + \check{g}^e)$.
	\item $\PGL_2 \curvearrowright T^*(\PGL_2 / T)$ is dual to $\SL_2 \curvearrowright T^* L(1)$.
	\item $\SO_5 \curvearrowright T^*(\SO_5 / \SO_4)$ is dual to the hyperspherical $\Sp_4$-variety with Whittaker induction data $\check{H}_X = (\SL_2)_{\epsilon_2} \subset \Sp_4$, $S_X = L(1)$, and
		\[
			f = \begin{bmatrix}
				0 & 0 & 0 & 1 \\
				0 & 0 & 0 & 0 \\
				0 & 0 & 0 & 0 \\
				0 & 0 & 0 & 0 \\
			\end{bmatrix} \in (\slf_2)_{\epsilon_1} \subset \spf_4.
		\]
		The highest weights of $S_X$ are given by $B$-invariant but not $G$-invariant divisors on $X$.
		Write $X = \{ x_1 x_5 + x_2 x_4 + x_3^2 = 1 \}$.
		Taking $D = \{ x_5 = 0 \}$, we know that $x_5$ has $T$-weight $\epsilon_2$, and we have $v_D(x_5) = 1 = \ip{\lambda}{\check{\epsilon}_2}$.
\end{itemize}

\subsection{Example 4: $G = H \times H$, $M = T^*H$}

Considering $G = H \times H$ acting on itself by left and right multiplication, we will show that the dual is $\check{H} \times \check{H} \curvearrowright T^*(\check{H} \times \check{H} / \check{H})$, where we embed $\check{H} \hookrightarrow \check{H} \times \check{H}$ via $h \mapsto (h, w_0\inv (h\inv)^T w_0)$.
The second factor of the embedding here is the ``Cartan involution.''
We may identify $T^*(\check{H} \times \check{H} / \check{H})$ with $\check{H} \times \check{\hf}^*$.

The relative local Langlands conjecture for this pair will recover the geometric Satake theorem for $H$.
The conjecture states
\[
	\Sh(H_\Kc / (H_\Oc \times H_\Oc)) \simeq \QCoh(\check{\hf}^* / \check{H}),
\]
and we can unfold everything to obtain the usual statement of (derived) geometric Satake.

Let's explain the computation in more detail.
Note
\[
	k[H]^{N \times N} = \oplus_{\lambda} L(\lambda)^N \otimes L(-w_0(\lambda))^N,
\]
where each summand is one-dimensional with $T \times T$-weight $(\lambda, -w_0 \lambda)$.
Thus the coweights $\Lambda_{\check{H}_X} = \Lambda_{\check{H}}$ embed into $\Lambda_{\check{H}} \times \Lambda_{\check{H}}$ via $\lambda \mapsto (\lambda, -w_0 \lambda)$.

Let's figure out the roots first in the case of $\PGL_2$.

\begin{ex}
	For $H = \PGL_2$, we may consider the usual embedding $G \hookrightarrow \PP^3$, viewing $\PGL_2$ as an open subvariety with complement $X_1 = \{ ad - bc = 0 \} = G/B \times G/B^-$.
	The function
	\[
		f_\alpha = \frac{c^2}{ad-bc} \in k[H]
	\]
	is $U \times U$-invariant and has $T \times T$-weight $(\alpha, \alpha)$ for $\alpha$ the root of $\PGL_2$.
	We have $v_{X_1}(f_\alpha) = -1$, so the valuation cone $V$ is a ray in $\Lambda_{\check{H}} \otimes \QQ$.
\end{ex}

For general $H$, we must use the wonderful compactification to understand the behavior at infinity.

\subsection{The wonderful compactification}

Let $G$ be an adjoint reductive group, and let $V$ be an irrep of $H$ with highest weight $\lambda$ and highest weight vector $v_\lambda$.
Assume $\lambda$ is regular.
The representation gives us a natural map $G \to \PP(\End(V))$.
The \emph{wonderful compactification} is $X = \ol{\psi(G)}$, which is independent of $V$ so long as $\lambda$ is regular.

Let 
\[
	\PP_0 = \{ [A] \in \PP(\End(V)) \, | \, v_\lambda^* A v_\lambda \neq 0 \}.
\]
Let $Z = \ol{\psi(T)} \cap \PP_0 = \CC^r$ where $r = \rk T = \# \Delta$ (where $\Delta$ is the set of simple roots).
Note that $\PP_0 \cap X = U^- Z U$.

\begin{rmk}
	The intuition is that we want to understand the points in the closure of $T$ at $\infty$, as $\ol{\psi(T)}$ is toric (and therefore accessible in terms of combinatorics).
	There are theorems telling us that ``the geometry of $G$ at infinity is smoothly equivalent to that of $T$ at infinity.''
	The closure $\ol{\psi{T}}$ is given by the toric variety with fan consisting of the Weyl chambers.
	Here $\PP_0$ should pick out a single Weyl chamber.
\end{rmk}

Note the following properties:
\begin{enumerate}
	\item The $T$-orbits in $Z$ are given by $Z_I = \{ \alpha_i(z) = 0 \forall i \in I\}$, where $I$ ranges over all subsets of $\Delta$.
	\item The $G \times G$-orbits are $X_I \subset X$ indexed by the same sets $I$ and satisfying $X_I \cap Z = Z_I$.
	\item We have $X \setminus G = \cup_{i \in \Delta} X_i$.
	\item The orbit $X_I$ is naturally a $L_I / Z$-bundle over $G/P_I \times G/P_I^-$
	\item The closures $\ol{X}_I$ are naturally bundles over $G/P_I \times G/P_I^-$ with fibers given by the wonderful compactifications of $L_I / Z$.
\end{enumerate}

We can use this last property to inductively compute the divisors / roots appearing in the construction of the Whittaker data above.

\end{document}
