\documentclass{article}

\usepackage{notes}

\title{GRT Seminar Fall 2024 -- Categorical Representation Theory}

\begin{document}

\maketitle

\begin{abstract}
	This semester, the GRT Seminar will focus on ``categorical representation theory.''
	A good reference is ``An informal introduction to categorical representation theory and the local geometric Langlands program'' by Gurbir Dhillon.
\end{abstract}

\tableofcontents

\section{9/5 (David Nadler) -- Introduction}

Today will be fairly informal -- we'll just talk about basic definitions and the example of Lusztig's character sheaves.

\subsection{What is a representation of a group?}

Consider the following well-studied classes of groups:
\begin{itemize}
	\item Finite groups
	\item Groups over local fields (e.g. Lie groups, $p$-adic groups, \dots)
\end{itemize}
Many interesting examples (e.g.\ finite groups of Lie type) arise as the points of an algebraic group.
These algebraic groups often have a \emph{geometric} interpretation.

Representations of groups are interesting for two reasons:
\begin{itemize}
	\item They allow us to understand abstract groups concretely.
	\item They allow us to study the symmetries of objects we already cared about.
\end{itemize}
We can describe a representation of a group as a homomorphism $G \to \Aut_{\Cc}(c)$ for some object $c$ of a category $\Cc$.

\begin{ex}
	In the most classical setup, we take $\Cc = \Vect_\CC$, the category of complex vector spaces.
	Here $\Aut_{\Vect_\CC}(V) = \GL(V)$, the \emph{general linear group}.
	Note that, at some point, we have to account for the presence of $\CC$ here.
	We also have to ask: do we consider continuous representations? Algebraic representations?
\end{ex}

Let's try to reformulate this in a way that makes the dependence on choices clearer.
Given a group $G$ and a coefficient field $k$, we can construct a \emph{coalgebra of functions} $\Funop_k(G)$ and an \emph{algebra of distributions} $\Dist_k(G)$.
For a finite group $G$, these are the same.
A representation of $G$ can be interpreted as either a \emph{comodule} over $\Funop_k(G)$ or a \emph{module} over $\Dist_k(G)$.

The choices we make are:
\begin{itemize}
	\item The coefficient field $k$
	\item The types of representation (i.e.\ types of ``functions'' or ``distributions'' considered)
\end{itemize}

Let's see this in an example.

\begin{ex}
	What are representations of $S^1$?
	\begin{itemize}
		\item One answer comes by viewing $S^1$ as a compact Lie group, so a representation $V$ decomposes as $\oplus_{n \in \ZZ} V_n$, where $e^{i \theta}$ acts on $V_n$ by multiplication by $e^{in\theta}$.
		\item If we use homotopy theory, all we can see about $S^1$ is the homotopy type -- in particular, $S^1$ and $\CC^\times$ should ``have the same representation theory.''
			The correct notion of a ``distribution'' here is $\Dist_\CC(S^1) = C_{-\bullet}(S^1; \CC) \simeq \CC[\epsilon]$, where $\deg \epsilon = -1$ and $\epsilon^2 = 0$.
			Thus representations of $S^1$ are $\CC[\epsilon]$-modules.

			Representations in the ordinary category $\Vect_\CC$ are not interesting -- $\epsilon$ must always act trivially.
			However, if we consider representations in the dg-category of $\CC$-dg-vector spaces, we get a more interesting answer -- the category of dg-representations of $S^1$ is $\CC[\epsilon]$-dg-modules.
	\end{itemize}
\end{ex}

\subsection{Why study categorical representation theory?}

Categorical representation theory is based on an old ``miracle.''

Consider the representation theory of finite groups (especially those of Lie type).
There are two classical ways to construct representations of such groups:
\begin{itemize}
	\item Induction from abelian groups (e.g.\ tori)
	\item Prayer (e.g.\ ``cuspidal representations'')
\end{itemize}
Lusztig realized a good way to make sense of cuspidal representations.
Recall that representations of a finite group $G$ are equivalent to characters of class functions on $G$.
So we can rephrase our problem: how do we construct characters of cuspidal representations?
The trick is to instead construct character \emph{sheaves} on the corresponding algebraic groups.
Once we have the character sheaves, we can recover the character functions by decategorification.

The ``miracle'' is that character sheaves can be accessed via induction from tori!
One can explain this by appealing to topological field theory: there's a tradeoff between asking seemingly ``easy'' questions about complicated manifolds and asking seemingly ``hard'' questions about simpler manifolds (e.g.\ points).
In many cases, one can answer questions about the former by finding the right questions to ask about the latter.

\section{9/12 (Raymond Guo) -- $D$-Modules}

\subsection{(David) -- Brief orientation}

Let $G$ be an algebraic group, e.g.\ $\GL_n$.
When considering a ``type of representations,'' we need to choose a notion of ``group algebra'' / ``distributions.''
In categorical representation theory, we take the ``group algebra'' to be a monoidal category.

\begin{ex}
	The fundamental example to consider is the category $D(G)$ of $D$-modules on $G$.
\end{ex}

We can find ``representations'' of $D(G)$ by considering $G$-varieties $X$.
The action of $G$ on $X$ gives an action of $D(G)$ on $D(X)$.

\begin{ex}
	The fundamental example is $X = G / B$, the flag variety of $G$.
\end{ex}

\subsection{(Raymond) -- Weyl algebras}

Consider a homogeneous linear ordinary differential equation
\[
	\sum_{i=0}^N p_i(t) \partial^i f(t) = 0.
\]
Here $\partial = d/dt$.
We'd like to understand this algebraically.

To accomplish this, we work in the Weyl algebra $D_{\AA^1} = \CC\langle t, \partial \rangle / ([\partial, t] - 1)$, where the relation comes from the Leibniz rule
\[
	\partial (f \cdot t) = 1 \cdot f + t \partial f.
\]
In higher dimensions, we use the Weyl algebra
\[
	D_{\AA^n} = \frac{\CC\langle t_1, \dots, t_2, \partial_1, \dots, \partial_n \rangle}{([t_i, t_j], [\partial_i, \partial_j], [\partial_i, \partial_j], [\partial_i, t_i] - 1)}.
\]
Here the relations are imposed for $i \neq j$.
This allows us to work algebraically with linear PDEs.

Let's consider modules over $D_{\AA^1}$.

\begin{ex}
	Letting $\partial$ act by $d / dt$, we see that $\CC[t]$ is a $D_{\AA^1}$-module.
\end{ex}

\begin{ex}
	The module $D_{\AA^1} / (D_{\AA^1} \cdot t)$ can be understood as $\CC[\delta]$ where $t$ acts by $t \delta^n = -n \delta^{n-1}$.
\end{ex}

David gave an exercise of finding all simple modules for $D_{\AA^1}$.

\subsection{$D$-modules on general varieties}

Let $X$ be a smooth affine variety over a field $k$ of characteristic zero.
The endomorphism algebra $\Hom_k(\Oc_X, \Oc_X)$ contains $\Oc_X(X)$ (embedded via $f \mapsto (a \mapsto f \cdot a)$) as well as 
\[
	\Vectrm_X = \{ F \in \Hom_k(\Oc_X, \Oc_X) \,|\, F(fg) = F(f)g + f F(g) \}.
\]
Taken together, these generate a subalgebra $D_X = k\langle \Oc_X, \Vectrm_X\rangle \subset \Hom_k(\Oc_X, \Oc_X)$.

\begin{ex}
	One can check that $D_{\AA^n}$ agrees with the presentation given in the previous section.
\end{ex}

All of the above discussion globalizes to give a quasicoherent sheaf of $k$-linear algebras $\Dc_X$ on any (not necessarily affine) algebraic variety $X$.
This is locally generated by $\Oc_X$ and the sheaf of vector fields / derivations $\Vectc_X$.
A $D$-module on such an $X$ is a sheaf of $\Dc_X$-modules which is quasicoherent as an $\Oc_X$-module.

\begin{ex}
	For any $X$, $\Oc_X$ may be viewed as a $D$-module by letting $\Dc_X$ act in the usual way.
	Because $\Vectc_X$ kills the constant function $1$, which generates $\Oc_X$, we see $\Oc_X \simeq \Dc_X / \Dc_X \cdot \Vectc_X$.
	One may think of this as the trivial vector bundle equipped with the standard flat connection.

	Maps $\Oc_X \to \Fc$ correspond to ``flat'' sections of $\Fc$, i.e.\ sections of $\Fc$ on which $\Vectc_X$ acts trivially.
	This can be interpreted as ``solutions of a system of PDEs.''
\end{ex}

\begin{ex}
	Let $x \in X$ be a closed point with maximal ideal $\mf_x$.
	Then $\delta_x = \Dc_X / \Dc_X \cdot \mf_x$ is a $D$-module on $X$.
	(Some authors twist this by a line.)
	Note that $\delta_x$ is supported (set-theoretically, as a quasicoherent sheaf) only at $x$.

	If we take a more sophisticated perspective, $\delta_x$ is supported on every infinitesimal neighborhood of $x$.
	Maps $\delta_x \to \Fc$ correspond to sections $f$ of $\Fc$ which are annihilated by $\mf_x$.
	Taking the image of $\partial_i$ under such a map recovers (up to a sign) the derivatives of $f$.
\end{ex}

\begin{ex}
	David suggested that, given a smooth variety $X$ and a smooth subvariety $Y \subset X$, we can construct a $D$-module of ``delta functions supported on $Y$.''
\end{ex}

\subsection{Functoriality}

Suppose $f: X \to Y$ is a morphism of algebraic varieties.
We would like to be able to push forward and pull back $D$-modules along $\Fc$.

\begin{dfn}
	For a morphism $f: X \to Y$, let $\Dc_{X \to Y} = \Oc_X \otimes_{\Oc_Y} \Dc_Y$.
	Given a $D$-module $\Fc$ on $Y$, we define $f^* \Fc = \Dc_{X \to Y} \otimes_{\Dc_Y} \Fc$.
\end{dfn}

As a quasicoherent sheaf, $f^* \Fc$ is just the usual $f^* \Fc := \Oc_X \otimes_{\Oc_Y} \Fc$, but now we have a compatible $\Dc_X$-action.

\begin{rmk}
	One must be very careful about handedness: do we consider \emph{left} or \emph{right} $D$-modules?
	Because $\Dc_X$ is noncommutative, these two are different.
	We have been working with left $D$-modules so far.
	A standard left $D$-module is $\Oc_X = \Dc / \Dc \cdot \Vectc_X$, with sections looking like $f$.
	Stated differently, functions are a left $D$-module.

	A standard right $D$-module is $\Omega_X^{\toprm}$, with sections looking like $f \cdot dx_1 \dots dx_n$ on charts.
	Here $\Dc_X$ acts (in a chart) by Lie derivatives.
	We can write $\Omega_x^{\toprm} = \det(T_X)^{\pm 1} \otimes (\Vectc_X \cdot \Dc_X) \backslash \Dc$.
	Stated differently, distributions are a right $D$-module.
	Note that we need to work with distributions to be able to pushforward.

	We can convert between left and right $D$-modules by $\Fc \leftrightarrow \Omega_X^{\mathrm{top}} \otimes_{\Dc_X} \Fc$.
\end{rmk}

\section{9/19 (Raymond Guo) -- Continued}

\subsection{Refresher}

Suppose $X = \Spec A$ is a smooth affine variety.
Recall that we have a module of vector fields / derivations
\[
	\Vectrm_X = \{ v: \Oc_X \to \Oc_X \,|\, v(fg) = v(f)g + f v(g) \}.
\]
the module of K\"ahler differentials $\Omega_{A/k}$ is the universal receiver of $k$-linear derivations from $A$.
That is,
\[
	\Der_k(A, M) \cong \Hom_A(\Omega_{A/k}, M).
\]
In particular, $\Vectrm_X \cong \Der_k(A, A) \cong \Hom_A(\Omega_{A/k}, A)$.

We can globalize / sheafify all of these definitions to obtain sheaves of vector fields and differentials.
Thus, in the following, we let $X$ be any smooth variety.

\subsection{Distributions}

Let $n = \dim X$, and let $\Omega_X^\toprm = \wedge^n \Omega_X$.
Given a vector field $\xi$ and a top form $\omega$, we define the \emph{Lie derivative} $\Lie(\xi)(\omega)$ by
\[
	\Lie(\xi)(\omega)(\xi_1, \dots, \xi_n) = \xi\big(\omega(\xi_1, \dots, \xi_n)\big) - \sum_{i=1}^n \omega(\xi_1, \dots, [\xi, \xi_i], \dots).
\]
This satisfies
\begin{align*}
	(\Lie(f\xi))(\omega) &= (\Lie(\xi))(f\omega) \\
	(\Lie(\xi))(f\omega) &= (\xi(f)) \omega + f \Lie(\xi)(\omega) \\
	(\Lie([\xi_1, \xi_2]))(\omega) &= -[\Lie(\xi_1), \Lie(\xi_2)] \omega.
\end{align*}
Thus we obtain a \emph{right} action of $\Dc_X$ on $\Omega_X^\toprm$.

\begin{rmk}
	David explains this as follows.
	In calculus, functions $f(t)$ have a left action by differential operators.
	Distributions $g(t) dt$ can be paired with functions via integration.
	Because functions have a left action by differential operators, the dual of functions (i.e.\ distributions) must admit a right action.
	Ansuman pointed out that the presence of minus signs in the above equations can be understood from this perspective: it's just the usual integration by parts!
\end{rmk}

\subsection{Passing between left and right}

There is a pair of equivalences
\[
	\Omega_X^\toprm \otimes_{\Oc_X} (-): \Dc_X\dMod \leftrightarrow \Dc_X\op\dMod :(-) \otimes_{\Oc_X} (\Omega_X^\toprm)\inv
\]
Here $\Dc_X$ acts on $M \otimes_{\Oc_X} \Omega_X^\toprm$ (on the right) via something like $\xi \cdot (m \omega) = -\xi(m) \omega + m \xi(\omega)$.

\begin{rmk}
	David explained that we can consider $\Diffc(\Oc_X, \Omega_X^\toprm)$, the sheaf of ``differential operators from $\Oc_X$ to $\Omega_X^\toprm$.''
	The left action of $\Dc_X$ on $\Oc_X$ and the right action of $\Dc_X$ on $\Omega_X^\toprm$ gives us a right $\Dc_X \otimes \Dc_X$-action on $\Diffc(\Oc_X, \Omega_X^\toprm)$ (so we may view this as a $(\Dc_X\op, \Dc_X)$-bimodule).
	The functor $\Dc_X\dMod \to \Dc_X\op\dMod$ can also be described as $\Diffc(\Oc_X, \Omega_X^\toprm) \otimes_{\Dc_X} (-)$.
	The inverse functor admits an analogous description.
\end{rmk}

\subsection{Functoriality and convolution}

Let's focus on the affine case, so everything is a module.
Given $f: X \to Y$, let $\Dc_{X \to Y} = \Oc_X \otimes_{\Oc_Y} \Dc_Y$.
We may define a pullback of $\Dc$-modules via
\[
	f^* M = \Dc_{X \to Y} \otimes_{\Dc_Y} M.
\]
This can also be interpreted as tensoring with the bimodule $\Diffc(f^* \Oc_Y, \Oc_X)$.

To define a pushforward of $\Dc$-modules, we need to pass from left to right $\Dc$-modules.
We define
\[
	f_*(M) = (\Omega_X^\toprm \otimes_{\Oc_X} M) \otimes_{\Dc_X} \Dc_{X \to Y}) \otimes_{\Oc_Y} (\Omega_X^\toprm)\inv
\]
As above, this can be interpreted as tensoring with a bimodule (in this case $\Diffc(f^* \Omega_Y^\toprm, \Omega_X^\toprm)$).

We may use these to define convolution products.

Let $G$ be an algebraic group (with multiplication $m: G^2 \to G$) over $\CC$.
At closed points $a \in G$, we may construct $\delta$-modules $\delta_a = \Dc_G / \Dc_G \cdot \mf_{G,a}$.
The convolution product of $\delta_a$ and $\delta_b$ is
\[
	m_*\big(\pi_1^* \delta_a \otimes_{\Oc_{G \times G}} \pi_2^* \delta_b\big) = \delta_{m(a,b)}
\]
In general, the convolution product defines a monoidal structure $\star$ on $\Dc_G\dMod$.
The monoidal category $(\Dc_G\dMod, \star)$ is the \emph{smooth categorical group algebra}.
Actions of $G$ on a space (e.g.\ $G / B$) give rise to module categories for the smooth categorical group algebra.

\section{9/26 (Brian Yang) -- Beilinson-Bernstein Localization}

Let $G$ be a semisimple Lie group with Lie algebra $\gf$.
Let $B \subset G$ be a Borel subgroup, so $G / B$ is the flag variety.

Beilinson-Bernstein localization gives an adjunction between:
\begin{itemize}
	\item Representations of $\gf$ with central character $0$ (this will be defined later).
	\item $D$-modules on $G / B$.
\end{itemize}
We can go from $D$-modules to representations by taking global sections.
The left adjoint of this is Beilinson-Bernstein localization.

Really, we should think of our representations as living in category $\Oc$, which includes the finite dimensional representations and the Verma modules.
This essentially consists of ``highest weight'' representations which can possibly be infinite-dimensional.
We'd like to connect this with $D$-modules on the flag varieties.

\subsection{Review of $\slf_2$}

Let $\gf = \slf_2(\CC)$, with basis
\[
	e = \begin{bmatrix}
		0 & 1 \\
		0 & 0
	\end{bmatrix}, \hspace{2em}
	f = \begin{bmatrix}
		0 & 0 \\
		1 & 0
	\end{bmatrix}, \hspace{2em}
	h = \begin{bmatrix}
		1 & 0 \\
		0 & -1 
	\end{bmatrix}.
\]
Let $V$ be a representation of $\gf$.
We say that $v \in V$ is a \emph{heighest weight} vector of weight $n \geq 0$ if, letting $v_k = \frac{1}{k!} f^k v$ for $k \geq 0$, we have
\[
	h v_k = (n - 2k) v_k, \hspace{2em} f v_k = (k + 1) v_{k+1}, \hspace{2em} e v_k = (n - k + 1) v_{k-1} \textrm{ with } e v_0 = 0.
\]
In particular, the weight $n$ is the eigenvalue of $h$ corresponding to the eigenvector $v$.

For fixed $n$, we define the \emph{Verma module} $M(n) = \Span_\CC \{ v_0, v_1, \dots \}$ as the free (infinite-dimensional) $\gf$-representation with a highest weight vector of weight $n$.
Letting $\bfr = \Span_\CC \{e, h\}$, we can also define $M(n)$ as $\Uc \gf \otimes_{\Uc \bfr} \CC_n \cdot v_0$.

We may view $M(-n-2)$ as $\Span_\CC \{v_{n+1}, v_{n+2}, \dots\} \subset M(n)$.
The quotient $V(n) = M(n) / M(-n-2)$ is the unique irrep of dimension $n + 1$.
We call the resolution
\[
	\SES{M(-n-2)}{M(n)}{V(n)}
\]
the \emph{Bernstein-Gelfand-Gelfand resolution}.

\subsection{$D$-modules on $\PP^1$ and representations}

Let $G = \SL_2(\CC)$.
The flag variety of $G$ is $X = \PP^1$.
We may write $X = U_\infty \cup U_0$ where $U_\infty = \{ [w : 1] \}$ and $U_0 = \{ [1 : z] \}$.
The sheaf $\Dc_X$ satisfies $\Dc_X(U_0) = \CC\langle z, \partial_z \rangle$ and $\Dc_X(U_\infty) = \CC\langle w, \partial_w \rangle$.

We have an action of $G$ on $X$ by
\[
	\begin{bmatrix}
		a & b \\
		c & d
	\end{bmatrix}
	[x : y] = [ax + by : cx + dy].
\]
This induces a Lie algebra homomorphism $\gf \to \Vectrm(X)$.

In the $U_0$ chart, we can compute
\begin{align*}
	e \cdot f(z) &= \frac{d}{dt}\bigg|_{t=0} \exp(te) f(z) \\
		     &= \frac{d}{dt}\bigg|_{t=0} f(\exp(-te) z) \\
		     &= \frac{d}{dt}\bigg|_{t=0} f\left(\begin{bmatrix}1 & -t\\ 0 & 1\end{bmatrix} [1 : z]\right) \\
		     &= z^2 f'(z).
\end{align*}
That is, $e$ acts by $z^2 \partial_z$.

We can make similar computations for other elements / charts: 
\begin{itemize}
	\item In $U_0$, we have $e \mapsto z^2 \partial_z$, $f \mapsto -\partial_z$, and $h \mapsto 2z\partial_z$.
	\item In $U_\infty$, we have $e \mapsto -\partial_w$, $f \mapsto w^2 \partial_w$, and $h \mapsto -2w\partial_w$.
\end{itemize}
These can all be interpreted geometrically: $e$ is a vector field with a zero of order two at $0$, $f$ is a vector field with a zero of order two at $\infty$, and $h$ generates the scaling action of $\CC^\times$.

Using the homomorphism $\gf \to \Vectrm(X)$, we see that the global sections of any $D$-module on $\PP^1$ give a representation of $\gf$!

\begin{ex}
	Letting $\Oc_X$ be the structure sheaf with the usual action of $\Dc_X$, we see that $\Gamma(X, \Oc_X) = \CC$ consists of constant functions.
	Because $\Vectrm(X)$ acts trivially on constant functions, we get $\Gamma(X, \Oc_X) \cong V(0)$ as $\gf$-representations.
\end{ex}

\begin{ex}
	Let
	\[
		N = \left\{ \begin{bmatrix}1 & *\\ 0 & 1\end{bmatrix}\right\}
	\]
	act on $X$ by restricting the $G$-action to $N$.
	This has two orbits: $\{0 \}$ and $U_\infty$.
	We can pushforward the structure sheaves of the orbits to $X$ and obtain $D$-modules on $X$.

	Let $\Mc_{\pt} = i_{\pt *} \Oc_{\pt}$.
	To compute this, note that 
	\[
		\Dc_{\pt \to U_0}(U_0) = \CC[z] / (z) \otimes_{\CC[z]} \CC\langle z, \partial_z\rangle = \oplus_{i \geq 0} \CC \cdot \delta \partial_z^i
	\]
	where $\delta$ denotes a (formal) $\delta$-function at $0$.
	Since $U_0 \cong \AA^1$ has trivial canonical bundle, we also get $\Dc_{U_0 \leftarrow \pt}(U_0) = \oplus_{i \geq 0} \CC \cdot \partial_z^i \delta$.
	It follows that
	\[
		\Gamma(X, \Mc_{\pt}) = \Mc_{\pt}(U_0) = \oplus_{i \geq 0} \CC \cdot \partial_z^i \delta = M(-2).
	\]
	Here the Lie algebra action is computed by taking the basis $m_k = \frac{(-1)^k}{k!} \partial_z^k \delta$ and using the fact that $z \delta = 0$.

	David made some comments about why we can think of $\delta$ as a $\delta$-function: because $z \delta = 0$, any map $\Mc_{\pt} \to \Fc$ must send $\delta$ to some $f$ satisfying $zf = 0$.
\end{ex}
	
We'd also like to understand the pushforward of the structure sheaf of the open orbit.
For this, it's useful to start with some general remarks.
If $i: U \hookrightarrow X$ is an open immersion and $\Mc$ is a $\Dc_U$-module, then $i_* \Mc(V) = \Mc(U \cap V)$.
The $\Dc_X$-module structure on $i_* \Mc$ agrees with the $\Dc_U$-module structure on $\Mc$.
That is, pushforwards of $D$-modules along open immersions are just the usual pushforwards with the ``obvious'' $D$-module structure!

\begin{ex}
	Let $\Mc_{\mathrm{open}} = i_{U_\infty *} \Oc_{U_\infty}$.
	Then $\Gamma(X, \Mc_{\mathrm{open}}) = \Gamma(U_\infty, \Oc_{U_\infty}) = \CC[w]$ with its usual $\Dc_{U_\infty}$-module structure.
	Choosing the basis $n_k = (-1)^k w^k$, we can compute the Lie algebra action and see that $\Gamma(X, \Mc_{\mathrm{open}}) = M(0)^\vee$, the dual Verma module.
\end{ex}

\end{document}
